\documentclass[12pt,a4paper]{report}
\textwidth=450pt\oddsidemargin=0pt


%\documentclass[a4paper,12pt,twoside,openright]{report}
\usepackage[T1]{fontenc}
\usepackage[utf8]{inputenc}
\usepackage[english]{babel}
\usepackage{amssymb}
\usepackage{amsmath}
\usepackage{amsthm}
\usepackage{pgfplots}
\usepackage{newlfont}
\usepackage{enumitem}
\usepackage{mathtools}
\usepackage{faktor}
\usepackage{pgfplots}
\usepackage{comment}
\usepackage{mathrsfs}
\usepackage{tikz}
    \usetikzlibrary{graphs}

\usepackage{natbib}

\DeclarePairedDelimiter\ceil{\lceil}{\rceil}
\DeclarePairedDelimiter\floor{\lfloor}{\rfloor}

%\textwidth=450pt\oddsidemargin=0pt

\usepackage{amssymb,stackengine}
\newcommand\ledot{\mathrel{\ensurestackMath{%
  \stackengine{-.5ex}{\lessdot}{-}{U}{c}{F}{F}{S}}}}

%\theoremstyle{definition}
\theoremstyle{plain}

\newtheorem{thm}{Theorem}[chapter]
\newtheorem*{thm*}{Theorem}
\newtheorem*{thmA*}{Theorem A}
\newtheorem*{thmB*}{Theorem B}
\newtheorem{lem}[thm]{Lemma}
\newtheorem{prop}[thm]{Proposition}
\newtheorem{cor}[thm]{Corollary}
\newtheorem{defn}[thm]{Definition}
\newtheorem{rmk}[thm]{Remark}
\newtheorem{es}[thm]{Example}
\newtheorem*{notat*}{Notation}

\pgfplotsset{compat=1.18}

\begin{document}

\begin{titlepage}
\begin{center}
{{\Large{\textsc{Alma Mater Studiorum $\cdot$ Universit\`a di
Bologna}}}} \rule[0.1cm]{15.8cm}{0.1mm}
\rule[0.5cm]{15.8cm}{0.6mm}
{\small{\bf SCUOLA DI SCIENZE\\
Corso di Laurea Magistrale in Matematica }}
\end{center}
\vspace{15mm}
\begin{center}
{\LARGE{\bf Building sets, Chow ring of matroids and group actions}}\\
% \vspace{3mm}
%{\LARGE{\bf IL TEOREMA DI MORSE-SARD}}\\
\end{center}
\vspace{40mm}
\par
\noindent
\begin{minipage}[t]{0.47\textwidth}
{\large{\bf Relatore:\\
Chiar.mo Prof.\\
Roberto Pagaria}}
\end{minipage}
\hfill
\begin{minipage}[t]{0.47\textwidth}\raggedleft
{\large{\bf Presentata da:\\
Marco Bagnara}}
\end{minipage}
\vspace{20mm}
\begin{center}
{\large{\bf Sessione 5\\%inserire il numero della sessione in cui ci si laurea
Anno Accademico 2023-2024}}
\end{center}
\end{titlepage}


\tableofcontents 
% 
% 
% \null\vspace{\stretch{1}}
% \begin{flushright}
% 	\textit{CITAZIONE/DEDICA}
% \end{flushright}
% \vspace{\stretch{2}}\null


% 
% \begin{abstract}
% %SOMMARIO
% \end{abstract}

\chapter*{Introduction}
\addcontentsline{toc}{chapter}{Introduction}

In matroid theory one of the open problems since the 70's were the log-concavity and unimodal conjectures concerning coefficients of the characteristic polynomial associated to a matroid.
A sequence of number $(a_1, \dots, a_n)$ is called unimodal if
\[ a_1 \le \dots \le a_l \ge a_{l+1} \ge \dots \ge a_n \]
and log-concave if
\[ a_{k-1} \, a_{k+1} \le a_k^2\]
for all $k \le n$.
The chromatic polynomial for graph is the polynomial that count all the possible coloration of the graph. The log-concavity and unimodal conjectures for the chromatic polynomial of a graph state that the sequence of the coefficients of the chromatic polynomial is log-concave and unimodal. They were introduced respectively by Read and Heron in 1968 \cite{Rea68} and 1974 \cite{Hoggar1974ChromaticPA}. These conjectures where latex extended from graphs to general matroids, that are objects that generalize graphs, by Rota and Heron in 1971 \cite{RotaCOMBINATORIALT} and 1972 \cite{Her72}.

Given a matroid $\mathcal{M}$ of rank $r+1$ and defined $\omega_k$ as the absolute value of the coefficient of the characteristic polynomial $\Large\chi_{\small\mathcal{M}}$ associated to the matroid $\mathcal{M}$, the log-concavity conjecture state that the sequence of $(\omega_1, \dots, \omega_r)$ is unimodal and log-concave.

In 2018 Adiprasito, Huh and Katz \cite{adiprasito2018hodgetheorycombinatorialgeometries} prove these conjectures using as main tool a certain graded $\mathbb{Z}$-algebra $A = \bigoplus_{k=0}^r A^k$ associated to a matroid, called the Chow ring,  
introduced in 2004 by Feichtner and Yuzvinsky \cite{Feichtner_2004}.
In particular they proved the Kahler package which is composed of three results about the ring $A$ that are the Poincarè duality, Hard Lefschetz and Hodge-Riemann relation. In particular they prove that the sequence $(a_1, \dots, a_k)$ with $a_k = rk_{\mathbb{Z}} A^k$ is unimodal as a consequence of Hard Lefschetz and symmetric as a consequence of Poincarè duality.
In the realizable case the Chow ring is isomorphic to he cohomology of the De Concini-Procesi wonderful model introduced by De Concini and Procesi in \cite{DP1}.

The Chow ring is dependent on the choice of a building set.  In  \cite{adiprasito2018hodgetheorycombinatorialgeometries} they considered the Chow ring associated to the maximal building set and under this assumptions they proved the Kahler package. It was later generalized by Pagaria and Pezzoli for general building sets in 2023 \cite{Pagaria_2023} and by Crowley, Huh, Larson, Simpson, and Wang in 2024 \cite{crowley2024bergmanfanpolymatroid}.

\begin{thmA*}
    Let $\mathcal{M}$ be a matroid with lattice of flats $\mathcal{L}$ of rank $r+1$, and $\mathcal{G}$ a building set that contains$\hat{1}$, the Chow ring 
    \[ \mathcal{A}(\mathcal{L}, \mathcal{G}) = \bigoplus_{k = 0}^r A^k  \]
    satisfy the following:
    \begin{enumerate}
        \item (Poincarè duality) For every $k\le \frac{r}{2}$, there is a perfect $\mathbb{Z}$-bilinear pairing:
        \[ A^k \times A^{r-k} \longrightarrow \mathbb{Z}\]
        \[ (a,b) \longmapsto \deg(a \cdot b)\]
        that induces an isomorphism $A^{r-k} \simeq \text{Hom}_{\mathbb{Z}}(A^k, \mathbb{Z}).$

        \item (Hard Lefschetz) The Chow ring with real coefficients \[ \mathcal{A}_{\mathbb{R}}(\mathcal{L}, \mathcal{G}) = \bigoplus_{k = 0}^r A^k_{\mathbb{R}} \]
        contains a non-empty convex cone $\mathcal{K} \subset A^1_{\mathbb{R}}$ of Lefschetz elements such that $ \forall \omega \in \mathcal{K}$ the map $a \longmapsto a \cdot \omega^{r- 2k}$ is an $\mathbb{R}$-linear isomorphism:
        \[ A^k_{\mathbb{R}} \longrightarrow A^{r-k}_{\mathbb{R}}, \text{ for $k \le \frac{r}{2}$}.\]
        In particular the multiplication by $\omega$ is an injection:
        \[ A^k_{\mathbb{R}} \hookrightarrow A^{k+1}_{\mathbb{R}}, \text{ for $k < \frac{r}{2}$}.\]

        \item (Hodge-Riemann inequality) Each Lefschetz element $\omega$ define a quadratic form 
        \[ a \longmapsto (-1)^k \deg(a \cdot \omega^{r-2k} \cdot a)\]
        on $A^k_{\mathbb{R}}$ that become positive definite upon restriction to the kernel of the map \[A^k_{\mathbb{R}} \longrightarrow A^{r-k+1}_{\mathbb{R}},\] that sends $a \longmapsto a \cdot \omega^{r-2k+1}$.
    \end{enumerate}
\end{thmA*}
In particular, the proof of the log-concavity conjecture follows by the Hodge-Riemann inequality in degree $k=1$.

Given the importance of these techniques it is interesting to study the Chow rings and building sets from which they depends.
Regarding building sets we report a 2024 work by Backman and Danner \cite{backman2024convexgeometrybuildingsets} about the geometry of the family of building sets, in particular they proves that building sets ordered by inclusion form a supersolvable convex geometry.

In this thesis we focused primarily on a 2024 work by Angarone, Nathanson and Reiner \cite{chowringsmatroidspermutation} where they studied how the Poincare duality and Hard Lefschetz properties interact with symmetries. More generally, a group action on a matroid $\mathcal{M}$ is a permutation of the elements that preserve the lattice $\mathcal{L}_{\mathcal{M}}$. They used a monomial base for the Chow ring provided by Feichtner and Yuzvinsky in \cite{Feichtner_2004}, referred as $FY$, and proved, under a certain assumptions for the G-action on $FY$ the existence of G-equivariant relations in the basis.

\begin{thmB*} \label{THMB}
    Let $\mathcal{M}$ be a simple matroid with lattice of flats $\mathcal{L}$, $\mathcal{G}$ a building set that contains $\hat{1}$. Let $G$ a group setwise stabilizing $\mathcal{G}$ acting on $\mathcal{M}$ as in Section 3.2. Then there exists:
    \begin{enumerate}
        \item $G$-equivariant bijections $\pi : FY^k \longrightarrow FY^{r-k}$, for $k \le \frac{r}{2}$;

        \item $G$-equivariant injections $\lambda : FY^k \longrightarrow FY^{k+1}$, for $k < \frac{r}{2}$.
    \end{enumerate}
\end{thmB*}


To prove this result they used symmetric chain decompositions of a poset. A symmetric chain in a ranked poset is a chain that is symmetric with respect to the middle degree of the poset. The idea is to decompose the $FY$ basis in a disjoint union of symmetric chains, ordered by divisibility, and to define the injections and isomorphisms as functions acting on the elements of the chains in the decomposition.
\newline

This thesis is divided into tree chapters. In the first one we firstly introduce lattices, matroids and define the action of a group on matroids. Then we define building and nested sets and we introduce a distance function for atomic lattices.

In Chapter 2 we define the Chow ring of a matroid and the action of the automorphism group of the matroid on the Chow ring. Then we describe a monomial basis of the Chow ring and we state some properties such as the Kahler package. We define the De Concini-Procesi wonderful models, and present their Chow rings.

In the last Chapter we introduce symmetric chains decomposition of a poset and prove the existence of it for the poset of divisors of a natural number.
Then we apply those tools proving the \ref{THMB}. We conclude by showing that the technical condition on the group action is necessary.


\chapter{Lattices, Matroids and Building sets}
In this chapter we first introduce lattices and matroids. Then we define building and nested sets for a lattice an discuss the convexity of the family of nested sets.

\section{Lattices and Matroids}
The aim of this section is to introduce the basics notions of lattice and matroid and state some theorem that will be required in the following.

\begin{defn}
A partial order is a relation $\le$ on a set $\mathcal{P}$ that is reflexive, antisymmetric, and transitive. 
    The pair $(\mathcal{P}, \le)$ is called a partially ordered set or poset. \\
    We denote with $\hat{0}$ the minimum and with $\hat{1}$ the maximum of $\mathcal{P}$ if they exists in $\mathcal{P}$.
    
\end{defn}
\begin{es} 
    
    \begin{enumerate}
    
        \item $(\mathbb{R}, \le)$ is a poset.
        
        \item Consider a set $X$, and define on $\mathcal{P}(X)$, for every $A, B \in 
        \mathcal{P}(X)$ the relation:
        \[ A \le B \iff A \subset B,\]
        then $(\mathcal{P}(X), \le)$ is a poset, and $\mathcal{P}(X)$ is called powerset of $X$.
        
        In Figure \ref{fig:poset123} is depicted the powerset of $X = \{ 1, 2, 3\}$.
        
        \begin{figure} [h!]
        
        \begin{center} \begin{tikzpicture}   

            \node (top) at (0,0) {$\{1, 2, 3\}$};
            \node (n12) at (-3, -2) {$\{1, 2\}$};
            \draw [thick] (top) -- (n12);

            \node (n23) at (3, -2) {$\{2, 3\}$};
            \node (n13) at (0, -2) {$\{1, 3\}$};

            \draw [thick] (top) -- (n13);
            \draw [thick] (top) -- (n23);

            \node (n1) at (-3, -4) {$\{1\}$};
            \draw [thick] (n12) -- (n1);
            \draw [thick] (n1) -- (n13);

            \node (n2) at (0, -4) {$\{2\}$};
            \draw [thick] (n23) -- (n2);
            \draw [thick] (n2) -- (n12);

            \node (n3) at (3, -4) {$\{3\}$};
            \draw [thick] (n13) -- (n3);
            \draw [thick] (n23) -- (n3);

            \node (n0) at (0, -6) {$\emptyset$};
            \draw [thick] (n1) -- (n0);
            \draw [thick] (n2) -- (n0);
            \draw [thick] (n3) -- (n0);
        \end{tikzpicture} \end{center}
        \caption{The Hasse diagram of the power set of $\{1,2,3\}$}
        \label{fig:poset123}
            \end{figure}
        
        \item Consider in $\mathbb{N}$ the divisibility relation, i.e. for all $m, n \in \mathbb{N}$, define:
        \[ m \le n \iff m \, \mid \, n,\]
        then $(\mathbb{N}, \le)$ is a poset.
    \end{enumerate}

\end{es}

\begin{defn}
    Let $(\mathcal{P}_1, \le_1)$, $(\mathcal{P}_2, \le_2)$ two posets, a function $f : \mathcal{P}_1 \longrightarrow \mathcal{P}_2$ is an order embedding if:
    \[ f(x) \le_2 f(y) \iff x \le_1 y.\]
    We call $f$ an isomorphism if $f$ is bijective.
\end{defn}

We introduce here some object that are required in the next section to define building and nested sets. In particular we define product of poset, indecomposable elements and atoms.

\begin{defn}
Given two posets $(\mathcal{P}_1, \le_1)$ and $(\mathcal{P}_2, \le_2)$ we define the product of $\mathcal{P}_1$ and $\mathcal{P}_2$ as the poset $(\mathcal{P}_1 \times \mathcal{P}_2, \le_{\times})$ such that:
\begin{equation}
    (x,y) \le_{\times} (x',y') \iff x \le x' \text{ and } y \le' y'. 
\end{equation}
\end{defn}

Product of poset allow us to define irreducible posets and atoms.

\begin{defn}
    An irreducible poset is a poset $(\mathcal{P}, \le)$ that is not a product of other poset.
\end{defn}

\begin{defn}
Given a poset $(\mathcal{P}, \le)$, we call $x \in \mathcal{P}$ irreducible if the interval $\left[\hat{0}, x \right]$ cannot be decomposed as a nontrivial product.
We denote the set of irreducible elements of $\mathcal{P}$ as $\mathcal{I}(\mathcal{P})$.
Given $x \in \mathcal{P}$ the set $\max \mathcal{I}(\mathcal{P})_{\le x}$ is the set of elementary divisors of $x$, and we denote it with $\mathcal{D}(x)$.
\end{defn}

\begin{defn}
Given a poset $(\mathcal{P}, \le)$, and two elements $x,y \in \mathcal{P}$, we say that $x$ cover $y$, and write $x \gtrdot y$ if:
    \begin{enumerate}
    \item $y < x$;
    \item there are no $w \in \mathcal{P}$ such that $y < w < x$.
    \end{enumerate}
\end{defn}

\begin{defn}
Given a poset $(\mathcal{P}, \le)$ with a minimum element $\hat{0}$, and $x \in \mathcal{P}$ we say that $x$ is an atom if $x \gtrdot \hat{0}$.
\end{defn}

\begin{es}
    \begin{enumerate}
        \item All atoms are irreducible elements.
        
        \item Consider the set $X$, and the poset $(\mathcal{P}(X), \subseteq)$, then every singleton $\{ x \}$ is an atom. In fact consider a set $X = \{ x_1, \dots, x_n \}$, then we have that for every $A \subset X$, we can write $A = \{ x_{i_1} \} \cup \dots \cup \{ x_{i_l} \}$, and for $i = 1, \dots, n$ the only subsets of the singleton $\{ x_i \}$ are $\emptyset$ and the singleton itself, so $\emptyset \subset \{ x_i \}$ and there are no other set between $\emptyset$ and $\{ x_i \}$. So $\{ x_i \}$ cover $\emptyset = \hat{0}$.    
        
        \item Given $n \in \mathbb{N}$, then in $(Div(n), \, \mid \,)$ every prime number is an atom. In fact, given a number $n \in \mathbb{N}$, we can write $n$ as $n = p_1 ^{k_1} \cdots p_l^{k_l}$, where $p_1, \dots, p_l$ are prime number.
        Since for $i = 1, \dots, l$, $p_i$ is divisible only by $1$ ad itself, we necessarily have that $p_i > 1$ and there are no other elements between $p_i$ and $1$. So $p_i \gtrdot 1 = \hat{0}$.
    \end{enumerate}
\end{es}

We are primarily interested in a particular kind of poset called lattices.
Lattices are characterized by to operation: join and meet.
To define these operations we need to introduce upper and lower bound of elements.

\begin{defn}
Given a poset $(\mathcal{P}, \le)$ and two elements $x,y \in \mathcal{P}$, we say that $m \in \mathcal{P}$ is a lower-bound for $x$ and $y$ if $m \le x$ and $m \le y$. \\
We say that $M \in \mathcal{P}$ is an upper-bound for $x$ and $y$ if $M \ge x$ and $M \ge y$.
\end{defn}

\begin{defn}
Given a poset $(\mathcal{P}, \le)$, and two elements $x,y \in \mathcal{P}$, we define, if it exists, the join of $x$ and $y$, denoted by $x \vee y$, as the least upper bound element of $x$ and $y$, i.e. $J \in P$ is the join of $x$ and $y$ if:
    \begin{enumerate}
    \item $J \ge x$ and $J \ge y$;
    \item $\forall w \in \mathcal{P}$ such that $w \ge x$ and $w \ge y$ then $w \le J$.
    \end{enumerate}
We define the meet of $x$ and $y$, denoted by $x \wedge y$, if it exists, the greatest lower-bound of $x$ and $y$, i.e. $M \in \mathcal{P}$ is the meet of $x$ and $y$ if:
    \begin{enumerate}
    \item $M \le x$ and $M \le y$;
    \item $\forall w \in \mathcal{P}$ such that $w \le x$ and $w \le y$ then $w \ge M$.
    \end{enumerate}
\end{defn}

\begin{defn}
A lattice is a poset $(\mathcal{P}, \le)$ where for every $x$, $y$ $\in P$, both $x \wedge y$ and $x \vee y$ exists in $\mathcal{P}$.
\end{defn}

\begin{rmk}
    \begin{enumerate}
        \item If $\mathcal{L}$ is a finite lattice, then $\hat{0}$ and $\hat{1}$ exits in $\mathcal{L}$. It's sufficient to take as $\hat{0}$ the meet of all the elements in $\mathcal{L}$ and as $\hat{1}$ the join of all the elements in $\mathcal{L}$.
        \item The notions of atom, irreducible elements and product are naturally extendable to lattices since lattices are posets.
    \end{enumerate}
\end{rmk}

We introduce here an object we will need later in Section \ref{sec:gbs}.

\begin{defn}
    Let $(\mathcal{L}, \le)$ a lattice and $\le^*$ another order relations on $\mathcal{L}$. Then $\le^*$ is a linear extension of $\mathcal{L}$ when
    \begin{enumerate}
        \item $\le^*$ is a total order;
        \item given $x,y \in \mathcal{L}$ 
        \[ x \le y \implies x \le^* y.\]
    \end{enumerate}
\end{defn}

\begin{es} \label{ed:lattice}
    \begin{enumerate}

        \item Let $X$ be a set then $(\mathcal{P}(X), \subseteq)$ is a lattice with $\hat{1} = X$, $\hat{0} = \emptyset$, where the join and meet are defined a follows:
        \begin{enumerate}
            \item $A,B \in \mathcal{P}(X)$: \[ A \vee B := A \cup B;\]
            \item $A,B \in \mathcal{P}(X)$: \[ A \wedge B := A \cap B.\]
        \end{enumerate}
    
        \item Consider $n \in \mathbb{N}$, define the set $Div(n) := \{ d \in \mathbb{N} \, : \, d \mid n \}$, then $(Div(n), \mid \, )$ is a lattice whose minimum element is $1$, and maximum element is $n$. In this case the meet and join are defined as follow:
        \begin{enumerate}
            \item for $p,q \in Div(n)$
            \[ p \vee q := lcm(p,q);\]

            \item for $p, q \in Div(n)$:
            \[ p \wedge q := gcd(p, q).\]
        \end{enumerate}

        \begin{figure}   

        \begin{center} \begin{tikzpicture}
            
            \node (top) at (0,0) {$36$};
            \node [below left  of=top] (n12) at (-0.3, -0.8)  {$12$};
            \node [below right of=top] (n18) at (0.3, -0.8) {$18$};
            \draw [thick] (top) -- (n12);
            \draw [thick] (top) -- (n18);
            \node [below left  of=n12] (n4) at (-1.3, -2.3)  {$4$};
            \node [below of=top] (n6) at (0,-2){$6$};
            \node [below right of=n18] (n9) at (1.3, -2.3) {$9$};
            \draw [thick] (n12) -- (n4);
            \draw [thick] (n12) -- (n6);
            \draw [thick] (n18) -- (n9);
            \draw [thick] (n18) -- (n6);
            \node  [below of=n12] (n2) at (-1, -3.5) {$2$};
            \node  [below of=n18] (n3) at (1, -3.5) {$3$};
            \draw [thick] (n6) -- (n2);
            \draw [thick] (n4) -- (n2);
            \draw [thick] (n6) -- (n3);
            \draw [thick] (n9) -- (n3);
            \node  [below of=top] (n1) at (0, -5) {$1$};
            \draw [thick] (n2) -- (n1);
            \draw [thick] (n3) -- (n1);
        \end{tikzpicture} \end{center}
        \caption{Hasse diagram of Div(36)}
        
        \end{figure}


        \item Take $n \in \mathbb{N}$, then the set $\Pi_n$ of partition of $\{1, \dots, n \}$ is a lattice with ordered defined, given $\pi = \pi_1 \, | \, \pi_2 \, | \dots | \, \pi_k$ and $\sigma = \sigma_1 \, | \, \sigma_2 \, | \dots | \, \sigma_l$ as
        \[ \pi \le \sigma \iff \forall i \in \{1, \dots, k\}, \, \exists j \in \{1, \dots , l\} \text{ such that } \pi_i \subset \sigma_j , \]
        with maximum element $12\dots n$ and minimum element $1\,|\,2\,| \dots | \, n$. Consider for example the set $\{1, 2, 3, 4\}$, see \ref{figpi4} for the Hasse diagram of $\Pi_4$.

        \begin{figure}   
        \begin{center} \begin{tikzpicture}
            \node (top) at (0,0) {$1234$};
            \node (n1) at (-5, -3)  {$14\,|\,23$};
            \node (n2) at (-3.2, -3) {$1\, | \, 234$};
            \node (n3) at (-1.6, -3) {$124 \, | \, 3$};
            \node (n4) at (0, -3) {$13 \, | \, 24$};
            \node (n5) at (1.6, -3) {$123 \, | \, 4$};
            \node (n6) at (3.2, -3) {$134 \, | 2$};
            \node (n7) at (5, -3) {$12 \, | \, 34$};

            \draw (top) -- (n1);
            \draw (top) -- (n2);
            \draw (top) -- (n3);
            \draw (top) -- (n4);
            \draw (top) -- (n5);
            \draw (top) -- (n6);
            \draw (top) -- (n7);
            
            \node (n8) at (-4, -7)  {$1 \, | \, 23 \, | \, 4$};
            \node (n9) at (-2.4, -7)  {$14 \, | \, 2 \, | \, 3$};
            \node (n10) at (-0.8, -7)  {$1 \, | \, 24 \, | \, 3$};
            \node (n11) at (0.8, -7)  {$13 \, | \, 2 \, | \, 4$};
            \node (n12) at (2.4, -7)  {$12 \, | \, 3 \, | \, 4$};
            \node (n13) at (4, -7)  {$1 \, | \, 2 \, | \, 34$};

            \draw (n1) -- (n8);
            \draw (n1) -- (n9);
            \draw (n2) -- (n8);
            \draw (n2) -- (n10);
            \draw (n2) -- (n13);
            \draw (n3) -- (n9);
            \draw (n3) -- (n10);
            \draw (n3) -- (n12);
            \draw (n4) -- (n10);
            \draw (n4) -- (n11);
            \draw (n5) -- (n8);
            \draw (n5) -- (n11);
            \draw (n5) -- (n12);
            \draw (n6) -- (n9);
            \draw (n6) -- (n11);
            \draw (n6) -- (n13);
            \draw (n7) -- (n12);
            \draw (n7) -- (n13);
            
            \node (bottom) at (0, -10) {$1 \, | \, 2 \, | \, 3 \, | \, 4$};

            \draw (n8) -- (bottom);
            \draw (n9) -- (bottom);
            \draw (n10) -- (bottom);
            \draw (n11) -- (bottom);
            \draw (n12) -- (bottom);
            \draw (n13) -- (bottom);
            
        \end{tikzpicture} \end{center}
        \caption{Hasse diagram of $\Pi_4$}
        \label{figpi4}
        \end{figure}

        \item Let $\mathcal{A}$ a central hyperplane arrangement in a complex vector space $V$, define $\mathcal{L}(\mathcal{A})$ the lattice of all intersection of elements of $\mathcal{A}$ ordered by inverse inclusion, then $\mathcal{L}(\mathcal{A})$ is a geometric lattice. $\mathcal{L}(\mathcal{A})$ is called lattice of flats of $\mathcal{A}$.       
        \end{enumerate}
\end{es}




The aim now is to define a particular kind of lattices, called geometric lattices. To this goal we have to define atomic and semi-modular lattices, we also define ranked lattices and state a characterization of semimodular lattices using the rank function.

\begin{defn}
A lattice $\mathcal{L}$ is said to be atomic if for any $F \in \mathcal{L}$ there exist $a_1, \dots, a_k \in \mathcal{L}$ atoms such that $F = a_1 \vee \dots \vee a_k$.
\end{defn}

\begin{defn}
A lattice $\mathcal{L}$ is semimodular if for any $F, F' \in \mathcal{L}$ the following holds:
\[ F, \  F' \gtrdot F \wedge F' \implies F \vee F' \gtrdot F, \  F'.\]
\end{defn}

\begin{defn}
A lattice $\mathcal{L}$ is called geometric if it is atomic and semimodular.
\end{defn}

\begin{es}
    \begin{enumerate}
        \item Let $n \in \mathbb{N}$, then the partition lattice $\Pi_n$ is a geometric lattice.
        
        \item Let $\mathcal{A}$ a central hyperplane arrangement in a complex vector space $V$, then the lattice of flats $\mathcal{L}(\mathcal{A})$ is a geometric lattice.
    \end{enumerate}
\end{es}

We introduce now the notion of ranked lattice and rank function because we want to describe a sort of distance between points in lattice.

\begin{defn}
A subset $\mathcal{C} \subseteq \mathcal{L}$ is said to be a chain if it is a totally ordered set.
A chain $\mathcal{C} \subseteq \mathcal{L}$ is said to be maximal if is maximal by inclusion.
The length of a chain is the number of his elements off by one, i.e. the chain $c_0 < c_1 < \dots < c_n$ has length n.
\end{defn}

\begin{defn}
A lattice $\mathcal{L}$ is called ranked of rank $n$ if every maximal chain has the same length $n$.
\end{defn}

\begin{defn}
Given a ranked lattice $\mathcal{L}$ we define on a rank function $rk : \mathcal{L} \longrightarrow \mathbb{N}$ as:
\begin{enumerate}
    \item $rk(\hat{0}) = 0$;
    \item if $x \gtrdot y$ then $rk(x) = rk(y) + 1$.
\end{enumerate}
\end{defn}

\begin{es}
    \begin{enumerate}
        \item Define $P = \{ a, b, c, d, e, f \}$, with $a \le b \le d \le f$, $a \le c \le e \le f$, $b \le e$, $c \le d$, then this is a ranked lattice.
        See \ref{esranked} for the Hasse diagram.

        \begin{figure} 
        \begin{center} \begin{tikzpicture}
            \node (f) at (0, 0) {$f$};
            \node (e) at (1, -1.4) {$e$};
            \node (d) at (-1, -1.4) {$d$};
        
            \draw (f) -- (e);
            \draw (f) -- (d);
            
            \node (b) at (-1, -2.6) {$b$};
            \node (c) at (1, -2.6) {$c$};
        
            \draw (d) -- (b);
            \draw (e) -- (c);
             
            \node (a) at (0, -4) {$a$};
        
            \draw (b) -- (a);
            \draw (c) -- (a);
        \end{tikzpicture} \end{center}
        \caption{Example of ranked lattice}
        \label{esranked}
        \end{figure}

        \item Let $\mathcal{A}$ a central hyperplane arrangement in a complex vector space $V$, then the lattice of flat $\mathcal{L}(\mathcal{A})$ is a ranked lattice with rank function given by: \[rk(H) := codim_V(H),\] for $H \in \mathcal{L}(\mathcal{A})$.
    \end{enumerate}
\end{es}

\begin{rmk}
The rank function is well defined only if the lattice is ranked. For example, consider the lattice:
\[ \mathcal{L} = \{ a, b, c, d, e \}\]
such that: $a = \hat{0}$, $a \le b \le c \le d$ and $a \le e \le d$. This lattice contain two maximal chains, $a \le b \le c \le d$ and $a \le e \le d$, and if we try to compute the rank of $d$, we obtain:
\[ rk(d) = rk(c) +1 = rk (b) + 2 = rk(a) + 3 = 3 \] and \[ rk(d) = rk(e) + 1 = rk(a) + 2 = 2.\]

\begin{center} \begin{tikzpicture}
    \node (d) at (0, 0) {$d$};
    \node (e) at (1, -2) {$e$};
    \node (c) at (-1, -1) {$c$};

    \draw (d) -- (e);
    \draw (d) -- (c);
    
    \node (b) at (-1, -3) {$b$};

    \draw (c) -- (b);
     
    \node (a) at (0, -4)   {$a$};

    \draw (b) -- (a);
    \draw (e) -- (a);
\end{tikzpicture} \end{center}

So, the rank function is not well defined.
\end{rmk}

We use now the rank function to characterize semimodular lattices.

\begin{lem} \label{LemSMiffRK}
Let $\mathcal{L}$ a finite lattice. Then $\mathcal{L}$ is semimodular if and only if $\mathcal{L}$ is ranked, with rank function $rk : \mathcal{L} \longrightarrow \mathbb{N}$, and $rk$ obey the semimodular inequality, i.e.
\begin{equation} \label{semimodineq}
    rk(F \vee F') + rk(F \wedge F') \le  rk(F) + rk(F').
\end{equation}
\end{lem}

To proof the lemma we need the induction principle on well ordered sets.
\begin{defn}
Let $I$ be a set and $\le$ a order relation on $I$, we call $(I, \le)$ a well order if $(I, \le)$ is a total order and every subset $S \subseteq I$ has a minimal element.
\end{defn}

\begin{lem}[Induction on well-ordered set]
    Let $(I, \le)$ be a well order and $P$ be a proposition on $I$, then:
    \[ \left( \forall x \in I, P(y) \text{ is true } \forall y < x  \implies P(x) \text{ is true } \right) \implies P(x) \text{ is true } \forall x \in I \]
\end{lem}

\begin{proof}
Let $A = \{ y \in I \, : \, \text{P($y$) is false} \} \subset I$, then since $I$ is a well order it has a minimum, call it $z \in A$.
Consider $x \in I$ such that $x < z$, then $x \notin A$ and then $P(x)$ is true.
So by hypothesis $P(z)$ is true so $z \notin A$, contradiction.
\end{proof}

\begin{rmk}
    In this type of induction it not necessary to verify the statement for $P(\min I)$, in fact this case is always true since the condition $\forall y < \min I, \, P(y)$ is true, is always verified since the set $\{ y \, : \, y < \min I\} = \emptyset$.
\end{rmk}

\begin{proof}[Proof of Lemma \ref{LemSMiffRK}]
    Suppose $\mathcal{L}$ is a ranked lattice with rank function $rk : \mathcal{L} \longrightarrow \mathbb{N}$. 

    Suppose $\mathcal{L}$ satisfy (\ref{semimodineq}). Let $x, y \in \mathcal{L}$, $x \neq y$,  such that $x \gtrdot x \wedge y$ and $ y \gtrdot x \wedge y$, then:
    \[ rk(x) = rk(x \wedge y) + 1 = rk(y).\]
    By hypothesis \[rk(x) + rk(y) \ge rk(x \vee y) + rk(x \wedge y) = rk(y) -1 + rk(x \vee y), \] then $rk(x) \ge rk(x \vee y) - 1$. 
    Since $rk(x \vee y) > rk(x)$, by what we just proved we have $rk(x \vee y) = rk(x) +1$, and thus $x \vee y \gtrdot x$.
    
    Analogously $x \vee y \gtrdot y$.

    Suppose now that $\mathcal{L}$ is semimodular, i.e.
    \[ \forall x, y \in \mathcal{L}, \; x, y \gtrdot x \wedge y \implies x \vee y \gtrdot x, y,\]
    and consider the well ordered set $(\mathbb{Z}, \le)$ where $\le$ is the lexicographic order:
    \[ (a,b) \le (c,d) \iff a < b \text{ or } a = b \text{ and } c < d.\]

    We proceed by induction on the subset of $(\mathbb{Z}, \le)$, $(\{ rk(x \vee y) - rk(x \wedge y) , rk(x) + rk(y) \}, \le)$.
    We split the poof in two cases:
    \begin{enumerate}
        \item $x \gtrdot x \vee y$ and $y \gtrdot x \vee y$, then 
        \[ rk(x) = rk(x \wedge y) + 1 = rk(y) \text{ and } rk(x) +1 = rk(x \vee y) =  rk(y) +1\]
        which imply \[ rk(x) + rk(y) = rk(x \vee y) + rk(x \wedge y).\]

        \item suppose $x \not\gtrdot x \wedge y$. Then $\exists z \in \mathcal{L}$ such that $ x > z > x \wedge y$ and so, applying $\vee y$, we have:
        \[ x \vee y \ge z \vee y > y \]
        where the last inequality come from $z > x \wedge y$.
        Observe now that $rk(z\vee y) - rk(z \wedge y) \le rk(x\vee y) - rk(x \wedge y)$ and if they are equal we have $ rk(z) + rk(y) < rk(x) + rk(y)$ so by lexicographic order $(z,y) < (x, y)$.

        Note also that
        \[ rk(x \vee (z \vee y)) - rk(x \wedge (z \vee y)) \le rk(x \vee y) - rk(z) < rk(x \vee y) - rk(x \wedge y)\]
        so by lexicographic order $(x, z \vee y) < (x, y).$

        Thus, by inductive hypothesis \[rk(z) + rk(y) \ge rk( z \vee y) + rk(z \wedge y)\] and \[rk(x) + rk(z \vee y) \ge rk(x \vee (z \vee y)) + rk(x \wedge (z \vee y))\]
        and since  $x \vee ( z \vee y) = x \vee y$ and $rk( x\wedge(z \vee y)) \ge rk(z)$, we conclude:
        \[rk(z) + rk(y) + rk(x) + rk(z \vee y) \ge rk(z \vee y) + rk(x \wedge y) + rk(x \vee y) + rk(z).\]
    \end{enumerate}
\end{proof}

Let $\mathcal{L}$ be an atomistic lattice and define $Int(\mathcal{L}) := \{ (F, F') \in \mathcal{L} \times \mathcal{L} \, : \, F \le F' \}$. Then the function \[ d: Int(\mathcal{L}) \longrightarrow \mathbb{N} \] is well-defined as
\[
d( F, F') := \min\{ d \in \mathbb{N} \, : \, F' = F \vee a_1 \vee \dots \vee a_d\text{, for some atoms $a_1, \dots a_d \in \mathcal{L}$} \}.
\]

If $\mathcal{L}$ is an atomistic semi-modular lattice, then the function $d$ can be duduced by the rank function:
\begin{equation} \label{eq:d(ff')}
d(F, F') = rk(F') - rk(F),
\end{equation}
where the rank function is well defined by Lemma \ref{LemSMiffRK}.

We introduce now matroids.

\begin{defn}
A matroid $\mathcal{M} = (E, \mathcal{L_{M}})$ consists of a pair of a set $E$, called ground set, and a collection of subsets of $E$, $\mathcal{L_{M}} \subseteq \mathcal{P}(E)$, called flats, such that
\begin{enumerate}
    \item $E \in \mathcal{L_{M}}$
    \item $F, F' \in \mathcal{L_{M}} \implies F \, \cap \, F' \in \mathcal{L_{M}}$
    \item for any $F \in \mathcal{L_{M}}$, and $ i \in E \, \setminus F$, there is a unique $F' \in \mathcal{L_{M}}$ containing $i$, which cover $F$, i.e. $F' \subset F$ and there are no $F'' \in \mathcal{L_{M}}$ such that $F' \subset F'' \subset F$.
    \end{enumerate}
    The elements of $\mathcal{L}_{\mathcal{M}}$ are called independent sets.
\end{defn}

\begin{es}
    \begin{enumerate}
        \item The simpler example of matroid is the lattice of subset of a set $X$, so $(X, \mathcal{P}(X))$. For example in figure \ref{hassemat} is depicted the Hasse diagram for $X = \{a, b, c ,d\}$.
        
        \begin{figure}[h!]
        \begin{center} \begin{tikzpicture}
            \node (top) at (0,0) {$\{ a, b, c, d \}$};
            \node (n1) at (-4.4, -2)  {$\{a, b, c \}$};
            \node (n2) at (-1.8, -2)  {$\{a, b, d\}$};
            \node (n3) at (1.8, -2)  {$\{a, c, d \}$};
            \node (n4) at (4.4, -2)  {$\{ b, c, d \}$};

            \draw (top) -- (n1);
            \draw (top) -- (n2);
            \draw (top) -- (n3);
            \draw (top) -- (n4);

            \node (n5) at (-5, -5)  {$\{ a,b \}$};
            \node (n6) at (-3, -5)  {$\{ a,c \}$};
            \node (n7) at (-1, -5)  {$\{ a,d \}$};
            \node (n8) at (1, -5)  {$\{ b,c \}$};
            \node (n9) at (3, -5)  {$\{ b, d \}$};
            \node (n10) at (5, -5)  {$\{ c,d \}$};

            \draw (n1) -- (n5);
            \draw (n1) -- (n6);
            \draw (n1) -- (n8);
            \draw (n2) -- (n5);
            \draw (n2) -- (n7);
            \draw (n2) -- (n9);
            \draw (n3) -- (n6);
            \draw (n3) -- (n7);
            \draw (n3) -- (n10);
            \draw (n4) -- (n8);
            \draw (n4) -- (n9);
            \draw (n4) -- (n10);
            
            
            \node (n11) at (-4.4, -8)  {$\{ a\}$};
            \node (n12) at (-1.8, -8)  {$\{ b \}$};
            \node (n13) at (1.8, -8)  {$\{ c \}$};
            \node (n14) at (4.4, -8)  {$\{ d \}$};

            \draw (n5) -- (n11);
            \draw (n5) -- (n12);
            \draw (n6) -- (n11);
            \draw (n6) -- (n13);
            \draw (n7) -- (n11);
            \draw (n7) -- (n14);
            \draw (n8) -- (n12);
            \draw (n8) -- (n13);
            \draw (n9) -- (n12);
            \draw (n9) -- (n14);
            \draw (n10) -- (n13);
            \draw (n10) -- (n14);
            
            \node (bottom) at (0, -10) {$\emptyset$};

            \draw (bottom) -- (n11);
            \draw (bottom) -- (n12);
            \draw (bottom) -- (n13);
            \draw (bottom) -- (n14);
            
        \end{tikzpicture} \end{center}
        \caption{Hasse diagram of a matroid.}
        \label{hassemat}
        \end{figure}

    We can verify if the conditions of the definition of matroids holds:
    \begin{enumerate}
        \item $\{1,2,3,4\} \in \mathcal{P}(\{1,2,3,4 \})$;
        \item since in $\mathcal{P}(\{1,2,3,4 \})$ there are all the subsets of $\{1,2,3,4\}$, then: \[ E, E' \in \mathcal{P}(\{1,2,3,4 \}) \implies E \cap E' \in \mathcal{P}(\{1,2,3,4 \});\]
        \item consider $A \in \mathcal{P}(\{1,2,3,4 \})$, with $A \neq \{1,2,3,4 \}$, then consider $a \in \{1,2,3,4\} \setminus A$. Define $A' = A \cup \{ a\}$, then $A'$ covers $A$, and $a \in A'$.
    \end{enumerate}
    
    \item Given $\mathcal{A}$ a central arrangements of hyperplane in a complex vector space $V$, then we define the matroid of $\mathcal{A}$ as $\mathcal{M}(\mathcal{A}) = (\mathcal{A}, \mathcal{L}(\mathcal{A}))$ with rank function 
        \[ rk(\mathcal{B}) := codim_V(\bigcap_{B \in \mathcal{B}}B) \]
    for $\mathcal{B} \subseteq \mathcal{A}$.
    \end{enumerate}
   
\end{es}

Finite geometric lattice are equivalent to lattices of flats of matroids. 
The above definition of matroid is pure combinatorical, but it generalize a more concrete definition of matroid, called realizable matroid.

\begin{defn} \label{def:realmat}
Let $\{ v_1, \dots, v_n \} $ a set of vector in a vector space $V$ over a field $\mathbb{F}$. A realizable matroid $\mathcal{M}$ is a pair $(E, \mathcal{L_M})$ where $E$ is the ground set $E = \{ 1, \dots, n \}$ with an operator
\[ T : \mathcal{P}(E) \longrightarrow \mathcal{P}(E) \]
defined for all $F \in \mathcal{P}(E)$
\[ T(F) := \{ i \in E \, : \, v_i \in span\{v_j\}_{j \in F} \}\]
and $\mathcal{L_M} \subset \mathcal{P}(E)$ is defined as
\[ F \in \mathcal{L_M} \iff F = T(F).\]
$T$ is called closure operator.
\end{defn}

In the next section we will also an alternative definition of matroids via closure operator.

\begin{defn}
    We define a matroid $\mathcal{M}$ to be loopless if \[ \hat{0} = \bigcap_{F \in \mathcal{M}} F = \emptyset \]
    is a flat.
\end{defn}

\begin{defn}
Given a matroid $\mathcal{M}$ we say that two elements $x,y \in \mathcal{M}$ are parallel if the set $\{x,y \}$ is a minimal dependent set.
\end{defn}

\begin{defn}
    We define a matroid $\mathcal{M}$ to be simple when is loopless and has no parallel elements.
\end{defn}

We introduce now a space of function on matroids.

\begin{defn}
    Let $\mathcal{M} = (E, \mathcal{L_M})$ be a matroid, we define the space of automorphism of $\mathcal{M}$, $Aut(\mathcal{M})$, to be the space of permutations $g$ of the ground set $E$ preserving the lattice $\mathcal{L_M}$ setwise, i.e. sending flat $F \in \mathcal{L_M}$ into a flat $g(F) \in \mathcal{L_M}$.
\end{defn}

\section{Building and Nested set} \label{sec12}
The aim of this section is to introduce the fundamental notions of building sets and nested sets, both required to define the Chow Ring, and give its geometrical interpretation via the De Concini-Procesi wonderful model.

Given a lattice $\mathcal{L}$, a subset $\mathcal{G} \subset \mathcal{L}$, and an element $F \in \mathcal{L}$ we define the set $\mathcal{G}_{\le F}$ as
\begin{equation}
\mathcal{G}_{\le F} := \{A \in \mathcal{G} \, : \, A \le F \}
\end{equation}
and we denote with $\max\mathcal{G}_{\le F}$ the set of maximal elements in $\mathcal{G}_{\le F}$.
Given $F, F' \in \mathcal{L}$ we define the set
\[  [F, F'] := \{ F'' \in \mathcal{L} \, : \, F \le F'' \le F'\}. \]

We start with a simple lemma about decomposition of interval in a poset. The proof of the Lemma can be found in \cite{backman2024convexgeometrybuildingsets}.

\begin{lem} \label{L:indecposprod}
    Let $\mathcal{P}$ a poset with a unique minimal element $\hat{0}$. Fox $x \in \mathcal{P}$ there exists a unique finest decomposition of the interval $\left [ \hat{0}, x \right ]$ in $\mathcal{P}$ as a direct product, which is given by an isomorphism 
        \[ \varphi_x^{el} : \prod_{j=1}^l \left [ \hat{0}, y_j \right ]  \longrightarrow \left [ \hat{0}, x \right ] ,\]
    with $\varphi_x^{el} (\hat{0}, \dots, y_j, \dots , \hat{0}) = y_j$ for $j = 1, \dots, l$.
    The factors of this decomposition are the intervals below the elementary divisors of $x$, $\mathcal{D}(x) = \{ y_1, \dots, y_l \}$.
\end{lem}

If the poset $\mathcal{P}$ has a maximum element, we have the special case $x = \hat{1}$. In this case $D(\hat{1}) = \max\mathcal{I}(\mathcal{P}) = \{ y_1, \dots, y_l \}$ is the set of elementary divisors of $\hat{1}$, and we have an isomorphism of posets:
\[ \phi :  \prod_{i = 1}^l \left[ \hat{0}, y_i \right] \longrightarrow  \mathcal{P}\]

\begin{defn}
Let $\mathcal{L}$ be a finite lattice and a subset $\mathcal{G} \subset \mathcal{L} \setminus\{ \hat{0} \}$. We say that $\mathcal{G}$ is a building set if for any $F \in \mathcal{L} \setminus \{ \hat{0} \}$, denoted $\max\mathcal{G}_{\le F} = \{ F_1,\dots, F_l \}$,  we have an isomorphism:

\[ \varphi_F : \left[ \hat{0}, F_1 \right] \times \dots \times \left[ \hat{0}, F_l \right] \longrightarrow \left[ \hat{0},F \right] \]
such that: 
\begin{equation} \label{def:bset}
    \varphi_F(y_1, \dots, y_l) = y_1 \vee \dots \vee y_l,
\end{equation}
for $y_i \in \left[ \hat{0}, F_i \right]$.
\end{defn}

\begin{rmk}
    \begin{enumerate}
        \item As a consequence of equation (\ref{def:bset}) we have \[\varphi_F(\hat{0}, \dots,\hat{0}, y_j,\hat{0}, \dots, \hat{0}) = y_j.\]
    
        \item Since every atom covers $\hat{0}$, 
        they are indecomposable, so:
        \[ \left\{ F \in \mathcal{L} \, : \, F \text{ is an atom} \right\} \subseteq \mathcal{G} \] for every building set $\mathcal{G} \subset \mathcal{L}$.
    \end{enumerate}    
\end{rmk}

\begin{es} \label{ex:buildset}
    \begin{enumerate}
        \item For any finite lattice $\mathcal{L}$ the maximal building set $\mathcal{G}_{\max} := \mathcal{L} \setminus \{ \hat{0} \}$ satisfy the condition of building sets since $\max(\mathcal{G}_{\max})_{\le F} = \{ F \}$.

        \item For any finite lattice $\mathcal{L}$ the minimal building set $\mathcal{G}_{\min}$ defined as the set of indecomposable elements is a building set. This is the special case of Lemma \ref{L:indecposprod} with $x = \hat{1}$.

        \item Consider the lattice $\Pi_4$ as in Example \ref{ed:lattice}.3. In this lattice we have a characterization of the indecomposable elements: $\pi \in \Pi_n$ is an indecomposable elements if and only if $\pi$ has at most one non-singleton block. So for example $\pi = 1 \, 2 \, 3 \, | \, 4$ is indecomposable instead $\pi = 1 \, 2 \, | \, 3 \, 4$ is decomposable. So the minimal building set is the set of elements colored in orange in Figure \ref{figminpi4}
        
    \begin{figure}
    \begin{center} \begin{tikzpicture}
            
        \node[orange] (top) at (0,0) {$1234$};
           
        \node (n1) at (-5, -3)  {$14\,|\,23$};
        \node[orange] (n2) at (-3.2, -3) {$1\, | \, 234$};
        \node[orange] (n3) at (-1.6, -3) {$124 \, | \, 3$};
        \node (n4) at (0, -3) {$13 \, | \, 24$};
        \node[orange] (n5) at (1.6, -3) {$123 \, | \, 4$};
        \node[orange] (n6) at (3.2, -3) {$134 \, | 2$};
        \node (n7) at (5, -3) {$12 \, | \, 34$};

        \draw (top) -- (n1);
        \draw (top) -- (n2);
        \draw (top) -- (n3);
        \draw (top) -- (n4);
        \draw (top) -- (n5);
        \draw (top) -- (n6);
        \draw (top) -- (n7);
          
        \node[orange] (n8) at (-4, -7)  {$1 \, | \, 23 \, | \, 4$};
        \node[orange] (n9) at (-2.4, -7)  {$14 \, | \, 2 \, | \, 3$};
        \node[orange] (n10) at (-0.8, -7)  {$1 \, | \, 24 \, | \, 3$};
        \node[orange] (n11) at (0.8, -7)  {$13 \, | \, 2 \, | \, 4$};
        \node[orange] (n12) at (2.4, -7)  {$12 \, | \, 3 \, | \, 4$};
        \node[orange] (n13) at (4, -7)  {$1 \, | \, 2 \, | \, 34$};

        \draw (n1) -- (n8);
        \draw (n1) -- (n9);
        \draw (n2) -- (n8);
        \draw (n2) -- (n10);
        \draw (n2) -- (n13);
        \draw (n3) -- (n9);
        \draw (n3) -- (n10);
        \draw (n3) -- (n12);
        \draw (n4) -- (n10);
        \draw (n4) -- (n11);
        \draw (n5) -- (n8);
        \draw (n5) -- (n11);
        \draw (n5) -- (n12);
        \draw (n6) -- (n9);
        \draw (n6) -- (n11);
        \draw (n6) -- (n13);
        \draw (n7) -- (n12);
        \draw (n7) -- (n13);
            
        \node (bottom) at (0, -10) {$1 \, | \, 2 \, | \, 3 \, | \, 4$};
            
        \draw (n8) -- (bottom);
        \draw (n9) -- (bottom);
        \draw (n10) -- (bottom);
        \draw (n11) -- (bottom);
        \draw (n12) -- (bottom);
        \draw (n13) -- (bottom);
            
    \end{tikzpicture} \end{center}
    \caption{Minimal building set of $\Pi_4$}
        \label{figminpi4}
    \end{figure}

    As we expected all the atoms are in the minimal building set.
    \end{enumerate}
\end{es}

We introduce here the notion of nested set which is central in this theory.

\begin{defn}
Let $\mathcal{L}$ be a lattice and $\mathcal{G} \subset \mathcal{L} \setminus \{ \hat{0} \}$ be a building set. A subset $N \subset \mathcal{G}$ is called $\mathcal{G}$-nested set if \[F_1 \vee F_2 \vee \dots F_t \notin \mathcal{G}\]
for any pairwise incomparable elements $F_1, \dots, F_t \in N$.
\end{defn}

Denote with $\mathcal{N}(\mathcal{L}, \mathcal{G})$ the family of $\mathcal{G}$-nested sets.

\begin{es}
    \begin{enumerate}
        \item Consider again the lattice $\Pi_4$ as in Example \ref{ex:buildset}.3 where as building set we consider $\mathcal{G} = \mathcal{G_{\min}}$. Then some example of $\mathcal{G}$-nested set are:
        
        \[N_1 = \{ 1\, | \, 2 \, 3 \, | \, 4 \, , \, 1 \, 4 \, | \, 2 \, | \, 3 \};\]
        \[N_2 = \{ 1 \, | \, 2 \, 4 \, | \, 3 \, , \, 1 \, 3 \, | \, 2 \, | \, 4 \};\]
        \[N_3 = \{ 1 \, 2 \, | \, 3 \, | \, 4 \, , \, 1 \, | \, 2 \, | \, 3 \, 4 \};\]
    
        as depicted in this Figure \ref{fig:n1n2n3},         where the elements of $\mathcal{G}_{\min}$ are depicted in \textcolor{orange}{orange}, \textcolor{red}{red}, \textcolor{blue}{blue} and \textcolor{green}{green}. The elements of $N_1$ are depicted in \textcolor{red}{red}, $N_2$ in \textcolor{blue}{blue} and $N_3$ in \textcolor{green}{green}.

        \begin{figure}
        \begin{center} \begin{tikzpicture}
                
            \node[orange] (top) at (0,0) {$1234$};
               
            \node (n1) at (-5, -3)  {$14\,|\,23$};
            \node[orange] (n2) at (-3.2, -3) {$1\, | \, 234$};
            \node[orange] (n3) at (-1.6, -3) {$124 \, | \, 3$};
            \node (n4) at (0, -3) {$13 \, | \, 24$};
            \node[orange] (n5) at (1.6, -3) {$123 \, | \, 4$};
            \node[orange] (n6) at (3.2, -3) {$134 \, | 2$};
            \node (n7) at (5, -3) {$12 \, | \, 34$};
    
            \draw (top) -- (n1);
            \draw (top) -- (n2);
            \draw (top) -- (n3);
            \draw (top) -- (n4);
            \draw (top) -- (n5);
            \draw (top) -- (n6);
            \draw (top) -- (n7);
              
            \node[red] (n8) at (-4, -7)  {$1 \, | \, 23 \, | \, 4$};
            \node[red] (n9) at (-2.4, -7)  {$14 \, | \, 2 \, | \, 3$};
            \node[blue] (n10) at (-0.8, -7)  {$1 \, | \, 24 \, | \, 3$};
            \node[blue] (n11) at (0.8, -7)  {$13 \, | \, 2 \, | \, 4$};
            \node[green] (n12) at (2.4, -7)  {$12 \, | \, 3 \, | \, 4$};
            \node[green] (n13) at (4, -7)  {$1 \, | \, 2 \, | \, 34$};
    
            \draw (n1) -- (n8);
            \draw (n1) -- (n9);
            \draw (n2) -- (n8);
            \draw (n2) -- (n10);
            \draw (n2) -- (n13);
            \draw (n3) -- (n9);
            \draw (n3) -- (n10);
            \draw (n3) -- (n12);
            \draw (n4) -- (n10);
            \draw (n4) -- (n11);
            \draw (n5) -- (n8);
            \draw (n5) -- (n11);
            \draw (n5) -- (n12);
            \draw (n6) -- (n9);
            \draw (n6) -- (n11);
            \draw (n6) -- (n13);
            \draw (n7) -- (n12);
            \draw (n7) -- (n13);
                
            \node (bottom) at (0, -10) {$1 \, | \, 2 \, | \, 3 \, | \, 4$};
                
            \draw (n8) -- (bottom);
            \draw (n9) -- (bottom);
            \draw (n10) -- (bottom);
            \draw (n11) -- (bottom);
            \draw (n12) -- (bottom);
            \draw (n13) -- (bottom);
                
        \end{tikzpicture} \end{center}
        \caption{Nested sets $N_1$, $N_2$ and $N_3$.}
        \label{fig:n1n2n3}
        \end{figure}
        

    
        \item If we consider $\mathcal{G}$ as the maximal building set then the $\mathcal{G}$-nested sets are the chains of $\mathcal{L} \setminus \{ \hat{0} \}$.
        In fact, consider $N$ a $\mathcal{G}$-nested set, and  $F_1, F_2 \in N$ two incomparable elements in $N$, then since $\mathcal{G} = \mathcal{L} \setminus \{ \hat{0} \}$ we have that $F_1 \vee F_2 \in \mathcal{G}$, this is a contradiction since $N$ is a $\mathcal{G}$-nested set. Thus in $N$ there are no incomparable elements, so $N$ is a chain. 
    \end{enumerate}
\end{es}

We want now to investigate a first geometrical property of the family of $\mathcal{G}$-nested sets, in particular we prove in the following that  $\mathcal{N}(\mathcal{L}, \mathcal{G})$ is a simplicial complex. 


\begin{defn}
Let $S$ be a set. A collection of subset $C \subset \mathcal{P}(S)$ is called abstract simplicial complex if it is closed under taking subsets.
\end{defn}

\begin{lem}
Let $\mathcal{L}$ be a lattice, $\mathcal{G} \subseteq \mathcal{L} \setminus \{ \hat{0} \}$ a building set, then $\mathcal{N}(\mathcal{L}, \mathcal{G})$ is a simplicial complex.
\end{lem}

\begin{proof}
Consider a $\mathcal{G}$-nested set $N$, and $A \subset N$ a subset of $N$.
Take $F_1, \dots F_t$ pairwise incomparable elements of $A$, then they are incomparable also in $N$, and by definition of $\mathcal{G}$-nested set follows that $F_1 \vee \dots F_t \notin \mathcal{G}$ and so $A$ is a $\mathcal{G}$-nested set.
\end{proof}

\begin{comment}
In the following are presented some useful characterizations of nested sets and building sets.

\begin{prop} \label{prop:charbset}
    For a semilattice $\mathcal{L}$ and a subset $\mathcal{G}$ of $\mathcal{L} \setminus \{ \hat{0} \}$ the following are equivalent:
    \begin{enumerate}
        \item $\mathcal{G}$ is a building set of $\mathcal{L}$;
        
        \item $I(\mathcal{L} \setminus \{ \hat{0} \}) \subset \mathcal{G}$, and for every $x \in \mathcal{L} \setminus \{ \hat {0} \} $ with $\mathcal{D}(x) = \{ y_1, \dots, y_l \}$, there exists a partition 
        $\pi_x = \pi_1 | \dots | \pi_k$ of $[l]$ with blocks $\pi_t = \{ i_1, \dots , i_{|\pi_t|}\}$ for $t = 1, \dots , k$ such that the elements in $\max\mathcal{G}_{\le x} = \{ x_1, \dots , x_k \}$ are of the form 
        \[ x_t = \varphi_x^{el} (\hat{0}, \dots, \hat{0}, y_{i_1}, \hat{0}, \dots, \hat{0}, y_{i_2}, \hat{0},\dots,\hat{0}, y_{i_{|\pi_t|}},\hat{0}, \dots, \hat{0});\]
        
        \item $\mathcal{G}$ generates $\mathcal{L} \setminus \{ \hat{0} \}$ by $\vee$, and for any $x \in \mathcal{L}$, any $\{ y, y_1, \dots, y_t \} \subseteq \max \mathcal{G}_{\le x}$, and $z \in \mathcal{L}$ with $z < y$, we have 
        \[ \mathcal{G}_{\le y} \bigcap \mathcal{G}_{\le z \vee y_1 \vee \dots \vee y_t} = \mathcal{G}_{\le z} \]
        
        \item $\mathcal{G}$ generates $\mathcal{L} \setminus \{ \hat{0} \}$ by $\vee$, and for any $x \in \mathcal{L}$, any $\{ y, y_1, \dots, y_t \} \subseteq \max \mathcal{G}_{\le x}$, and $z \in \mathcal{L}$ with $z < y$, the following two conditions are satisfied:
        
        \begin{enumerate}
            \item $\mathcal{G}_{ \le y} \bigcap \mathcal{G}_{y_1 \vee \dots \vee y_t} = \emptyset$;
            
            \item $z \vee y_1 \vee \dots \vee y_t < y \vee y_1 \vee \dots \vee y_t$.
        \end{enumerate}
        
    \end{enumerate}
\end{prop}

\begin{proof}
    (1) $\implies$ (2): the fact that $I(\mathcal{L}) \setminus \{ \hat{0} \} \subseteq \mathcal{G}$ follows directly by the definition of building set. 
    By the same definition we also have the isomorphism:
    \[ \varphi_x : \prod_{j = 1}^k \left[ \hat{0}, x_j \right]  \longrightarrow \left[ \hat{0}, x \right] \]
    and by Lemma \ref{L:indecposprod} we have 
    \[ \varphi_{x_j}^{el} : \prod_{y \in D(x_j)} \left[ \hat{0}, y \right] \longrightarrow \left[ \hat{0}, x_j \right]. \]

    Considering the composition of functions: 
    \[\phi_x := \varphi_x \circ \left( \prod_{j = 1}^k \varphi_{x_j}^{el} \right) : \prod_{j=1}^k \prod_{y \in D(x_j)} \left[ \hat{0}, y \right] \longrightarrow \left[ \hat{0}, x \right], \]
    this gives us the finest decomposition of the interval $\left [ \hat{0}, x \right]$.
    Thus, $D(x) = \bigsqcup_{j = 1}^k D(x_j)$. Observe now that $D(x) = \{ y_1, \dots, y_l\}$ so if we define $\pi_{t} = D(x_t) $, then by what we previously proved we have the partition of $[l]$, $\pi_1, \dots, \pi_{k}$.

    (2) $\implies $ (1): we want to prove that there is an isomorphism:
    \[ \varphi_x : \prod_{j = 1} ^ k \left[ \hat{0}, x_j \right] \longrightarrow \left[ \hat{0}, x \right]. \]
    By Lemma \ref{L:indecposprod} we have a decomposition:
    \[ \varphi_x^{el} : \prod_{j=1} ^ l \left[ \hat{0}, y_j \right] \longrightarrow \left[ \hat{0}, x \right] \]
    where $\{ y_1, \dots, y_l\} = D(x)$.
    We can now assemble factors $\left[ \hat{0}, y_j \right]$ with maximal elements indexed by elements from the same block of the partition $\pi_x$ into one factor. 

    (1) $\implies$ (3): by $(1)$ we have the decomposition:
    
    \[ \varphi_x : \prod_{ j = 1 }^k \left[ \hat{0}, x_j \right] \longrightarrow \left[ \hat{0}, x \right],\]
    such that $\varphi_x(y_1, \dots , y_k) = y_1 \vee \dots \vee y_k$, for every $y_j \in [\hat{0}, x_j]$.
    Chosen $\{ y, y_1, \dots, y_t \} \subset \max \mathcal{G}_{\le x}$ and $z \in \mathcal{L}$, we want to prove that 
    \[ \mathcal{G}_{\le y} \bigcap \mathcal{G}_{ \le z \vee y_1 \vee \dots \vee y_t} = \mathcal{G}_{\le z }.\]
    
    Consider $x \in \mathcal{G}_{\le z}$ then $ x \le z < y$, so $x < y$. Since $x \le z$ we have that 
    \[x \le z \le z \vee y_1 \vee \dots \vee y_t,\]
    so $x \in \mathcal{G}_{\le y} \bigcap \mathcal{G}_{ \le z \vee y_1 \vee \dots \vee y_t}$. 
    
    Viceversa, if $ x \in \mathcal{G}_{\le y} \bigcap \mathcal{G}_{ \le z \vee y_1 \vee \dots \vee y_t}$, then:
    \[\varphi_x(\hat{0}, \dots, \hat{0}, y_1, \hat{0}, \dots , \hat{0}, y_t, \hat{0},\dots, \hat{0}) = y_1 \vee \dots \vee y_t \in [\hat{0}, x],\]
    in particular hold the strict inequality $y_1 \vee \dots \vee y_t < x$.
    
    Since $x > y_1 \vee \dots \vee y_t$ and $x \le z \vee y_1 \vee \dots \vee y_t$ then $x \le z$, and so $x \in \mathcal{G}_{\le z}$.
    
    (3) $\implies$ (4):
    \begin{enumerate}[label=(\alph*)]
        \item it follows by setting $z = \hat{0}$ in (3);
        
        \item equality in (b) implies together with (3) that $\mathcal{G}_{\le y} = \mathcal{G}_{\le z}$, in particular, $y \in \mathcal{G}_{\le z}$, a contradiction with $z < y$.
    \end{enumerate}
    
    (4) $\implies$ (1): for $x \in \mathcal{L} \setminus \{ \hat{0} \}$ and $\max \mathcal{G}_{\le x} = \{ x_1, \dots, x_k \}$ consider the poset map:
    \[ \varphi_x : \prod_{j = 1}^k \left[ \hat{0}, x_j \right] \longrightarrow \left[ \hat{0}, x \right] \]

    which sends $ \left( \alpha_1, \dots, \alpha_k  \right) \mapsto \alpha_1 \vee \dots \vee \alpha_k$. 
    
    We want now to prove that $\varphi_x$ is a bijection.
    
    \begin{enumerate}
        \item  $\varphi_x$ is surjective: for $\hat{0} \neq y \le x$, let $\max \mathcal{G}_{\le y} = \{ y_1, \dots, y_t \}$. 
        Since, by (3), $\mathcal{G}$ generates $\mathcal{L}$ by $ \vee $ we have that:
        \[ y = \bigvee_{i = 1} ^ t y_i.  \]
        Define now $S_j := (\max \mathcal{G}_{\le y}) \bigcap \mathcal{G}_{\le x_j}$, and $\gamma_j := \bigvee_{y_i \in S_j} y_i$, for $j = 1, \dots , k $. Clearly, since $\gamma_j$ is the join of elements in $\mathcal{G}_{\le x_j}$, $\gamma_j \in \left[ \hat{0}, x_j \right]$. Observe also that $\bigcup_{j=1}^k S_j = \max \mathcal{G}_{\le x}$, in fact:
        \[ \bigcup_{j=1}^k S_j = \bigcup_{j=1}^k (\max \mathcal{G}_{\le y}) \cap \mathcal{G}_{\le x_j} = \max \mathcal{G}_{\le y} \cap \bigcup_{j=1}^k \mathcal{G}_{\le x_j} = \max \mathcal{G}_{\le y},\]
        where the last equality follow from the fact that $\mathcal{G}_{\le y } \subset \mathcal{G}_{\le x} \subset \bigcup_{j = 1}^k \mathcal{G}_{\le x_j}$.
        Hence:
        \[ \varphi_x(\gamma_1, \dots , \gamma_t) = \bigvee_{i = 1}^t y_i = y.\]
    \item $\varphi_x$ is injective: assume that $\varphi_x( \alpha_1, \dots, \alpha_k) = \varphi_x(\beta_1,
        \dots,\beta_k)$, we split the proof in two cases:
        \begin{enumerate}
            \item $\varphi_x(\alpha_1, \dots , \alpha_k) = y \neq x.$ Let $\max \mathcal{G}_{\le y} = \{ y_1, \dots, y_t \}$, by induction on the number of elements in $\left[ \hat{0}, x \right]$ we can assume that $\left[ \hat{0}, y \right]$ decomposes as:
            \[ \varphi_y : \prod_{i = 1}^t \left[ \hat{0}, y_i \right] \longrightarrow \left[ \hat{0}, y \right].\]
            Moreover, the subsets $S_j$ of $\max \mathcal{G}_{\le y}$, defined in the proof of $\varphi_x$ injective, partitions $\max \mathcal{G}_{\le y}$ by $(a)$ in the point $(3)$ applied to pairwise intersections of the $\mathcal{G}_{\le x_j}$. Thus, \[ \left[ \hat{0}, y \right] \simeq \prod_{j = 1}^k \left[ \hat{0}, \gamma_j \right],\]
            whit elements $\gamma_j \in [\hat{0}, x_j]$ as above, and it follows that $\alpha_j = \beta_j = \gamma_j$ for $j = 1, \dots, k$. 

            \item Assume that $\varphi_x( \alpha_1, \dots, \alpha_k) = x$. By the property $(b)$ in the point $(3)$ it follows that $\alpha_j = \beta_j = x_j$ for $j = 1, \dots,k$.
        \end{enumerate}
    \end{enumerate}
    \end{proof}

\begin{prop}
    For a building set $\mathcal{G}$ of $\mathcal{L}$, the following holds:
    \begin{enumerate}
        \item let $x \in \mathcal{L}$, $\max \mathcal{G}_{\le x} = \{ x_1, \dots, x_k\}$, and $ \hat{0} \neq y \in \mathcal{G}$ with $y \le x$. Then there exists a unique $j \in \{ 1, \dots, k\}$ such that $y \le x_j$; i.e., $\max \mathcal{G}_{\le x}$ induces a partition of $\mathcal{G_{\le x}}$;
        \item for $x \in \mathcal{L}$ and $x_0 \in \max \mathcal{G}_{\le x}$,
        \[ \bigvee (\max \mathcal{G}_{\le x} \setminus \{ x_0\}) < \bigvee \max \mathcal{G}_{\le x} = x,\]
        i.e., each factor of $x$ in $\mathcal{G}$ is needed to generate $x$;
        \item if $h_1, \dots, h_k \in \mathcal{G}$ are such that $\left] h_i, \bigvee_{j=1}^k h_j\right] \cap \mathcal{G} = \emptyset$ for $i = 1, \dots, k$, then \[\max \mathcal{G}_{\bigvee_{j=1}^k h_j} = \{ h1, \dots, h_k\}.\]
    \end{enumerate}
\end{prop}

\begin{proof}
\begin{enumerate}
    \item Is a consequence of the point (4) in the Proposition \ref{prop:charbset}, as we alredy noted in the proof of the implication (4) $\implies$ (1). 

    \item Is a consequence of (4) part (b) of Proposition \ref{prop:charbset} considering the full set of factors and $z = \hat{0}$.

    \item By (3) note that $\{ h_1, \dots, h_k \} \subset \max \mathcal{G}_{\vee_{j=1}^k h_j}$ by assumption. 
    Suppose $\{ h_1, \dots, h_k \}$ are not the complete set of factors, then by point (2) we should have
    \[ \bigvee_{j=1}^k h_j = h_1 \vee \dots \vee h_k < \bigvee_{j=1}^k h_j\]
    which is contradiction.
\end{enumerate}
\end{proof}

The next proposition is a simple characterization of a nested set.

\begin{prop} \label{prop:charnset}
    FINIRE DI SISTEMARE LA DIMOSTRAZIONE.
    Let $\mathcal{L}$ be a semilattice and $N$ a nested set in a building set $\mathcal{G}$ of $\mathcal{L}$, then the following are equivalent:
    \begin{enumerate}
        \item $N$ is nested;
        \item Whenever $x_1, \dots , x_t$ are incomparable elements in $N$, the join $x_1 \vee \dots \vee x_t$ exists and $\max \mathcal{G}_{<x_1 \vee \dots \vee x_t} = \{ x_1, \dots ,\ x_t\}$.
    \end{enumerate}
\end{prop}

\begin{proof}
    (1) $\implies$ (2): Let $N$ be a nested set, and $M = \{ x_1, \dots, x_t\} \subseteq N$ a set of incomparable elements with $\bigvee_{i = 1} ^ t x_i \notin \mathcal{G}$.
        We have two cases:
        \begin{enumerate}
            \item $\exists x_i \, : \, \left ] x_i , \bigvee_{i=1}^t x_i \right] \bigcap \mathcal{G} \neq \emptyset$: without loss of generality we assume $i = 1$ and that exists $y \in \left ] x_1, \bigvee_{i=1}^t x_i \right] \bigcap \mathcal{G}$ such that $y \in \max \mathcal{G}_{< \bigvee M}$.
            Define:
            
            \[ M' := \{ x_1, \dots, x_t \} \cap \mathcal{G}_{ \le y}: = \{ x_1, x_{j_1}, \dots , x_{j_k} \} \text{, and } z = \bigvee_{l = 0}^{k} x_{j_l} \] 
            
            where we set $x_{j_0} = x_1$.
            Since $M' = \{ x_{j_0}, x_{j_1}, \dots x_{j_k} \}$ is nested, because is a subset of a nested set, we have $z < y$. Moreover:
            
            \[ \bigvee_{i=1}^{t} x_i = z \vee \bigvee{ (M \setminus M')} \le z \vee \bigvee(\max \mathcal{G}_{\bigvee M} \setminus \{ y \}) < \bigvee_{i=1}^t x_i,\]
            which is a contradiction.

            \item $\forall x_j, \,\left] x_j, \bigvee_{i=1}^t \right] = \emptyset$: 
            
            
        \end{enumerate}
    (2) => (1): obvious.
    
\end{proof}
\end{comment}

We define now a function called $m_N$, where $N \subseteq \mathcal{G}$ is a $\mathcal{G}$-nested set, using the distance function $d$ previously defined.

\begin{defn}
Let $\mathcal{L}$ a lattice, $\mathcal{G} \subset \mathcal{L}$ a building set and $N \subset \mathcal{G}$ a $\mathcal{G}$-nested set. Given $F \in N$ define:
\[ m_N(F) := d(\vee N_{<F}, F)\]
\end{defn}

If $\mathcal{L}$ is a geometric lattice then, by \ref{eq:d(ff')} we have
\begin{equation}
    m_N(F) = rk(F) - rk(\vee N_{<F}).
\end{equation}

The goal now is to prove the following identity for geometric lattices:
\begin{equation} \label{useful_id}
    rk(\vee N) = \sum_{F \in N} m_N(F).
\end{equation} 

To proof the identity we start with some lemmas about building and nested sets, the first lemma is relative to the maximal elements of a nested sets.

\begin{lem} \label{lem:charnset}
    Let $\mathcal{L}$ be a lattice and $\mathcal{G}$ a building set. Let $N \in \mathcal{N}(\mathcal{L}, \mathcal{G})$ with $x_1, \dots , x_k \in N$ incomparable and let $x = \bigvee_{i=1}^k x_i$, then $\max\mathcal{G}_{\le x} = \{x_1, \dots ,x_n \}$.
\end{lem}

\begin{proof}
    Let $\max\mathcal{G}_{\le x} = \{y_1, \dots, y_l \}$, with $y_1, \dots, y_l$ incomparable elements of $N$. Define for $i \in \{ 1, \dots, l\}$ \[M_i = N \cap \mathcal{L}_{\le y_i} \text{ and } z_i  = \bigvee_{x_j \in M_i} x_j.\] Observe now that \[x = \bigvee _{i=1}^{k} x_i = \bigvee_{i=1}^l \bigvee_{x_j \in M_i} x_j = \bigvee_{i=1}^l z_i.\] 
    Suppose $\{x_1, \dots , x_k \} \neq \{ y_1, \dots, y_l \}$, then there exists $i \in \{1, \dots, l\}$ such that $M_i \neq \{y_i \}$, so $M_i$ is not a nested set and $z_i \notin \mathcal{G}$ by definition of nested set. Then $z_i < y_i$, and by definition of building set we have \[\bigvee_{i=1}^l z_i < \bigvee_{i=1}^l y_i = x\] which is a contradiction. So $\{x_1, \dots , x_k \} = \{ y_1, \dots, y_l \} = \max\mathcal{G}_{\le x}$;
\end{proof}

The second lemma describes how the function $m_N$ act on a nested set with its maximal element removed.

\begin{lem}
    Let $N \subset \mathcal{G}$ a nested set, where $\mathcal{G}$ is a building set, define $F = \vee N$ and $\hat{N} : = N \setminus \{ F \}$ then:\
    \[ m_{N}(F') = m_{\hat{N} }(F'),  \, \forall F' \in \hat{N}\]
\end{lem}

\begin{proof}
    Let $F' \in \hat{N}$, then:
    \[ m_N(F') = d(F', \vee N_{<F'})\]
    but since $F' \in \hat{N}$ and $F = \vee N$ we have $F' < F$ and so $N_{< F'} = \hat{N}_{< F'}$.
    Thus: \[ m_N(F') = d(F', \vee N_{<F'}) = d(F', \vee \hat{N}_{<F'}) = m_{\hat{N}}(F')\]
\end{proof}

The third lemma describes the form of a nested, in particular it gives us also a disjointedness property for ideals in a nested set.
The proof of the following lemma can be found in \cite{Pagaria_2023}.
 
\begin{lem} \label{lem:nsform}
    Let $\mathcal{L}$ a lattice, and $\mathcal{G} \subseteq \mathcal{L}\setminus \{ \hat{0} \}$ a building set then for any $N$, $\mathcal{G}$-nested set, $N$ is a forest poset i.e. every element if covered by at most one element of $N$.
    
    In particular given $F, F' \in N$ incomparable, with $N$ a $\mathcal{G}$-nested set then 
    \[ N_{\le F} \cap N_{\le F'} = \emptyset.\]
\end{lem}

Now we proceed proving the identity \ref{useful_id}.

\begin{lem} \label{lem:usefulineq}
Let $\mathcal{L}$ a geometric lattice, $\mathcal{G}$ a building set and $N \subset \mathcal{G}$ a $\mathcal{G}$-nested set, then:
\[ rk(\vee N) = \sum_{F \in N} m_N(F).\]
\end{lem}

\begin{proof}
Proceed by induction on the cardinality of $N$.
\newline
The base case is $|N| = 0$ i.e. $N = \emptyset$. In this case we have:
\[ rk(\vee N) = rk(\hat{0}) = 0,\]
which is equal to the empty sum.

Let now $|N| > 0$ and suppose $\max(N) = \{ F_1, \dots, F_l \}$. We split the proof in two cases:
\item Case 1: $l = 1$.
In this case we have $\max(N) = {F_1}$, therefore $\vee N = F_1$. Denoting with $\hat{N}$ the set $N \setminus {F_1}$, we obtain:
    \[ \sum_{F \in N} m_N(F) = m_N(F_1) + \sum_{F \in \hat{N}} m_{N} (F) = m_N(F_1) + \sum_{F \in \hat{N}} m_{\hat{N}}(F)\]
using now the inductive step on the set $\hat{N}$ we obtain:
\[  m_N(F_1) + \sum_{F \in \hat{N}} m_{\hat{N}}(F) = rk(F_1) + rk(\vee \hat{N}) = rk(F_1) = rK(\vee N)\]
\item Case 2: $l > 1$.
By Lemma \ref{lem:charnset} we have that $\max\mathcal{G}_{\le {\vee N}} =  \{ F_1, \dots , F_l \}$, and therefore by definition of $\mathcal{G}$-nested set:
\[ [\hat{0}, \vee N] \cong \prod_{i = 1} ^{l} [\hat{0}, F_i]. \]
Using the the above equality and the inductive hypothesis on sets $N_{\le F_i}$ we obtain:
 
\[ rk(\vee N) = \sum_{i = 1} ^l rk(F_i) = \sum_{i  = 1} ^l \sum_{F \in N_{\le F_i}} m_{N_{\le F_i}} (F) = \sum_{i  = 1} ^l \sum_{F \in N_{\le F_i}} m_N (F) .\]
Observe now that by Lemma \ref{lem:nsform} we have a disjoint decomposition $N = \bigsqcup_{i = 1} ^l N_{\le F_i}$, from which we obtain:
\[ \sum_{i  = 1} ^l \sum_{F \in N_{\le F_i}} m_N (F) = \sum_{F \in N} m_N(F).\]
\end{proof}

%subsection sulla geometry of building set con semilattici
\begin{comment}

\subsection{Geometry of Building Sets}

In this section we introduce some results about the geometry of the set of building sets. In this section we restrict to meet-semilattice that are posets in which given two elements their meet exists.

Given a meet-lattice $\mathcal{L}$ we denote with $\mathbb{B}(\mathcal{L})$ the family of building sets of $\mathcal{L}$.

\begin{defn}
    Let $\mathcal{L}, \mathcal{K}$ two meet-semilattices, then a meet-semilattice order embedding of $\mathcal{L}$ and $\mathcal{K}$ is a poset order embedding  $f:\mathcal{L} \longrightarrow \mathcal{K}$ such that:
    \[ \forall x, y \in \mathcal{L}, \, x \wedge^{\mathcal{L}} y = f(x) \wedge^{\mathcal{K}} f(y). \]
\end{defn}

\begin{rmk}
    Observe that a lattice order embedding need not to respect the joins, in fact if $\mathcal{L}$ is a meet semilattice with a maximal element $\hat{1} \in \mathcal{L}$, then given $x,y \in \mathcal{L}$, we can define:
    \[ x \vee y := \bigwedge_{x, y \le z} z. \]
    Note also that if $\mathcal{L}$ is a meet-semilattice, and $x,y \in \mathcal{L}$ with $x \le y$, then $\left[x,y \right]$ is a lattice. 
\end{rmk}

\begin{defn}
    Let $\mathcal{E}$ be a set, an intersection lattice is $\mathbb{S} \subset \mathcal{P}(\mathcal{E})$ ordered by inclusion with meet given by:
    \[ A, B \in \mathbb{S}, \, A \wedge B := A \cap B.\]
\end{defn}



The definitions of $\mathcal{I}(\mathcal{P})$, and $\mathcal{D}(x)$ for a poset $\mathcal{P}$, can be extended to a meet-semilattice $\mathcal{L}$ as follows.
\begin{defn}
    Let $\mathcal{L}$ be a meet-semilattice, we define the set:
    \[ I(\mathcal{L}) = \{x \in \mathcal{L} \, : \, [\hat{0}, x] \text{ is irreducible}\} \]
    and the elements of the set $\mathcal{D}(x) := \max I(\mathcal{L})_{\le x}$ are called elementary divisors of $x$.
\end{defn}

\begin{prop} \label{P:mslChar}
    Let $\mathcal{L}$ be a meet-semilattice and $\mathcal{G} \subset \mathcal{L} \setminus \{\hat{0}\}$. Then $\mathcal{G}$ is a building set if and only if:
    \begin{enumerate}
        \item $I(\mathcal{L}) \subset \mathcal{G}$
        \item if $x, y \in \mathcal{G}$ such that $x \wedge y \neq \hat{0}$ and $x \vee y \in \mathcal{L}$, then $x \vee y \in \mathcal{G}$.
    \end{enumerate}
\end{prop}

\begin{proof}
($\implies$) Let $\mathcal{G} \in \mathbb{B}(\mathcal{L})$. As we alredy noted is clear that $\mathcal{I}(\mathcal{L}) \subseteq \mathcal{G}$.
Let $x, y \in \mathcal{G}$ such that $x \vee y \notin \mathcal{G}$. If $x \vee y \notin \mathcal{L}$, there is nothing to prove, suppose that $x \vee y \in \mathcal{L}$, and let $\max \mathcal{G}_{\le x \vee y} = \{ z_1, \dots, z_k \}$.
Since $x, y \in \mathcal{G}$, and by definition $x,y \le x\vee y$, then $x$ and $y$ are strictly less than $x \vee y$ as $x \vee y \notin \mathcal{G}$. Therefore there exists $s,t$ with $s \neq t$ such that $x \in \left[ \hat{0}, z_s \right]$ and $y \in \left[ \hat{0}, z_t \right]$.
Since $\mathcal{G}$ is a building set there is an isomorphism
\[ \phi_{x \vee y} \, : \, \prod_{j=1}^k \left[ \hat{0}, z_j \right] \longrightarrow \left[ \hat{0}, x \vee y \right]. \]
Since $\phi_{x \vee y}$ in injective, then $\bigcap_{j=1}^k \left[ \hat{0}, z_j \right] = \hat{0}$. It follows that $x \wedge y = \hat{0}$.

($\impliedby$) Let $\mathcal{G} \subset \mathcal{L} \setminus \{ \hat{0} \}$ which satisfies the conditions (1) and (2). Let $x \in \mathcal{L} \setminus \{ \hat{0} \}$ and let $\max \mathcal{G}_{\le x} =\{ x_1, \dots, x_k\}$. We wish to show that:
\[ \prod_{j=1} ^k \left[ \hat{0}, x_j \right] \simeq \left[ \hat{0}, x \right].\]
We begin with the observation that for each $1 \le i,j \le k$ with $i \neq j$, $\left[ \hat{0}, x_i \right] \cap \left[ \hat{0}, x_j \right] = \hat{0}$. In fact suppose that exists $z \in \left[ \hat{0}, x_i \right] \cap \left[ \hat{0}, x_j \right]$ with $ z \neq \hat{0}$. Then $x_i \wedge x_j \neq \hat{0}$, since $\hat{0} \neq z \le x_i \wedge x_j$ and $x_i, x_j < x$, so $x_j \vee x_j \in \mathcal{L}$, thus $x_i \vee x_j \in \mathcal{G}$. But $x_i, x_j < x_i \vee x_j$ and $x_i \vee x_j \le x$ which contradicts the maximality of $x_i$ and $x_j$.
Let $\mathcal{D}(x) = \{y_1, \dots, y_l \}$. For each $1 \le t \le k$, let
\[ \pi_t = \{ y_i \in \mathcal{D}(x) \, : \, y_i \le x_t \},\]
we want to prove that the set of $\pi_t$ form a partition of $\mathcal{D}(x)$.
By assumption $I(\mathcal{L}) \subseteq \mathcal{G}$, so for each $y_i$ there exists some $x_t$ such that $y_i \le x_t$, and since the sets $\left[ \hat{0}, x_i \right]$ are all disjoint, we have the uniqueness of $x_t$, and this establish the partition.
Next we want to prove that $\mathcal{D}(x_t) = \pi_t$. Let $y \in \mathcal{D}(x_t)$, and let $y_i$ be such that $y \le y_i$. Then there exists a unique $t$ such taht $y_i \le x_t$, and so $y = y_i$. Therefore we know that there exists two isomorphisms
\[ \phi_x : \prod_{j=1}^l \left[ \hat{0}, y_j \right] \longrightarrow \left[ \hat{0}, x \right] \]
and
\[ \phi_{x_t} : \prod_{y \in \pi_t} \left[ \hat{0}, y \right] \longrightarrow \left[ \hat{0}, x_t \right] \]
and so the map
\[ \phi_x \circ \prod_{t = 1}^k (\phi_{}x_t)^{-1} : \prod_{t=1}^k \left[ \hat{0}, x_t \right] \longrightarrow \left[ \hat{0}, x \right] \]
is an isomorphism.

\end{proof}

We want now to investigate the poset $(\mathbb{B}(\mathcal{L}), \subseteq)$.
The first thing we prove is the fact that $(\mathbb{B}(\mathcal{L}), \subseteq)$ is an intersection lattice.

\begin{prop}
    Let $\mathcal{L}$ a meet-semilattice. The poset $(\mathbb{B}(\mathcal{L}), \subseteq)$ is an intersection lattice.
\end{prop}

\begin{proof}
    We start by proving that the intersection of two building sets is a building sets. Let $B, B' \in \mathbb{B}(\mathcal{L})$ two building sets, by Proposition \ref{P:mslChar} we have that $I(\mathcal{L}) \subseteq B$ and $I(\mathcal{L}) \subseteq B'$ so $I(\mathcal{L}) \subseteq B \cap B'$ thus condition 1 of Proposition  \ref{P:mslChar}  is statisfied. Let now $x, x' \in B \cap B'$ such that $x \wedge x' \neq \hat{0}$. Suppose $x \vee x' \in \mathcal{L}$ then since $B$ and $B'$ are building sets we have that $x \vee x' \in B$ and $x \vee x' \in B'$ and so $x \vee x' \in B \cap B'$. Thus also condition 2 of Propositione \ref{P:mslChar} is satisfiend, so $B \cap B'$ is a building set.
    We proved that $(\mathbb{B}(\mathcal{L}), \subseteq)$ is an intersection meet-semilattice, and we alredy know that $\mathcal{L} \setminus \{ \hat{0} \}$ is the maximal building sets so $(\mathbb{B}(\mathcal{L}), \subseteq)$ is an intersection lattice.
\end{proof}

In the following we define set systems and closure systems which we will use to define convex geomtries.

\begin{defn}
    We define a set system as a pair $(\mathcal{E}, \mathbb{S})$ where $\mathcal{E}$ is a set and $\mathbb{S} \subset \mathcal{P}(\mathcal{E})$. We refer to $\mathcal{E}$ as ground set.
\end{defn}

\begin{es}
\begin{enumerate}
    \item Topological spaces are set systems.

    \item Matroids are set systems.
\end{enumerate}
\end{es}

\begin{defn}
    Let $\mathcal{E}$ a set, a clousure operator on $\mathcal{E}$ is a map $\sigma : \mathcal{P}(\mathcal{E}) \longrightarrow \mathcal{P}(\mathcal{E})$ such that:
    \begin{enumerate}
        \item $A \subseteq \sigma(A)$;
        \item $A \subseteq B \implies \sigma(A) \subseteq \sigma(B)$;
        \item $\sigma(\sigma(A)) = \sigma(A)$.
    \end{enumerate}
We say that $A \in \mathcal{P}(\mathcal{E})$ is closed if $\sigma(A) = A$.
\end{defn}

\begin{defn}
    Let $\sigma$ be a closure operator on the ground set $\mathcal{E}$, define \[ \mathbb{S} = \{ A \in \mathcal{P}(\mathcal{E}) \, : \, \sigma(A) = A \} \] the closed subsets of $\mathcal{E}$. Then $(\mathcal{E}, \mathbb{S}, \sigma)$ is called closure system.
\end{defn}

\begin{es}
    \begin{enumerate}
        \item Let $(X, \tau)$ a topological space, define on $\mathcal{P}(X)$ the operator $cl(A)$ as the smallest closed subset of $X$ containing $A$ i.e.
        \[ cl(A) := \bigcap_{\substack{K \supseteq A \\ K \text{ closed}}} K. \]
        Then cl is a closure operator, in fact 
        \begin{enumerate}
            \item $A \subset$ $cl(A)$ by definition;
    
            \item let $A \subset B$ then the smallest closed subset containing $B$ contains also $A$ and the smallest closed subset containing $A$ is contained in the smallest closed subset containing $B$, so $cl(A)$ $\subseteq$ $cl(B)$;
    
            \item since $cl(A)$ is by definition closed, the smallest closed subset containing $cl(A)$ is $cl(A)$, so $cl(cl(A)) = cl(A)$. 
        \end{enumerate}
    
        Defining $\mathcal{C}(X) = \{ A \subset X \, | \, cl(A) = A\}$,  then $(X, \mathcal{C}(X), cl)$ is a closure system.

        \item AGGIUGNERE ESEMPIO SUI MATROIDI REALIZZABILI

        \item 
    \end{enumerate}
\end{es}

Observe now that a closure system $(\mathcal{E}, \mathbb{S}, \sigma)$ comes with a natural meet an join. Define in fact $A \wedge B = A \cap B$ and $A \vee B = \sigma(A \cup B)$, and so $\mathbb{S}$ is an intersection lattice. 

Viceversa, given an intersection lattice $\mathbb{S}$ with ground set $\mathcal{E}$ we can define a closure operator $\sigma : \mathcal{P}(\mathcal{E}) \longrightarrow \mathcal{P}(\mathcal{E})$ as
\[ \sigma(A) = \bigcap_{C \in \mathbb{S}, A \subseteq C} C.\]
This induces a canonical identification between closure operator and intersection lattices.
In particular a closure system $(\mathcal{E}, \mathbb{S}, \sigma)$ is encoded by the pair $(\mathcal{E}, \mathbb{S})$ alone.

We can now define what a convex geometry is.

\begin{defn}
    Let $(\mathcal{E}, \mathbb{S}, \sigma)$ a closure system. We say that $(\mathcal{E}, \mathbb{S}, \sigma)$ si a convex geometry if for all $A \in \mathbb{S}$ and for $x, y \notin A$ we have that $y \in \sigma(A \cup \{x \})$ or $x \in \sigma(A \cup \{ y \})$. 
In this case we call $\sigma$ an anti-exchange closure operator.
\end{defn}

\begin{es}
    Consider $\mathcal{E} = \mathbb{R}^n$ and $\mathbb{S} = \{ A \subset \mathbb{R}^n \, | \, \text{A is convex} \}$. Define the closure operator $co(A)$ as the smallest convex set of $\mathbb{R}^n$ containing $A$. Then $(\mathbb{R}^n, \mathbb{S}, co)$ is a convex geometry.
\end{es}

We want now to define antimatroids and investigate the connection with convex geometries. 

\begin{defn}
    An antimatroid is a set system $(\mathcal{E}, \mathbb{S})$ sucht that:
    \begin{enumerate}
        \item $\emptyset \in \mathbb{S}$
        \item for every $A, B \in \mathbb{S}$ such that $ B \not\subset A$, there exists $e \in B \setminus A$ such that $A \cup \{ e \} \in \mathbb{S}$.
    \end{enumerate}
\end{defn}

Let's define the dual notion of set system.

\begin{defn}
    Let $(\mathcal{E}, \mathbb{S})$ be a set system, we call the complementary set system the set system $(\mathcal{E}, \mathbb{S}^c).$.
\end{defn}

It's possible to prove that a set system $(\mathcal{E}, \mathbb{S})$ is an antimatroid if and only if the complementary set system $(\mathcal{E}, \mathbb{S}^c)$ is a convex geometry. Thus we can alternately define convex geometries as the dual notion of antimatroids.

\begin{defn} \label{def:convgeo2}
    A convex geometry is a set system $(\mathcal{E}, \mathbb{S})$ such that:
    \begin{enumerate}
        \item $\mathcal{E} \in \mathbb{S}$;
        \item for every $A, B \in \mathbb{S}$ such that $A \not\subseteq B$, there exists $e \in A\setminus B$ such that $A \setminus \{ e \} \in \mathbb{S}$.
    \end{enumerate}
\end{defn}

We now prove that a convex geometry is an intersection lattice.

\begin{prop}
    Let $(\mathcal{E}, \mathbb{S})$ be a convex geometry then $\mathcal{S}$ is an intersection lattice.
\end{prop}

\begin{proof}
    Let $A, B \in \mathbb{S}$. By condition (2) of Definition \ref{def:convgeo2} we can take an element $e \in A \setminus B$, remove it from $A$ and $A \setminus\{e\} \in \mathbb{S}$. Repeating this process we can remove all elements in $A \setminus B$ and obtain the set $A \cap B$. By condition (2) we have $A \cap B \in \mathbb{S}$. Hence given $A, B \in \mathbb{S}$ we have $A \cap B \in \mathbb{S}$ and thus $\mathbb{S}$ is an intersection meet-semilattice with a maximum elements $\mathcal{E}$, hence $\mathbb{S}$ is an intersection lattice.
\end{proof}

We introduce now the supersolvable version of what we stated before to investigate the relation between these notions.

\begin{defn}
    Let $(\mathcal{E}, \mathbb{S})$ be a set system. Let $<$ be a total order on $\mathcal{E}$. We say that $\mathbb{S}$ is a supersolvable antimatroid if:
    \begin{enumerate}
        \item $\emptyset \in \mathbb{S}$;
        \item for every $A, B \in \mathbb{S}$ with $B \not\subseteq A$, for $e = \min_{<}(B \setminus A)$ we have $A \cup \{ e \} \in \mathbb{S}$.
    \end{enumerate}
    We say that $\mathbb{S}$ is supersolvable with respect to $<$.
\end{defn}

\begin{defn} \label{def:scg}
    Let $(\mathcal{E}, \mathbb{S})$ be a set system. Let $<$ be a total order on $\mathcal{E}$. We say that $\mathbb{S}$ is a supersolvable convex geometry if:
    \begin{enumerate}
        \item $\mathcal{E} \in \mathbb{S}$;
        \item for every $A, B \in \mathbb{S}$ with $A \not\subseteq B$, for $e = \min_{<}(A \setminus B)$ we have $A \setminus \{ e \} \in \mathbb{S}$.
    \end{enumerate}
    We say that $\mathbb{S}$ is supersolvable with respect to $<$.
\end{defn}

\begin{prop}
    Let $\mathcal{L}$ be a meet-semilattice and $B \in \mathbb{B}(\mathcal{L})$. Let $<$ be a linear extension of $\mathcal{L}$, then $x = \min_<(B \setminus I(\mathcal{L}))$ we have $B \setminus \{x \} \in \mathbb{B}(\mathcal{L})$.
\end{prop}

We now propose an extension of the previous proposition.

\begin{prop}
    Let $\mathcal{L}$ be a meet-semilattice, then $(\mathcal{L}^+, \mathbb{B}(\mathcal{L}))$ is a supersolvable convex geometry which is supersolvable with respect to any linear extension on $\mathcal{L}$.
\end{prop}

\begin{proof}
    Let $<$ be a linear extension of $\mathcal{L}$. We want to verify both conditions in Definition \ref{def:scg}. We proved that $\mathcal{L}^+ \in \mathbb{B}(\mathcal{L})$, so condition $(1)$ is satisfied.
    
    We now verify condition (2). Consider $B, B' \in \mathbb{B}(\mathcal{L})$ with $B' \not\subseteq B$, and $x=\min_{<}(B' \setminus B)$. We claim $B'\setminus \{x\} \in \mathbb{B}(\mathcal{L})$, and to prove this, we verify the conditions of Proposition \ref{P:mslChar}.
    Condition (1) is simple since \[x \notin B, \, x \notin I(\mathcal{L}) \implies I(\mathcal{L}) \subseteq B' \] and so
    \[ I(\mathcal{L}) \subseteq B \cap B'.\]
    
    Take now $y, z \in B' \setminus \{ x \}$ such that $y \wedge z \neq \hat{0}$. Since $B'$ is a building set by Proposition \ref{P:mslChar} we see that if $y \vee z \in \mathcal{L}$, then $y \vee z \in B'$. We have only to show that $y \wedge z \neq x$, and is sufficient to consider only $y,z < x$. By the minimality of $x$, we know $y,z \in B$, which is a building set, so by Proposition \ref{P:mslChar}, $y \vee z \in B$ thus $y \vee z \neq x$ since $x \notin B$. Hence $y \vee z \in B' \setminus \{x\}$. By Proposition \ref{P:mslChar} we conclude that $B' \setminus \{ x \} \in \mathbb{B}(\mathcal{L})$.
\end{proof}

\begin{cor}
    Let $\mathcal{L}$ be a meet-semilattice, then $\mathbb{B}(\mathcal{L})$ is a supermodular lattice.
\end{cor}

\begin{proof}
    Let $B \in \mathbb{B}(\mathcal{L})$, set $B_0 = B \setminus I(\mathcal{L})$, and define a function $r : \mathcal{B}(\mathcal{L}) \longrightarrow \mathbb{Z}_{\ge 0}$ by $r(B) = |B_0|$.
    Note that $r(I(\mathcal{L})) = 0$. Let $B, B' \in \mathbb{B}(\mathcal{L})$ such that $B'$ covers $B$. By the previous proposition, if $x = \min_{<}(B' \setminus B)$, then $B' \setminus \{ x \}$, thus $B' \setminus \{ x \} \in \mathbb{B}(\mathcal{L})$, then $B' = B \cup \{ x \}$ and $r(B') = r(B) + 1$. Therefore $\mathbb{B}(\mathcal{L})$ is ranked with $r$ as rank function.

    Let $B, B' \in \mathbb{B}(\mathcal{L})$, then
    \[ r(B \vee B') = |(B \vee B')_0| \ge |B_0 \cup B'_0|  \]
    and 
    \[ r(B \wedge B') = |(B \wedge B')_0| = |(B \cap B')_0| = |B_0 \cap B'_0| \]
    together imply
    \[r(B \vee B') + r(B \wedge B') \ge  |B_0 \cup B'_0|  + |B_0 \cap B'_0| = |B_0| + |B'_0| = r(B) + r(B') \]
    
\end{proof}

We now define supersolvable closure operators and characterize supersolvable convex geometry using supersolvable closure operators, and we conclude describing a method to build building sets.

\begin{defn}
    Let $(\mathcal{E}, \mathbb{S}, \sigma)$ a closure system, and let $A \subseteq \mathcal{E}$. An element $x \in A$ is called and extreme point of $A$ if $x \notin \sigma(A \setminus \{ x \})$. We denote the set of extreme points of $A$ as ex$(A)$.
\end{defn}

We have the following characterization of extreme points for building sets.

\begin{prop}
    Let $\mathcal{L}$ be a meet-semilattice and $B \in \mathbb{B}(\mathcal{L})$. Then $x \in ex(B)$ if and only if:
    \begin{enumerate}
        \item $ x \notin I(\mathcal{L})$;
        \item if $y, z \in B \setminus \{ x \}$ such that $y \vee z = x$ then $y \wedge z = \hat{0}$.
    \end{enumerate}
\end{prop}

\begin{proof}
    This is a consequence of Proposition \ref{P:mslChar}.
\end{proof}

\begin{defn}
    Let $\sigma : \mathcal{P}(\mathcal{E}) \longrightarrow \mathcal{P}(\mathcal{E})$ be a closure operator and $<$ a total order on $\mathcal{E}$. We say that $\sigma$ is a supersolvable closure operator (with respect to $\sigma$) if for each closed set $A$ and $e \in \mathcal{E} \setminus A$, we have that:
    \[\text{ if } f \in \sigma(A \cup \{ e \}) \setminus A \cup \{ e \} \text{ then } e<f.\]
\end{defn}

\begin{prop} \label{P:scgChar}
    Let $(\mathcal{E}, \mathbb{S})$ be a set system, then $\mathbb{S}$ is a supersolvable convex geometry if and only if $\mathbb{S}$ is the collection of closed sets of a supersolvable closure operator.
\end{prop}

\begin{proof}
    Suppose that $(\mathcal{E}, \mathbb{S})$ be a supersolvable convex geometry with associated a closure operator $\sigma$. Suppose by contradiction that $\sigma$ is not supersolvable.
    Thus there exits a closed set $A$ and $e, f \in \mathcal{E} \setminus A$ such that $f \in \sigma(A \cup \{ e \}) \setminus A \cup \{ e \}$, and $f < e$.
    Suppose $f$ to be the minimum between such elements with respect to $<$. Applying condition 2 of Definition \ref{def:scg} to the pair of sets $A, \sigma(A \cup \{ e \})$ we have that $\sigma(A \cup \{ e \}) \setminus \{ f \} \in \mathbb{S}$, but this contradicts the definition of $\sigma(A \cup \{ e \})$.

    Viceversa, suppose that $\sigma$ is a supersolvable closure operator on $\mathcal{E}$ with $\mathbb{S}$ the associated closed sets. Let $A, B \in \mathbb{S}$ and let $e = \min_< (B \setminus A)$. Suppose by contradiction that $B \setminus \{ e \}$ is not a closed set. Since $\sigma$ is a closure operator we have that $A \cap B$ is a closed set. Let $e = e_0 < \dots < e_k$ be the elements of $B \setminus A$. Let $i$ be the smallest index such that $e_0 \in \sigma((A \cap B)\cup \{e_1, \dots, e_i \})$, let $X = (A \cap B)\cup \{e_1, \dots, e_{i-1} \}$ then $e_0 \in \sigma(X \cup \{ e_i \}) \setminus (X \cup \{ e_i \})$, but $e_0 < e_i$ contradicting supersolvability of $\sigma$.
\end{proof}

Proposition \ref{P:scgChar} and Proposition \ref{P:mslChar} shows that the closure operator of building sets is supersolvable with respect to any linear extension of $\mathcal{L}$.

We now use Proposition \ref{P:mslChar} to construct $\sigma(X)$.

\begin{prop} \label{P:conBS}
    Let $\mathcal{L}$ be a meet-semilattice and $X \subset \mathcal{L}^+$. Set $X_0 = X \cup I(\mathcal{L})$. Given $X_i$, set $Y_i = \{ a \vee b \in \mathcal{L} : a,b \in X_i \, , \, a \wedge b \neq \hat{0} \}$ and $X_{i+1} := X_i \cup Y_i$. Let $k$ be such that with $X_{k+1} = X_k$, then $\sigma(X) = X_k$.
\end{prop}

\begin{proof}
    Since $\mathcal{L}$ is finite and $X_i \subset X_{i+1}$, there exits $k$ such that $X_k = X_{k+1}$. By definition of $X_k$ and Proposition \ref{P:mslChar} we have that $X_k$ is a building set, thus $\sigma(X) \subseteq X_k$. We prove the converse inclusion by induction.
    Clearly, by definition of $X_0$, $X_0 \subseteq \sigma(X)$. Suppose now that $X_i \subset \sigma(X)$, then $Y_i \subseteq \sigma(X)$ (??), hence $X_{i+1} \subseteq \sigma(X)$. 
\end{proof}

This proposition gives us a method to produce building sets:
\begin{enumerate}
    \item take some $X, X' \subset \mathcal{L}^+$;
    \item construct $\sigma(X), \sigma(X')$ using Proposition \ref{P:conBS};
    \item if $\sigma(X) \not\subseteq \sigma(X')$ by supersolvability we can remove elements from $\sigma(X)$ one at a time. Each intermediate set is a building set.
\end{enumerate}



\newpage
\end{comment}

\section{Geometry of Building Sets} \label{sec:gbs}

In this section we introduce some results about the geometry of the set of building sets. 
Given a lattice $\mathcal{L}$ we denote with $\mathbb{B}(\mathcal{L})$ the family of building sets of $\mathcal{L}$.

\begin{defn}
    Let $\mathcal{L}, \mathcal{K}$ two lattices, then a lattice order embedding of $\mathcal{L}$ and $\mathcal{K}$ is a poset order embedding  $f:\mathcal{L} \longrightarrow \mathcal{K}$ such that:
    \[ \forall x, y \in \mathcal{L}, \, x \wedge^{\mathcal{L}} y = f(x) \wedge^{\mathcal{K}} f(y). \]
\end{defn}

\begin{defn}
    A poset $\mathcal{P}$ such that for all $x,y \in \mathcal{P}$ there exists the meet $x \wedge y \in \mathcal{P}$ is called meet semilattice.
\end{defn}

\begin{rmk}
    Observe that a lattice order embedding need not to respect the joins, in fact if $\mathcal{L}$ is a lattice with a maximal element $\hat{1} \in \mathcal{L}$, then given $x,y \in \mathcal{L}$, we can define:
    \[ x \vee y := \bigwedge_{x, y \le z} z \]
    Note also that if $\mathcal{L}$ is a lattice, and $x,y \in \mathcal{L}$ with $x \le y$, then $\left[x,y \right]$ is a lattice. 
\end{rmk}

\begin{defn}
    Let $\mathcal{E}$ be a set, an intersection lattice is $\mathbb{S} \subset \mathcal{P}(\mathcal{E})$ ordered by inclusion with meet given by:
    \[ A, B \in \mathbb{S}, \, A \wedge B := A \cap B.\]
\end{defn}



We recall here the definitions of $\mathcal{I}(\mathcal{L})$, and $\mathcal{D}(x)$ for a lattice $\mathcal{L}$.

\begin{defn}
    Let $\mathcal{L}$ be a lattice, we define the set:
    \[ I(\mathcal{L}) = \{x \in \mathcal{L} \, : \, [\hat{0}, x] \text{ is irreducible}\} \]
    and the elements of the set $\mathcal{D}(x) := \max I(\mathcal{L})_{\le x}$ are called elementary divisors of $x$.
\end{defn}

\begin{prop} \label{P:mslChar}
    Let $\mathcal{L}$ be a finite lattice and $\mathcal{G} \subset \mathcal{L} \setminus \{\hat{0}\}$. Then $\mathcal{G}$ is a building set if and only if:
    \begin{enumerate}
        \item $I(\mathcal{L}) \subset \mathcal{G}$
        \item if $x, y \in \mathcal{G}$ such that $x \wedge y \neq \hat{0}$, then $x \vee y \in \mathcal{G}$.
    \end{enumerate}
\end{prop}

\begin{proof}
($\implies$) Let $\mathcal{G} \in \mathbb{B}(\mathcal{L})$. As we alredy noted is clear that $\mathcal{I}(\mathcal{L}) \subseteq \mathcal{G}$.
Let $x, y \in \mathcal{G}$ such that $x \vee y \notin \mathcal{G}$. Let $\max \mathcal{G}_{\le x \vee y} = \{ z_1, \dots, z_k \}$.
Since $x, y \in \mathcal{G}$, by definition of $\vee$ we have $x,y \le x\vee y$, then $x$ and $y$ are strictly less than $x \vee y$ as $x \vee y \notin \mathcal{G}$. Therefore there exists $s,t$ with $s \neq t$ such that $x \in \left[ \hat{0}, z_s \right]$ and $y \in \left[ \hat{0}, z_t \right]$.
By definition of building set there is an isomorphism
\[ \phi_{x \vee y} \, : \, \prod_{j=1}^k \left[ \hat{0}, z_j \right] \longrightarrow \left[ \hat{0}, x \vee y \right], \]
and since $\phi_{x \vee y}$ in injective $\bigcap_{j=1}^k \left[ \hat{0}, z_j \right] = \hat{0}$. It follows that $x \wedge y = \hat{0}$.

($\impliedby$) Let $\mathcal{G} \subset \mathcal{L} \setminus \{ \hat{0} \}$ which satisfies the conditions (1) and (2). Let $x \in \mathcal{L} \setminus \{ \hat{0} \}$ and let $\max \mathcal{G}_{\le x} =\{ x_1, \dots, x_k\}$. We wish to show that:
\[ \prod_{j=1} ^k \left[ \hat{0}, x_j \right] \simeq \left[ \hat{0}, x \right].\]
We first observe that for each $1 \le i,j \le k$ with $i \neq j$, $\left[ \hat{0}, x_i \right] \cap \left[ \hat{0}, x_j \right] = \hat{0}$. In fact suppose that exists $z \in \left[ \hat{0}, x_i \right] \cap \left[ \hat{0}, x_j \right]$ with $ z \neq \hat{0}$. Then $x_i \wedge x_j \neq \hat{0}$, since $\hat{0} \neq z \le x_i \wedge x_j$ and $x_i, x_j < x$, so $x_j \vee x_j \in \mathcal{L}$, thus $x_i \vee x_j \in \mathcal{G}$. But $x_i, x_j < x_i \vee x_j$ and $x_i \vee x_j \le x$ which contradicts the maximality of $x_i$ and $x_j$.

Let $\mathcal{D}(x) = \{y_1, \dots, y_l \}$. For each $1 \le t \le k$, let
\[ \pi_t = \{ y_i \in \mathcal{D}(x) \, : \, y_i \le x_t \},\]
we want to prove that the set of $\pi_t$ form a partition of $\mathcal{D}(x)$.
By assumption $I(\mathcal{L}) \subseteq \mathcal{G}$, so for each $y_i$ there exists some $x_t$ such that $y_i \le x_t$, and since the sets $\left[ \hat{0}, x_i \right]$ are all disjoint, we have the uniqueness of $x_t$.
Thus $\{ \pi_t \}_{t=1}^k$ is a partition of $\mathcal{D}(x)$.

Last thing we want to prove is $\mathcal{D}(x_t) = \pi_t$. Let $y \in \mathcal{D}(x_t)$, and let $y_i$ be such that $y \le y_i$. Then there exists a unique $t$ such that $y_i \le x_t$, and so $y = y_i$. Therefore we know that there exists two isomorphisms
\[ \phi_x : \prod_{j=1}^l \left[ \hat{0}, y_j \right] \longrightarrow \left[ \hat{0}, x \right] \]
and
\[ \phi_{x_t} : \prod_{y \in \pi_t} \left[ \hat{0}, y \right] \longrightarrow \left[ \hat{0}, x_t \right] \]
and so the map
\[ \phi_x \circ \prod_{t = 1}^k (\phi_{x_t})^{-1} : \prod_{t=1}^k \left[ \hat{0}, x_t \right] \longrightarrow \left[ \hat{0}, x \right] \]
is an isomorphism.

\end{proof}

We want now to investigate the poset $(\mathbb{B}(\mathcal{L}), \subseteq)$.
The first thing we prove is the fact that $(\mathbb{B}(\mathcal{L}), \subseteq)$ is an intersection lattice.
We first prove a preliminary lemma.

\begin{lem}
    Let $\mathcal{L}$ a finite meet semilattice with maximal element $\hat{1}$, then $\mathcal{L}$ is a lattice.
\end{lem}

\begin{proof}
    It is sufficient to define, given $x,y \in \mathcal{L}$, the join
    \[ x \vee y := \bigwedge_{x, y \le z} z. \]
    Let $W := \{ w \in \mathcal{L} \, : \, w \ge x, \, w \ge y\}$, observe that $W \neq \emptyset$ since $\hat{1} \in W$. Then $x$ and $y$ are both lower bound of $W$, so $x \vee y \ge x,y$.
    Let $w'$ be an upper bound for $x$ and $y$ then $w' \in W$ and $w' \ge x \vee y$.
\end{proof}

\begin{prop}
    Let $\mathcal{L}$ a finite lattice. The poset $(\mathbb{B}(\mathcal{L}), \subseteq)$ is an intersection lattice.
\end{prop}

\begin{proof}
    We start by proving that the intersection of two building sets is a building sets. Let $B, B' \in \mathbb{B}(\mathcal{L})$ two building sets, by Proposition \ref{P:mslChar} we have that $I(\mathcal{L}) \subseteq B, B'$ so $I(\mathcal{L}) \subseteq B \cap B'$ thus condition 1 of Proposition  \ref{P:mslChar}  is satisfied. 
    
    Let now $x, x' \in B \cap B'$ such that $x \wedge x' \neq \hat{0}$. Suppose $x \vee x' \in \mathcal{L}$ then since $B$ and $B'$ are building sets we have that $x \vee x' \in B$ and $x \vee x' \in B'$, so $x \vee x' \in B \cap B'$. Thus also condition 2 of Proposition \ref{P:mslChar} is satisfied, so $B \cap B'$ is a building set.
    We proved that $(\mathbb{B}(\mathcal{L}), \subseteq)$ is an intersection meet-semilattice, and we already know that $\mathcal{L} \setminus \{ \hat{0} \}$ is the maximal building sets so $(\mathbb{B}(\mathcal{L}), \subseteq)$ is an intersection lattice.
\end{proof}

In the following we define set systems and closure systems which we will use to define convex geomtries.

\begin{defn}
    We define a set system as a pair $(\mathcal{E}, \mathbb{S})$ where $\mathcal{E}$ is a set and $\mathbb{S} \subset \mathcal{P}(\mathcal{E})$. We refer to $\mathcal{E}$ as ground set.
\end{defn}

\begin{es} 
    \begin{enumerate}  
        \item[$ $]
        \item Topological spaces are set systems.
    
        \item Matroids are set systems.
    \end{enumerate} 
\end{es}

\begin{defn}
    Let $\mathcal{E}$ a set, a clousure operator on $\mathcal{E}$ is a map $\sigma : \mathcal{P}(\mathcal{E}) \longrightarrow \mathcal{P}(\mathcal{E})$ such that:
    \begin{enumerate}
        \item $A \subseteq \sigma(A)$;
        \item $A \subseteq B \implies \sigma(A) \subseteq \sigma(B)$;
        \item $\sigma(\sigma(A)) = \sigma(A)$.
    \end{enumerate}
We say that $A \in \mathcal{P}(\mathcal{E})$ is closed if $\sigma(A) = A$.
\end{defn}

\begin{defn}
    Let $\sigma$ be a closure operator on the ground set $\mathcal{E}$, define \[ \mathbb{S} = \{ A \in \mathcal{P}(\mathcal{E}) \, : \, \sigma(A) = A \} \] the closed subsets of $\mathcal{E}$. Then $(\mathcal{E}, \mathbb{S}, \sigma)$ is called closure system.
\end{defn}

\begin{es}
    \begin{enumerate}
        \item[$ $]
        \item Let $(X, \tau)$ a topological space, define on $\mathcal{P}(X)$ the operator $cl(A)$ as the smallest closed subset of $X$ containing $A$ i.e.
        \[ cl(A) := \bigcap_{\substack{K \supseteq A \\ K \text{ closed}}} K. \]
        Then cl is a closure operator, in fact 
        \begin{enumerate}
            \item $A \subset$ $cl(A)$ by definition;
    
            \item let $A \subset B$ then the smallest closed subset containing $B$ contains also $A$ and the smallest closed subset containing $A$ is contained in the smallest closed subset containing $B$, so $cl(A)$ $\subseteq$ $cl(B)$;
    
            \item since $cl(A)$ is by definition closed, the smallest closed subset containing $cl(A)$ is $cl(A)$, so $cl(cl(A)) = cl(A)$. 
        \end{enumerate}
    
        Defining $\mathcal{C}(X) = \{ A \subset X \, | \, cl(A) = A\}$,  then $(X, \mathcal{C}(X), cl)$ is a closure system.

        \item  Consider $\mathcal{E} = \mathbb{R}^n$ and $\mathbb{S} = \{ A \subset \mathbb{R}^n \, | \, \text{A is convex} \}$. Define the closure operator $co(A)$ as the smallest convex set of $\mathbb{R}^n$ containing $A$.
        
        \item The operator $T$ defined in the Definition \ref{def:realmat} used to define a realizable matroid satisfy the conditions of closure operator.
    \end{enumerate}
\end{es}

Using closure systems we can also define matroids.
\begin{defn}
    A matroid $\mathcal{M}$ on a set $X$ is a closure operator
    \[ cl : \mathcal{P}(X) \longrightarrow \mathcal{P}(X)\]
    such that for all $x, y \in X$ and $S \subseteq X$
    \[ a \in cl(S \cup \{b\}) \setminus cl(S) \implies b \in cl(S \cup \{a\}) \setminus cl(S).\]
\end{defn}

Observe now that a closure system $(\mathcal{E}, \mathbb{S}, \sigma)$ comes with a natural meet an join. Define in fact $A \wedge B = A \cap B$ and $A \vee B = \sigma(A \cup B)$, and so $\mathbb{S}$ is an intersection lattice. 

Viceversa, given an intersection lattice $\mathbb{S}$ with ground set $\mathcal{E}$ we can define a closure operator $\sigma : \mathcal{P}(\mathcal{E}) \longrightarrow \mathcal{P}(\mathcal{E})$ as
\[ \sigma(A) = \bigcap_{ \substack{C \in \mathbb{S} \\ A \subseteq C}} C.\]
This induces a canonical identification between closure operator and intersection lattices.
In particular a closure system $(\mathcal{E}, \mathbb{S}, \sigma)$ is encoded by the pair $(\mathcal{E}, \mathbb{S})$ alone.

Since $\mathbb{B}(\mathcal{L})$ is an intersection lattice it induces a closure operator defined as
\[ \sigma_B(X) :=  \bigcap_{\substack{B \in \mathbb{B}(\mathcal{L}) \\ X \subseteq B}} B,\]
for all $X \in \mathcal{P}(\mathcal{L})$, and we have that $A \subseteq \mathcal{L}$ is a building set if and only if $\sigma_B(A) = A$, i.e.
\[ \mathbb{B}(\mathcal{L}) = \{ A \subset \mathcal{L} \, : \, \sigma_B(A) = A\}.\]

We can now define convex geometries.

\begin{defn}
    Let $(\mathcal{E}, \mathbb{S}, \sigma)$ a closure system. We say that $(\mathcal{E}, \mathbb{S}, \sigma)$ si a convex geometry if for all $A \in \mathbb{S}$ and for $x, y \notin A$, $x \neq y$, we have that \[y \in \sigma(A \cup \{x \})\implies x \notin \sigma(A \cup \{ y \}).\] 
In this case we call $\sigma$ an anti-exchange closure operator.
\end{defn}

Convex geometries are a combinatorial abstraction of convex sets in $\mathbb{R}^n$.

\begin{es}
    Consider $\mathcal{E} = \mathbb{R}^n$ and $\mathbb{S} = \{ A \subset \mathbb{R}^n \, | \, \text{A is convex} \}$.
    Then $(\mathbb{R}^n, \mathbb{S}, co)$ is a convex geometry.

    In fact suppose that $A \subseteq \mathbb{R}^n$ convex, and consider $x,y \in A$, $x \neq y$. Suppose that $y \in co(A\cup \{x\})$, and by contradiction that $x \in co(A\cup \{y\})$. Then since $A \subseteq co(A \cup \{x\}) \cap co(A \cup \{y\})$, we have
    \[ x \in co(A \cup \{y\}) \implies co(A \cup \{x\}) \subseteq co(A \cup \{y\})\]
    and
    \[ y \in co(A \cup \{x\}) \implies co(A \cup \{y\}) \subseteq co(A \cup \{x\}),\]
    so $co(A \cup \{y\}) = co(A \cup \{x\})$. Contradiction since $x\neq y$.
\end{es}

We want now to define antimatroids and investigate the connection with convex geometries. 

\begin{defn}
    An antimatroid is a set system $(\mathcal{E}, \mathbb{S})$ sucht that:
    \begin{enumerate}
        \item $\emptyset \in \mathbb{S}$
        \item for every $A, B \in \mathbb{S}$ such that $ B \not\subset A$, there exists $e \in B \setminus A$ such that $A \cup \{ e \} \in \mathbb{S}$.
    \end{enumerate}
\end{defn}

\begin{es} A simple example of antimatroid is:

    \begin{figure}
    \begin{center} \begin{tikzpicture}
            
        \node (top) at (0,0) {$\{a,b,c,d\}$};
           
        \node (n1) at (0, -2)  {$\{a,b,c \}$};

        \draw[thick] (top) -- (n1);
          
        \node (n2) at (-2, -4)  {$\{a,b\}$};
        \node (n3) at (0, -4)  {$\{a,c\}$};
        \node (n4) at (2, -4)  {$\{b,c\}$};

        \draw[thick] (n1) -- (n2);
        \draw[thick] (n1) -- (n3);
        \draw[thick] (n1) -- (n4);

        \node (n5) at (-2, -6)  {$\{a\}$};
        \node (n6) at (2, -6)  {$\{c\}$};

        \draw[thick] (n2) -- (n5);
        \draw[thick] (n3) -- (n5);
        \draw[thick] (n3) -- (n6);
        \draw[thick] (n4) -- (n6);
        
        \node (bottom) at (0, -8) {$\emptyset$};
            
        \draw[thick] (n5) -- (bottom);
        \draw[thick] (n6) -- (bottom);
            
    \end{tikzpicture} \end{center}
    \caption{The Hasse diagram of an antimatroid.}
    \label{fig:antimat}
    \end{figure}
\end{es}
Let's define the dual notion of set system.

\begin{defn}
    Let $(\mathcal{E}, \mathbb{S})$ be a set system. We define the complementary set system of $(\mathcal{E}, \mathbb{S})$ as the set system $(\mathcal{E}, \mathbb{S}^c).$
\end{defn}

It's possible to prove that a set system $(\mathcal{E}, \mathbb{S})$ is an antimatroid if and only if the complementary set system $(\mathcal{E}, \mathbb{S}^c)$ is a convex geometry. Thus we can alternately define convex geometries as the dual notion of antimatroids.

\begin{defn} \label{def:convgeo2}
    A convex geometry is a set system $(\mathcal{E}, \mathbb{S})$ such that:
    \begin{enumerate}
        \item $\mathcal{E} \in \mathbb{S}$;
        \item for every $A, B \in \mathbb{S}$ such that $A \not\subseteq B$, there exists $e \in A\setminus B$ such that $A \setminus \{ e \} \in \mathbb{S}$.
    \end{enumerate}
\end{defn}

We can use the latter definition of convex geometry to give an example on poset.

\begin{es} \label{es:upperset}
Let $(\mathcal{P}, \le)$ be a finite poset. We call a subset $U \subset \mathcal{P}$ an upper set if for all $u  \in U$ and $x \in \mathcal{P}$ 
\[ x \ge u \implies x \in U.\]
Define now the family of all upper set in $\mathcal{P}$
\[ \mathcal{U} := \{ U \subset \mathcal{P}  \, : \, U \text{ is an upper set}\}.\]
A principal upper set is an upper set of the form 
\[ I_{x} := \{ z \in \mathcal{P} \, : \, y \ge x \}.\]
We now use Definition \ref{def:convgeo2} to prove that $(\mathcal{P}, \mathcal{U})$ is a convex geometry.
The condition $(1)$ is simple since $\mathcal{P} = \mathcal{I}_{\hat{0}}$.

Consider now $U, V \in \mathcal{U}$, such that $U \not \subseteq V$, we want to prove that exists $z \in U \setminus V$.
Consider the set $\min U := \{z_1, \dots,  z_t\}$ the set of minimum element of $U$ and suppose $\forall i \in \{1, \dots, t\}$, $z_i \in V$. Then for all $u \in U$, then exists $z_i \in \min U$ such that $u \ge z_i \in V$, so by definition of upper set $u \in V$. Thus $U \subseteq V$, which is a contradiction.
So there exists a minimum element $z \in U$ such that $z \notin V$, and since $z$ is a minimum for $U$ we have that $U \setminus \{z\} \in \mathcal{U}$. 

Thus $(\mathcal{P}, \mathcal{U})$ is a convex geometry.
\end{es}



We now prove that a convex geometry is an intersection lattice.

\begin{prop} \label{prop:convgeointla}
    Let $(\mathcal{E}, \mathbb{S})$ be a convex geometry then $\mathbb{S}$ is an intersection lattice.
\end{prop}

\begin{proof}
    Let $A, B \in \mathbb{S}$. By condition (2) of Definition \ref{def:convgeo2} we can take an element $e \in A \setminus B$, remove it from $A$ and obtain that $A \setminus\{e\} \in \mathbb{S}$. Repeating this process we can remove all elements in $A \setminus B$ and obtain the set $A \cap B$. By condition (2) we have $A \cap B \in \mathbb{S}$. Hence given $A, B \in \mathbb{S}$ we have $A \cap B \in \mathbb{S}$ and thus $\mathbb{S}$ is an intersection meet-semilattice with a maximum elements $\mathcal{E}$, hence $\mathbb{S}$ is an intersection lattice.
\end{proof}

We introduce now the supersolvable version of antimatroids, convex geometries and closure operator, and investigate their relation with building sets.

\begin{defn}
    Let $(\mathcal{E}, \mathbb{S})$ be a set system. Let $<$ be a total order on $\mathcal{E}$. We say that $\mathbb{S}$ is a supersolvable antimatroid if:
    \begin{enumerate}
        \item $\emptyset \in \mathbb{S}$;
        \item for every $A, B \in \mathbb{S}$ with $B \not\subseteq A$, for $e = \min_{<}(B \setminus A)$ we have $A \cup \{ e \} \in \mathbb{S}$.
    \end{enumerate}
    We say that $\mathbb{S}$ is supersolvable with respect to $<$.
\end{defn}

Thus one has the following dual definition of supersolvable convex geometries.

\begin{defn} \label{def:scg}
    Let $(\mathcal{E}, \mathbb{S})$ be a set system. Let $<$ be a total order on $\mathcal{E}$. We say that $\mathbb{S}$ is a supersolvable convex geometry if:
    \begin{enumerate}
        \item $\mathcal{E} \in \mathbb{S}$;
        \item for every $A, B \in \mathbb{S}$ with $A \not\subseteq B$, for $e = \min_{<}(A \setminus B)$ we have $A \setminus \{ e \} \in \mathbb{S}$.
    \end{enumerate}
    We say that $\mathbb{S}$ is supersolvable with respect to $<$.
\end{defn}

\begin{prop} \label{prop:buildsetlinearext}
    Let $\mathcal{L}$ be a finite lattice and $B \in \mathbb{B}(\mathcal{L})$. Let $<$ be a linear extension of $\mathcal{L}$, then $x = \min_<(B \setminus I(\mathcal{L}))$ we have $B \setminus \{x \} \in \mathbb{B}(\mathcal{L})$.
\end{prop}

Proposition \ref{prop:buildsetlinearext} is a direct conseguence of Definition \ref{def:scg} and next proposition which is, in fact, an extension of the latter.

\begin{prop} \label{prop:bsetsupersolvale}
    Let $\mathcal{L}$ be a finite lattice, then $(\mathcal{L}\setminus\{\hat{0}\}, \mathbb{B}(\mathcal{L}))$ is a supersolvable convex geometry with respect to any linear extension on $\mathcal{L}$.
\end{prop}

\begin{proof}
    Let $<$ be a linear extension of $\mathcal{L}$. We want to verify both conditions in Definition \ref{def:scg}. We proved that $\mathcal{L}\setminus\{\hat{0}\} \in \mathbb{B}(\mathcal{L})$, so condition $(1)$ is satisfied.
    
    We now verify condition (2). Consider $B, B' \in \mathbb{B}(\mathcal{L})$ with $B' \not\subseteq B$, and $x=\min_{<}(B' \setminus B)$. We claim that $B'\setminus \{x\} \in \mathbb{B}(\mathcal{L})$, and to prove this, we verify the conditions of Proposition \ref{P:mslChar}.
    Condition (1) is simple since \[x \notin B, \, x \notin I(\mathcal{L}) \implies I(\mathcal{L}) \subseteq B' \] and so
    \[ I(\mathcal{L}) \subseteq B \cap B'.\]
    
    Take now $y, z \in B' \setminus \{ x \}$ such that $y \wedge z \neq \hat{0}$. Since $B'$ is a building set by Proposition \ref{P:mslChar} we see that if $y \vee z \in \mathcal{L}$, then $y \vee z \in B'$. We have only to show that $y \wedge z \neq x$. It is sufficient to consider only $y,z < x$. By the minimality of $x$, we know $y,z \in B$, which is a building set, so by Proposition \ref{P:mslChar}, $y \vee z \in B$ thus $y \vee z \neq x$ since $x \notin B$. Hence $y \vee z \in B' \setminus \{x\}$. By Proposition \ref{P:mslChar} we conclude that $B' \setminus \{ x \} \in \mathbb{B}(\mathcal{L})$.
\end{proof}

\begin{cor}
    Let $\mathcal{L}$ be a finite lattice, then $\mathbb{B}(\mathcal{L})$ is a supermodular lattice.
\end{cor}

\begin{proof}
    Let $B \in \mathbb{B}(\mathcal{L})$, set $B_0 = B \setminus I(\mathcal{L})$, and define a function $r : \mathcal{B}(\mathcal{L}) \longrightarrow \mathbb{Z}_{\ge 0}$ by $r(B) = |B_0|$.
    Note that $r(I(\mathcal{L})) = 0$. Let $B, B' \in \mathbb{B}(\mathcal{L})$ such that $B'$ covers $B$. By the previous proposition, if $x = \min_{<}(B' \setminus B)$, then $B' \setminus \{ x \}$, thus $B' \setminus \{ x \} \in \mathbb{B}(\mathcal{L})$, then $B' = B \cup \{ x \}$ and $r(B') = r(B) + 1$. Therefore $\mathbb{B}(\mathcal{L})$ is ranked with $r$ as rank function.

    Let $B, B' \in \mathbb{B}(\mathcal{L})$, then
    \[ r(B \vee B') = |(B \vee B')_0| \ge |B_0 \cup B'_0|  \]
    and 
    \[ r(B \wedge B') = |(B \wedge B')_0| = |(B \cap B')_0| = |B_0 \cap B'_0| \]
    together imply
    \[r(B \vee B') + r(B \wedge B') \ge  |B_0 \cup B'_0|  + |B_0 \cap B'_0| = |B_0| + |B'_0| = r(B) + r(B'). \qedhere\] 
    
\end{proof}

We now define supersolvable closure operators and characterize supersolvable convex geometry using supersolvable closure operators. We conclude by describing a method to construct building sets.

\begin{defn}
    Let $(\mathcal{E}, \mathbb{S}, \sigma)$ a closure system, and let $A \subseteq \mathcal{E}$. An element $x \in A$ is called and extreme point of $A$ if $x \notin \sigma(A \setminus \{ x \})$. 
    
    We denote the set of extreme points of $A$ as ex$(A)$.
\end{defn}

We have the following characterization of extreme points for building sets.

\begin{prop}
    Let $\mathcal{L}$ be a finite lattice and $B \in \mathbb{B}(\mathcal{L})$. Then $x \in ex(B)$ if and only if:
    \begin{enumerate}
        \item $ x \notin I(\mathcal{L})$;
        \item if $y, z \in B \setminus \{ x \}$ such that $y \vee z = x$ then $y \wedge z = \hat{0}$.
    \end{enumerate}
\end{prop}

\begin{proof}
    This is a consequence of Proposition \ref{P:mslChar} and definition of the closure operator for building set $\sigma_B$.
\end{proof}

\begin{defn}
    Let $\sigma : \mathcal{P}(\mathcal{E}) \longrightarrow \mathcal{P}(\mathcal{E})$ be a closure operator and $<$ a total order on $\mathcal{E}$. We say that $\sigma$ is a supersolvable closure operator (with respect to $\sigma$) if for each closed set $A$ and $e \in \mathcal{E} \setminus A$, we have that:
    \[f \in \sigma(A \cup \{ e \}) \setminus A \cup \{ e \} \implies  e<f.\]
\end{defn}

\begin{prop} \label{P:scgChar}
    Let $(\mathcal{E}, \mathbb{S})$ be a set system, then $\mathbb{S}$ is a supersolvable convex geometry if and only if $\mathbb{S}$ is the collection of closed sets of a supersolvable closure operator.
\end{prop}

\begin{proof}
    Suppose that $(\mathcal{E}, \mathbb{S})$ be a supersolvable convex geometry with associated a closure operator $\sigma$. Suppose by contradiction that $\sigma$ is not supersolvable.
    Thus there exits a closed set $A$ and $e, f \in \mathcal{E} \setminus A$ such that \[f \in \sigma(A \cup \{ e \}) \setminus A \cup \{ e \}\text{ and } f < e.\]
    Suppose $f$ to be the minimum between such elements with respect to $<$. Applying condition (2) of Definition \ref{def:scg} to the pair of sets $A, \sigma(A \cup \{ e \})$ we have that \[\sigma(A \cup \{ e \}) \setminus \{ f \} \in \mathbb{S},\] but this contradicts the definition of $\sigma(A \cup \{ e \})$.

    Viceversa, suppose that $\sigma$ is a supersolvable closure operator on $\mathcal{E}$ with $\mathbb{S}$ the associated family of closed sets. Let $A, B \in \mathbb{S}$ and let $e = \min_< (B \setminus A)$. 
    
    Suppose by contradiction that $B \setminus \{ e \}$ is not a closed set. Since $\sigma$ is a closure operator we have that $A \cap B$ is a closed set.
    Let $e = e_0 < \dots < e_k$ be the elements of $B \setminus A$ and $i$ be the smallest index such that $e_0 \in \sigma((A \cap B)\cup \{e_1, \dots, e_i \})$. Let $X = (A \cap B)\cup \{e_1, \dots, e_{i-1} \}$ then $e_0 \in \sigma(X \cup \{ e_i \}) \setminus (X \cup \{ e_i \})$ but $e_0 < e_i$, contradicting the supersolvability of $\sigma$.
\end{proof}

Proposition \ref{P:scgChar} and Proposition \ref{P:mslChar} shows that the closure operator of building sets is supersolvable with respect to any linear extension of $\mathcal{L}$.

\begin{es}
    Consider $(\mathcal{P}, \mathcal{U})$ of Example \ref{es:upperset} where $\mathcal{U}$ is the family of upper set of $\mathcal{P}$. Then by Proposition \ref{prop:convgeointla} is an intersection lattice, i.e.
    \[ U,V \in \mathcal{U} \implies U \cap V \in \mathcal{U}.\]
    Define for $A \subset \mathcal{P}$ the closure operator
    \[ \sigma(A) := \bigcap_{\substack{U \in \mathcal{U} \\ A \subset U}} U.\]
    So $\sigma(A) \in \mathcal{U}$ for all $A \subset \mathcal{P}$, an observe that \[ \mathcal{U} = \{ U \subseteq  \mathcal{P} \, : \, \sigma(U) = U \}.\]
    If we prove that $\sigma$ si a supersolvable closure operator then by Proposition \ref{P:scgChar} $(\mathcal{P}, \mathcal{U})$ is a supersolvable convex geometry.
    Consider $U \in \mathcal{U}$ closed with respect to $\sigma$ and take $e \in \mathcal{P} \setminus U$. Since $U$ is closed in the set $\sigma(U \cup \{ e\})$ we are only adding all the elements greater than $e$.
    Consider $f \in \sigma(U \cup \{e \}) \setminus U \cup \{e \}$ we have that $f \in I_e \setminus \{e\}$, so $f > e$.
    So $\sigma$ si a supersolvable closure operator.
    
\end{es}


We finish with an algorithmic procedure based on Proposition \ref{P:mslChar} to construct building sets. 
Thus, given $\mathcal{L}$ a finite lattice and $X \subset \mathcal{L} \setminus \{ \hat{0} \}$, we construct $\sigma_B(X)$.

\begin{prop} \label{P:conBS}
    Let $\mathcal{L}$ be a finite lattice and $X \subset \mathcal{L} \setminus \{\hat{0}\}$. Set 
    \begin{enumerate}
        \item $X_0 = X \cup I(\mathcal{L})$;

        \item given $X_i$, set \[Y_i = \{ a \vee b \in \mathcal{L} : a,b \in X_i \, , \, a \wedge b \neq \hat{0} \}\] and \[X_{i+1} := X_i \cup Y_i.\]
    \end{enumerate}  Let $k$ be such that $X_{k+1} = X_k$, then $\sigma_B(X) = X_k$.
\end{prop}

\begin{proof}
    Since $\mathcal{L}$ is finite and $X_i \subset X_{i+1}$, there exits $k$ such that $X_k = X_{k+1}$. By definition of $X_k$, $X \subseteq X_k$, and by Proposition \ref{P:mslChar} we have that $X_k$ is a building set, thus $\sigma_B(X) \subseteq X_k$. 
    
    We prove the converse inclusion by induction.
    Clearly, by definition, $X_0 \subseteq \sigma_B(X)$. Suppose now that $X_i \subset \sigma_B(X)$. Then by Proposition \ref{P:mslChar}, $Y_i \subseteq \sigma_B(X)$, hence $X_{i+1} \subseteq \sigma_B(X)$. So $X_k = \sigma_B(X).$
\end{proof}

This proposition gives us an algorithmic method to produce building sets:
\begin{enumerate}
    \item take some $X, X' \subset \mathcal{L}^+$;
    \item construct $\sigma_B(X), \sigma_B(X')$ using Proposition \ref{P:conBS};
    \item if $\sigma(X) \not\subseteq \sigma_B(X')$ by supersolvability we can remove elements from $\sigma_B(X)$ one at a time. Each intermediate set is a building set.
\end{enumerate}





\newpage
\chapter{Chow ring, monomial basis and Kahler package}

In this chapter we define the Chow ring $\mathcal{A}(\mathcal{L}, \mathcal{G})$ associated to a lattice $\mathcal{L}$ and a building set $\mathcal{G}$. Then we provide a monomial basis for $\mathcal{A}(\mathcal{L}, \mathcal{G})$ and discuss the Kahler package. We finish with a brief introduction to De Concini-Procesi wonderful model and a geometrical interpretations for $\mathcal{A}(\mathcal{L}, \mathcal{G})$.

\section{Chow ring and monomial basis}

In this section we introduce a new object: the Chow ring. This object is the main object of this work which has also some important geometric interpretation which we will see later.
Later we provide a monomial basis for the Chow Ring.

We refer to Section\ref{sec12} for the definition of building sets.
Consider the ring $S:= \mathbb{Z}[x_F]_{F \in \mathcal{G}}.$
\begin{defn}
    Let $\mathcal{L}$ a finite lattice and $\mathcal{G} \subset \mathcal{L}$ a building set. We define two ideals in $S$:
    \begin{enumerate}
        \item the Stanley-Reisner ideal $I \subset S$ associated to $\mathcal{N(\mathcal{L}, \mathcal{G})}$ as the ideal generated by elements \[x_{F_1} \cdots x_{F_l}\] for $\{x_{F_1}, \dots , x_{F_l} \} \subset \mathcal{G}$ which are not $\mathcal{G}$-nested set;
    
        \item the ideal $J$ as the ideal generated by elements \[\sum_{a \le F \in \mathcal{G}} x_F\] for every atom $a \in \mathcal{L}$.
    \end{enumerate}
\end{defn}

Now we can define the Chow ring.

\begin{defn}
    Let $\mathcal{L}$ a finite lattice and $\mathcal{G} \subset \mathcal{L}$ a building set, the Chow ring $\mathcal{A}( \mathcal{L}, \mathcal{G})$ is the quotient
    \[ \mathcal{A}( \mathcal{L}, \mathcal{G}) := \frac{S}{I + J}. \]
\end{defn}

Observe now that the ideals $I$ and $J$ are both graded ideals, generated by homogeneous elements of $S$, and $S$ is a graded $\mathbb{Z}$-algebra, with grading given by the degree $\deg(x_F) = 1$. Thus $\mathcal{A}$($\mathcal{L}$, $\mathcal{G}$) inherits a structure of graded $\mathbb{Z}$-algebra, that is \[\mathcal{A}(\mathcal{L}, \mathcal{G}) = \bigoplus_{n = 1}^{\infty} \,  A^{n}\] where $A^{n}$ is the $\mathbb{Z}$-module obtained quotienting the homogeneous polynomial of degree $n$ in $S$.

Consider now Aut($\mathcal{L})$ the group of automorphism of $\mathcal{L}$ and a subgroup $G \subset$Aut($\mathcal{L})$. From now on we assume that the building set $\mathcal{G} \subset \mathcal{L} \setminus \{ \hat{0} \}$ is setwise stabilized by $G$, i.e.
\[    g(F) \in \mathcal{G} \, \text{ for all } F \in \mathcal{G} \text{ and } g \in G   \]
or, equivalently: for all $g \in G$, $g(\mathcal{G}) = \mathcal{G}$.

Define now the action of $G$ on $S$ by $g(x_F) := x_{g(F)}$ and extended by algebra automorphism. We want to show that the action descent to a $\mathbb{Z}$-algebra automorphism of $\mathcal{A}$($\mathcal{L}$, $\mathcal{G}$).

An automorphism of poset $f:\mathcal{L} \longrightarrow \mathcal{L}$ has the property:
\[ x \le_{\mathcal{L}} y \iff f(x) \le_{\mathcal{L}} f(y). \]
We show the following fact:
\begin{lem}
    Let $\mathcal{L}$ be a finite lattice, $f \in$ Aut($\mathcal{L}$), if $a \in \mathcal{L}$ is an atom then also is $f(a)$.
\end{lem}

\begin{proof}
Suppose $a$ is an atom of $\mathcal{L}$, this is equivalent to $a \gtrdot \hat{0}$. This means that there does not exist $z \in \mathcal{L}$ such that $a > z > \hat{0}$. So consider now $f(a) > \hat{0}$, and suppose that there exists $w \in \mathcal{L}$ such that $f(a) > w > \hat{0}$. Since $f$ is bijective, there must be $y \in \mathcal{L}$ such that $f(y) = w$, and so $f(a) > f(y) > \hat{0}$.
Since $f$ is an automorphism, $f^{-1}$ is also an automorphism of $\mathcal{L}$, so applying $f^{-1}$ to $f(a) > f(y) > \hat{0}$ we must have $a > y > \hat{0}$, which is a contradiction whit the hypothesis that $a$ is an atom.
\end{proof}

The above Lemma reveals that a function $g \in$Aut($\mathcal{L}$) preserves atoms in the lattice $\mathcal{L}$, and so sends generators of $J$ to generators of $J$.

We recall now that if $f \in$Aut($\mathcal{L}$) then has the property:
\[ f(x \vee y)= f(x)\vee f(y).\]

\begin{lem}
Let $\mathcal{L}$ be a finite lattice, $\mathcal{G} \subset \mathcal{L}$ a building set, $f \in$Aut($\mathcal{L}$) setwise preserving $\mathcal{G}$ and $N \subset \mathcal{G}$ a $\mathcal{G}$-nested set then $f(N)$ is a $\mathcal{G}$-nested set.   
\end{lem}

\begin{proof}
    The map $f$ preserves $\mathcal{G}$ pointwise so $f(N) \subset \mathcal{G}$.
    
    Consider $\{K_1, \dots, K_t \} \subset f(N)$ a set of pairwise incomparable elements of $\mathcal{L}$, since $f \in$Aut($\mathcal{L}$) there exists $\{ F_1, \dots, F_t\} \subset N$ of pairwise incomparable elements such that $f(F_i) = K_i$ for all $i \in \{ 1, \dots, t \}$, and by definition of $\mathcal{G}$-nested set $F_1 \vee \dots \vee F_t \notin \mathcal{G}$, so:
    \[ K_1 \vee \dots \vee K_t = f(F_1) \vee \dots \vee f(F_t) = f(F_1 \vee \dots \vee F_t) \notin f(\mathcal{G}).\]
    Since $f(\mathcal{G}) = \mathcal{G}$ we have $K_1 \vee \dots \vee K_t \notin \mathcal{G}$.
\end{proof}

The latter Lemma tells us that a function $g \in$Aut($\mathcal{L}$) setwise preserving $\mathcal{G}$, send $\mathcal{G}$-nested sets in $\mathcal{G}$-nested sets so sends generators of $I$ in generators of $I$. 

To summarize we proved that if $g \in$Aut($\mathcal{L}$) setwise preserve $\mathcal{G}$, then the action of $g$ preserves the ideal $I$ and $J$, so $g$ acts as a $\mathbb{Z}$-algebra automorphism on $\mathcal{A}$($\mathcal{L}$,$\mathcal{G}$).
Thus in the decomposition $\mathcal{A}$($\mathcal{L}$, $\mathcal{G}$) = $\bigoplus_{n = 1}^{\infty} \, A^{n}$, every $A^{n}$ has a $\mathbb{Z}G$-module structure.

Now we provide a monomial basis for the Chow ring.
 We start by recalling the definition of monomial order and Gröbner basis. For more information about Grobner basis see \cite{usingalg}.

\begin{defn}
    Let $(X, \prec)$ a poset, then we say that $\prec$ is a well-ordering if every non-empty subset of $X$ has a $\prec$-minimum element.
\end{defn}

\begin{es}
    \begin{enumerate}
        \item The set $\mathbb{N}$ with the usual order is a well-ordering.
        \item The set $\mathbb{Z}$ with the usual order is not a well-ordering, in fact the set $\{ 2n \, : \, n \in \mathbb{Z}\}$ does not have a minimum.
    \end{enumerate}
\end{es}

\begin{defn}
    A linear order $\prec$ on the set of all monomials $\ \textbf{x}^\alpha := x_1^{\alpha_1} \cdots x_n^{\alpha_n}$ in $\mathbb{Z}[\textbf{x}]$ is called a monomial ordering if it is a well-ordering and:
    \[ \textbf{x}^{\alpha} \prec \textbf{x}^\beta \implies \textbf{x}^\alpha \textbf{x}^\gamma \prec \textbf{x}^\beta \textbf{x}^\gamma, \]
    for all $\alpha, \gamma, \beta \in \{ 0, 1, \dots\}^n$.
\end{defn}

\begin{es}
    Fix a linear order on the variable $x_1 < \dots < x_n$, we define the lexicographic order as:
    \[ x_1^{\alpha_1} \cdots x_n^{\alpha_n} \prec x_1^{\beta_1} \cdots x_n^{\beta_n}  \iff \exists i \in \{ 1, \dots, n \} \, \text{ such that } \, \alpha_k = \beta_k , \; \forall k<i \text{ and } \alpha_i = \beta_i.  \]
    Then the lexicographic order is a monomial ordering, in fact:
    consider $\{ \alpha_1, \dots \, \alpha_n \}$, $\{ \beta_1, \dots , \beta_n \}$ and $\{ \gamma_1, \dots , \gamma_n \}$ in $\{0, 1, \dots \}^n$ and suppose that $\exists i \in \{0, 1, \dots \}^n$ such that $ \alpha_k = \beta_k$ for all $k < i$ and $\alpha_i = \beta_i$. Then $\alpha_k + \gamma_k = \beta_k + \gamma_k$ for all $k < i$ and $\alpha_i + \gamma_i < \beta_i + \gamma_i$, thus $\textbf{x}^\alpha \cdot \textbf{x}^\gamma \prec \textbf{x}^\beta \cdot \textbf{x}^\gamma$
\end{es}

\begin{defn}
    For every polynomial $f = \sum_{\alpha}  c_\alpha\textbf{x}^\alpha$ in $\mathbb{Z}[x_1, \dots, x_n]$ let $in_{\prec}(f)$ the $\prec$-largest monomial in f with $c_\alpha \neq 0$. We call $f$ $\prec$-monic if the coefficient of $in_{\prec}(f)$ is $1$ or $-1$.
\end{defn}

\begin{defn}
    Consider an ideal $I \subset \mathbb{Z}[x_1, \dots, x_n]$ with a fixed monomial order $\prec$. We say that a subset $\{g_1, \dots, g_t \} \subset I$ is a monic Grobner basis for $I$ with respect to $\prec$ if:
    \begin{enumerate}
        \item each $g_i$ is $\prec$-monic;
        \item  every $f \in I$ has $in_{\prec}(f)$ divisible by at least one of the initial monomials \newline $\{ in_{\prec}(g_1), \dots , in_{\prec}(g_n)\}$.
    \end{enumerate}

    A monomial $\textbf{x}^\alpha$ is called standard monomial if is not divisible by any of the initial monomials $\{ in_{\prec}(g_1), \dots , in_{\prec}(g_n)\}$.
\end{defn}

The proof of the following proposition can be found in \cite{noauthororeditor}.
\begin{prop}
    Given $I \subset \mathbb{Z}[x_1, \dots, x_n]$ and a Grobner basis $\{g_1, \dots, g_t \}$ for $I$ with respect to a monomial order $\prec$, then:
    \begin{enumerate}
        \item $I = (g_1, \dots , g_n)$, i.e. $\{ g_1, \dots , g_n \}$ generate $I$ as an ideal;
        
        \item the quotient ring $\faktor{\mathbb{Z}[x_1, \dots, x_n]}{I}$ is a free $\mathbb{Z}$-module, with a $\mathbb{Z}$-basis given by the set of standard monomials $\{ \textbf{x}^\alpha\}$.
    \end{enumerate}
\end{prop}

\begin{defn}
    Define on the ring $S$ the monomial order based on the linear order of the variable given by $X_F > X_{F'}$ if $F < F'$ called FY-monomial order.
\end{defn}

We now want to define a Grobner basis for the ring $\mathbb{Z}[X_F]_{F \in \mathcal{G}}$ and use it to find a monomial basis.
The next theorem and corollary are proved in \cite{Feichtner_2004}.

\begin{thm} \label{thm:GrobB}
    Let $\mathcal{L}$ be a finite atomic lattice, and $\mathcal{G} \subset \mathcal{L}\setminus\{\hat{0}\}$ a building set. Then for any choice of an FY-monomial order on the ring $S$, the ideal $I+J$ presenting $\mathcal{A}(\mathcal{L}, \mathcal{G}) = S/I+J$ has a monic Grobner basis consisting of the following elements:
    \begin{enumerate}
        \item all $x_{F_1} x_{F_2} \cdots x_{F_n}$ for all $\{ x_{F_1}, \dots, x_{F_n} \} \subset \mathcal{G}$ which are not $\mathcal{G}$-nested;
        
        \item for each $\mathcal{G}$-nested set $N$ with maximal elements $\{ F_1, \dots , F_l \}$, and for each $F \in \mathcal{G}$ with $F > \vee N$ the product:
    \[ x_{F_1}x_{F_2} \cdots x_{F_l} \cdot \left( \sum_{F' \ge F} x_{F'} \right)^{d(\vee N, F)} \]
    with his initial $\prec$-term which is:
    \[ x_{F_1} \cdots x_{F_l} x_F^{d(\vee N, F)}.\]
    \end{enumerate}

\end{thm}

Now we are ready to define a monomial basis for the Chow ring.

\begin{cor} \label{cor:monbase}
    Let $\mathcal{L}$ be a finite atomic lattice, and $\mathcal{G} \subset \mathcal{L}\setminus\{\hat{0}\}$ a building set. Then the Chow ring $\mathcal{A}(\mathcal{L}, \mathcal{G})$ is free as a $\mathbb{Z}$-module, with the $\prec$-standard monomial $\mathbb{Z}$-basis given by the set of FY-monomials:
    \begin{equation} \label{monbase}
    FY := \{ x_{F_1}^{m_1} \cdots x_{F_l}^{m_l} \, : \, N = \{F_1, \dots, F_l \} \text{ is }\mathcal{G} \text{-nested, and } 0 < m_i < m_{N}(F_i) \}.
    \end{equation}
\end{cor}

Observe now by quotienting $S$ by $I + J$ we are sending to zero the monomials generated by non $\mathcal{G}$-nested sets.
Thus the elements in the monomial basis \ref{monbase} are all generated by $\mathcal{G}$-nested sets, and they are exactly the monomial that are non divisible by the initial term of the elements of the Grobner basis described in Theorem \ref{thm:GrobB}.

\begin{notat*} 
    We denote with $FY^k$ the monomial in $FY$ with degree k.
\end{notat*}

\begin{cor} \label{cor:ZGmod}
    Let $\mathcal{L}$ be a finite atomic lattice, and $\mathcal{G} \subset \mathcal{L}\setminus\{\hat{0}\}$ a building set, if $G \subset Aut(\mathcal{L})$ is a subgroup setwise stabilizing $\mathcal{G}$, then $G$ permutes the set $FY$, as well as its subsets $FY^k$.
    
    Moreover, the $\mathbb{Z}G$-module on the Chow ring $\mathcal{A}(\mathcal{L}, \mathcal{G})$ and each of its components $A^k$ is a $G$-permutation representation with basis $FY$ and $FY^k$ respectively.
\end{cor}

\begin{proof}
    We already proved that if $g \in G$, and $N \subset \mathcal{G}$ is a $\mathcal{G}$-nested set then $g(N)$ is a $\mathcal{G}$-nested set, and since $g \in G$ are lattice automoprhisms we have that $g(\vee N) = \vee g(N)$.
    Since we previously proved that if $a \in \mathcal{L}$ is an atom then $g(a)$ is also an atom, we can observe that $d(F, F') = d(g(F), g(F'))$ in fact: 
    \begin{equation*}
        \begin{aligned}
        d(F, F') &= \min \{ d \in \mathbb{N} \, : \, F' =  F \vee a_1 \vee \dots \vee a_d \text{ where } a_1, \dots, a_d \text{ are atoms}\} \\
        &=  \min \{ d \in \mathbb{N} \, : \, g(F') = g(F \vee a_1 \vee \dots \vee a_d) \text{ where } a_1, \dots, a_d \text{ are atoms} \} \\
        &=  \min \{ d \in \mathbb{N} \, : \, g(F') = g(F) \vee g(a_1) \vee \dots \vee g(a_d)) \text{ where } g(a_1), \dots, g(a_d) \text{ are atoms} \} \\
        &=  d(g(F), g(F')).
        \end{aligned}
    \end{equation*}
    So we proved that $m_N(F) = m_{g(N)}(g(F))$, and thus by definition of $FY$-monomial we have that $g$ sends $FY^k$-monomials $x_{F_1}^{m_1} \cdots x_{F_l}^{m_l}$ to other $FY^k$-monomials $x_{g(F_1)}^{m_1} \cdots x_{g(F_l)}^{m_l}$.
    
    The second statement follow from Corollary \ref{cor:monbase} and $g$ preserve the degree.
\end{proof}

With those result we can give some degree bounds on $\mathcal{A}(\mathcal{L}, \mathcal{G})$.

\begin{cor} \label{cor:degree}
    Let $\mathcal{L}$ be a geometric lattice of rank $r + 1$, and $\mathcal{G} \subset \mathcal{L} \setminus \{ \hat{0} \}$ a building set. Then $\mathcal{A}(\mathcal{L}, \mathcal{G})$ vanishes in degrees strictly above $r$, that is: \[ \mathcal{A}(\mathcal{L},\mathcal{G}) = \bigoplus_{k = 0}^{r} A^k.\]
    
    In particular:
    \begin{enumerate}
        \item $A^r = 0$ unless $\mathcal{G}$ contains $\hat{1}$, in this latter case $A^r$ has  a $\mathbb{Z}$-basis $\{ x_{\hat{1}}^r \}$, and hence one has a $\mathbb{Z}$-module isomorphism:
        \[ \deg : A^r \longrightarrow \mathbb{Z} \text{ sending } x_{\hat{1}}^r \longmapsto 1;\]

        \item if $\hat{1} \notin \mathcal{G}$, defined $\max \mathcal{G}_{\le \hat{1}} = \{F_1, \dots, F_l\}$, then 
        $A^k = 0$, for all $k > r-l+1$,
        and one has a $\mathbb{Z}$-module isomorphism
        \[\deg : A^{r-l+1} \longrightarrow \mathbb{Z} \text{ sending } x_{F_1}^{m_1} \cdots x_{F_l}^{m_l} \longmapsto 1.\]
    \end{enumerate}
    
\end{cor}


\begin{proof}
Consider an $FY$-monomial $x_{F_1}^{m_1}x_{F_2}^{m_2} \cdots x_{F_l}^{m_l}$ with $N = \{F_1, \dots, F_l\}$ a $\mathcal{G}$-nested set, and compute his total degree:
    \begin{equation} \label{eq:maxdegree}
            \sum_{i = 1}^l m_i \le \sum_{i = 1}^l (m_N(F_i) - 1) = rk(\vee N) - l \le (r+1) - 1 = r    
    \end{equation}
where in the second inequality we used Lemma \ref{lem:usefulineq}.
    \begin{enumerate}
        \item To prove 1 is sufficient to note that in \ref{eq:maxdegree} the equality holds when $l = 1$ and $rk(\vee N) = r+1$, but $r+1 = rk (\mathcal{L})$, so $N = \{ \hat{1} \}$.
        This implies that the $FY$-monomial must be $x_{\hat{1}}^r$.
        
        \item Following \ref{eq:maxdegree} we have
        \[ \sum_{i = 1}^l m_i \le \sum_{i = 1}^l (m_N(F_i) - 1) = rk(\vee N) - l \le (r+1) - l = r - l +1, \]
        so the maximum degree achieved by an element of $\mathcal{A}(\mathcal{L}, \mathcal{G})$ is $r - l +1$.
            In this case the equality holds only if $rk(\vee N) = r+1$, so when $\vee N = \hat{1}$. Thus $N=\{ F_1, \dots, F_l \}$ and the basis for $A^{r-l+1}$ is $ \{x_{F_1}^{m_N(F_1)-1}x_{F_2}^{m_N(F_2)-1} \cdots x_{F_l}^{m_N(F_l)-1} \}$. \qedhere
        \end{enumerate}
\end{proof}

We conclude the discussion with and example.

\begin{es}
    Consider the lattice $\mathcal{L} = 2^{\{ 1, 2, \dots n \}}$ of rank $n$, along with these two building sets:
    \[ \mathcal{G}_{\min} = \{ \{1\}, \{2\}, \dots, \{n\} \}, \]
    \[ \mathcal{G} = \mathcal{G}_{\min} \cup \{ \hat{1} \} = \{ \{1\}, \{2\}, \dots, \{n\}, \{1, 2, \dots, n\} \}. \]
    then their Chow rings have the following descriptions:
    \[ \mathcal{A}(\mathcal{L}, \mathcal{G}_{\min})) = \mathbb{Z}[x_1, \dots, x_n] / (x_1, \dots, x_n) = \mathbb{Z}\]
    and
    \begin{equation*}    
        \begin{aligned}
        \mathcal{A}(\mathcal{L}, \mathcal{G}) &= \mathbb{Z}[x_1, \dots, x_n, x_{\hat{1}}]/(x_1 x_2 \cdots x_n, x_1 + x_{\hat{1}}, x_2 + x_{\hat{1}}, \dots, x_n + x_{\hat{1}}) \\
        &\simeq \mathbb{Z}[x_{\hat{1}}]/(x_{\hat{1}}^n) = span_{\mathbb{Z}}\{1, x_{\hat{1}}, x_{\hat{1}}^2, \dots, x_{\hat{1}}^r, \}
        \end{aligned}
    \end{equation*}
    where $rk(\mathcal{L}) = n = r +1$, so $r = n -1$.
\end{es}

\section{Properties of the Chow Ring}
We expose here some properties of the Chow ring. In particular we discuss here the Kahler package that is resumed in the next theorem.

The proof of the theorem can be found in \cite{Pagaria_2023}.

\begin{thm} \label{thm:kahler}
    For a simple matroid $\mathcal{M}$ with lattice of flats $\mathcal{L}=\mathcal{L}_\mathcal{M}$ of rank $r+1$, and any choice of building sets $\mathcal{G}\subset \mathcal{L}\setminus \{ \hat{0} \}$ that contains the ground set $\hat{1}=E$, the Chow ring 
    \[ \mathcal{A}(\mathcal{L}, \mathcal{G}) = \bigoplus_{k = 0}^r A^k  \]
    satisfy the following:
    \begin{enumerate}
        \item (Poincarè duality) For every $k\le \frac{r}{2}$, one has a perfect $\mathbb{Z}$-bilinear pairing:
        \[ A^k \times A^{r-k} \longrightarrow \mathbb{Z}\]
        \[ (a,b) \longmapsto \deg(a \cdot b)\]
        that is, $b \longmapsto \varphi_b(-)$ defined by $\varphi_b(a) = \deg(a \cdot b)$ is a $\mathbb{Z}$-linear isomorphism
        \[ A^{r-k} \simeq \text{Hom}_{\mathbb{Z}}(A^k, \mathbb{Z})\]

        \item (Hard Lefschetz) Tensoring over $\mathbb{Z}$ with $\mathbb{R}$, the real Chow ring \[ \mathcal{A}_{\mathbb{R}}(\mathcal{L}, \mathcal{G}) = \bigoplus_{k = 0}^r A^k_{\mathbb{R}} \]
        contains a non-empty convex cone $\mathcal{K} \subset A^1_{\mathbb{R}}$ of Lefschetz elements such that $ \forall \omega \in \mathcal{K}$ the map $a \longmapsto a \cdot \omega^{r- 2k}$ is an $\mathbb{R}$-linear isomorphism:
        \[ A^k_{\mathbb{R}} \longrightarrow A^{r-k}_{\mathbb{R}}, \text{ for $k \le \frac{r}{2}$}.\]
        In particular the multiplication by $\omega$ is an injection:
        \[ A^k_{\mathbb{R}} \hookrightarrow A^{r-k}_{\mathbb{R}}, \text{ for $k < \frac{r}{2}$}.\]

        \item (Hodge-Riemann-Minkowski inequality) Each Lefschetz element $\omega$ define a quadratic form 
        \[ a \longmapsto (-1)^k \deg(a \cdot \omega^{r-2k} \cdot a)\]
        on $A^k_{\mathbb{R}}$ that become positive definite upon restriction to the kernel of the map \[A^k_{\mathbb{R}} \longrightarrow A^{r-k+1}_{\mathbb{R}},\] that sends $a \longmapsto a \cdot \omega^{r-2k+1}$.
    \end{enumerate}
\end{thm}


Consider now a subgroup $G \subset \text{Aut}(\mathcal{L} \setminus \hat{0})$ setwise stabilizing $\mathcal{G}$ i.e.
\[ \forall g \in G, \; x \in \mathcal{G} \implies g(x) \in \mathcal{G}\]

and hence, as we already noted previously, it acts on $\mathcal{A}(\mathcal{L}, \mathcal{G})$ and on each $A^k$.
We now prove two corollaries of Theorem \ref{thm:kahler}.

\begin{cor}
    Consider $\mathcal{L}$ a lattice of rank $r + 1$, and consider $\mathcal{G} \subset \mathcal{L}$ a building set containing $E := \hat{1}$, then one has an isomorphism of $\mathbb{
    Z}G$-modules \[ A^{r-k} \longrightarrow A^k\]
    for each $k \le \frac{r}{2}.$
\end{cor}

\begin{proof}
    By Corollary \ref{cor:degree}, a basis for $A^r$ as $\mathbb{Z}$-module is $\{ x_E^r \}$.
    Let $g \in G$ setwise stabilizing $\mathcal{G}$, then the element $x_E^r$ is fixed by $g$.
    Thus considering on $\mathbb{Z}$ the trivial $G$-action, i.e. $g \cdot n= n$ for all $g \in G$ and $n \in \mathbb{Z}$, we have that:
    \[ \deg(g \cdot x_E^r) = \deg(x_{g(E)}^r) = \deg(x_E^r) = g \cdot \deg(x_E^r).\]
    Thus the map $\deg : A^r \longrightarrow \mathbb{Z}$ is a $G$-equivariant map.
    Consider now the Poincarè duality isomorphism $A^{r-k} \longrightarrow Hom_{\mathbb{Z}}(A^k, \mathbb{Z})$, sending $b \longmapsto \varphi_b(-)$, with $\varphi_b(a) = \deg(a \cdot b)$, then this is also $G$-equivariant, in fact
    \[ \varphi_{g \cdot b}(a) = \deg(g\cdot b a) = g \cdot \deg(ba) = g \cdot \varphi_b(a).\]

    It remains to exhibit a $G$-equivariant isomorphism $Hom(A^k, \mathbb{Z}) \longrightarrow A^k$. Using Corollary \ref{cor:ZGmod} we can choose a $\mathbb{Z}$-basis $\{ e_i \}$ permuted by $G$, so that every element $g \in G$ acts on this basis as a permutation matrix $P(g)$.
    Let $\{ f_i \}$ be the dual basis of $\{ e_i \}$ for $Hom(A^k, \mathbb{Z})$, hence $g \in G$ acts via the matrix $^tP(g^{-1})$ on $Hom(A^k, \mathbb{{Z}})$.
    
    But since $P(g)$ is permutation matrix one has \[P(g^{-1})^T = (P(g)^{-1})^T = P(g),\] so in the end $G$ acts on $Hom(A^k, \mathbb{{Z}})$ via permutation matrix $P(g)$.
    Hence the map \[ \rho:Hom(A^k, \mathbb{{Z}}) \longrightarrow A^k \] defined as $\rho(e_i) = f_i$ is a $G$-equivariant isomorphism. In fact
    \[ \rho(g\cdot e_i) = \rho(P(g)e_i) = P(g)f_i = P(g) \rho(e_i) = g \cdot \rho(e_i). \qedhere\]
\end{proof}

\begin{cor}
    Let $\mathcal{L} = \mathcal{L}_{\mathcal{M}}$ for a simple matroid $\mathcal{M}$, with $\mathcal{G}$ any $G$-stable building set containing $E := \hat{1}$, one has a $\mathbb{R}G$-modules injective maps $A^{k}_{\mathbb{R}} \longrightarrow A^{k+1}_\mathbb{R}$ for $k < \frac{r}{2}$.     
\end{cor}

\begin{proof}
    If exists a Lefschetz $G$-fixed element $\omega$ in $A^1_\mathbb{R}$, then the multiplication by $\omega$ si a $\mathbb{R}G$-module injections.
    Let $\mathcal{G}$ a $G$-stable building set, in \cite{Pagaria_2023} they provided a $G$-stable convex cone $\mathcal{K}$ of Lefschetz elements inside $A^1(\mathcal{L}_{\mathcal{M}}, \mathcal{G})$. If we choose $\xi \in \mathcal{K}$, then considering the $G$-average $\omega := \frac{1}{|G|}\sum_{g \in G} g(\xi) \in \mathcal{K}$. Then $\omega$ is a $G$-fixed Lefschetz element since for all $g \in G$
    \[g \omega = g \left( \frac{1}{|G|} \sum_{h \in G} h(\xi) \right) = \frac{1}{|G|} \sum_{h \in G} gh(\xi) = \frac{1}{|G|} \sum_{k \in G} k(\xi) = \omega. \qedhere\]
\end{proof}

The previous two results will be extended and specified by the main theorem.

\section{Geometric interpretation of the Chow Ring}
In this section we provide two different definitions of the De Concini-Procesi wonderful arrangement models. The first one describes the model as the closure of an open embeddings of the arrangement complement in a product of projective spaces. The second definition describes the model as the result of a sequence of blowups of $\mathbb{P}^n$ along certain subspaces.
A more complete treatment of these arguments can be found in \cite{feichtner2004conciniprocesiwonderfularrangementmodels}.

Consider a vector space $V$ of finite dimension on a field $\mathbb{K}$, we define the projective space on $V$ as the quotient:
\[ \mathbb{P}(V) := \frac{V \setminus \{ 0\}}{\sim} \]
where the relation $\sim$ is defined as:
\[ v \sim w \iff \exists \lambda \in \mathbb{K} \text{ such that } v = \lambda w.\]
If the vector space $V$ is on $\mathbb{R}$ the relation is defined with $\lambda \in \mathbb{R}$.
We also have a natural projection map $\pi:V  \setminus \{ 0 \}  \longrightarrow\mathbb{P}(V)$.

The space $\mathbb{P}(V)$ parametrizes 1-dimensional subspace of $V$.

Given $X$ a subset of $V$, we shall denote by $<X>$ the subspace of $V$ generated by the elements of $X$.

Consider now a subspace arrangement $\mathcal{A} := \{ H_i \, : \, i \in I \}$ in $V$ and define the complement of $\mathcal{A}$ as:
\[ \mathcal{M}(\mathcal{A}) := \bigcup_{i \in I} V \setminus H_i.\]
For each $H_i \in \mathcal{A}$ there exists two natural projection maps:
\[ p : V \longrightarrow \faktor{V}{H_i} \text{ and } \pi:\faktor{V}{H_i} \setminus \{ 0 \} \longrightarrow \mathbb{P} \left( \faktor{V}{H_i} \right).\]
Thus composition $\pi \circ p$ is a surjective map:
\[ \pi_{H_i} : V \setminus H_i \longrightarrow \faktor{V}{H_i} \setminus \{ 0 \} \longrightarrow \mathbb{P} \left( \faktor{V}{H_i} \right)\]
defined outside $H_i$, and thus we can define a regular morphism 
\[ \mathcal{M}(\mathcal{A}) \longrightarrow \prod_{i \in I} \mathbb{P} \left( \faktor{V}{H_i} \right) .\]
The graph of this morphism is a closed subset of $\mathcal{M}(\mathcal{A}) \times \prod_{i \in I} \mathbb{P} \left( \faktor{V}{H_i} \right)$ with embeds as open subset in $V \times \prod_{i \in I} \mathbb{P}\left( \faktor{V}{H_i}\right)$.
Finally we have an embedding
\[ \rho \, : \, \mathcal{M}(\mathcal{A}) \longrightarrow V \times \prod_{i \in I} \mathbb{P} \left( \faktor{V}{H_i} \right).\]
Explicitly the map $\rho$ act like this: 
\[\rho(x) = \left(x, \, \left(\frac{<x, H_i>}{H_i}\right)_{i \in I} \right),\]
and we use it to define the De Concini-Procesi model:

\begin{defn} \label{DP:proj}
We define the De Concini-Procesi wonderful model for $\mathcal{A}$ as the closure of $\rho(\mathcal{M}(\mathcal{A}))$ and we denote it by $Y_\mathcal{A}$.    
\end{defn}

Another possible definition of the De Concini-Procesi wonderful model is given by successive blow-ups of subspaces.

\begin{comment}
\begin{defn}
    Let $\mathcal{L}$ a lattice and $\le$ $\le^*$ two order relations on $\mathcal{L}$. $\le^*$ is a linear extension of $(\mathcal{L}, \le)$ when
    \begin{enumerate}
        \item $\le^*$ is a total order;
        \item given $x,y \in \mathcal{L}$ 
        \[ x \le y \implies x \le^* y.\]
    \end{enumerate}
\end{defn}
\end{comment}

We give here some notions about blowups in $\mathbb{C}^n$.

\begin{defn}
    Let $n \in \mathbb{N}$ and consider the space $\mathbb{C}^n$ with coordinates $z_1, \dots, z_n$ and the space $\mathbb{P}^{n-1}$ with homogeneous coordinates $[x_1: \dotsc :x_n]$. We define the blowup of $\mathbb{C}^n$ at $0$ as the subset of $\mathbb{C}^n \times \mathbb{P}^{n-1}$
    \[ Bl_{\{ 0 \}} \mathbb{C}^n := \{ (z,x) \in \mathbb{C}^n \times \mathbb{P}^{n-1} \, : \, z_i x_j = z_j x_i, \, i,j=1, \dots, n \},\]
    with a projection map
    \[ \sigma : Bl_{\{0\}} \mathbb{C}^n \longrightarrow \mathbb{C}^n\]
    called blow up map, given by $\sigma(z,x) = z$.
\end{defn}

We are essentially substituting every point in $\mathbb{C}^n$ with a copy of $\mathbb{P}^{n-1}$. We define the exceptional divisor to be the preimage of $\{ 0 \}$ via $\sigma$
\[ E := \sigma^{-1}(\{0 \}).\]
Let now $C \subset \mathbb{C}^n$ be a subspace, then we call 
\[ C' := \overline{\sigma^{-1}( C \setminus \{ 0\})} \]
the proper transform of $C$, and
\[ \sigma^{-1}(C)\]
the total transform of $C$.

We can generalize the above definition in he case of linear subspaces.
\begin{defn}
    Let $\mathbb{C}^m \subset \mathbb{C}^n$ be the linear subspace satisfying the conditions 
    \[ z_{m+1} = 0, \dots, z_n =0,\]
    and denote with $[x_{m+1}: \dots :x_n]$ the homogeneous coordinates of $\mathbb{P}^{n-m-1}$. We define the blow up of $\mathbb{C}^n$ in $\mathbb{C}^m$ as the subset of $\mathbb{C}^n \times \mathbb{P}^{n-m-1}$
    \[ Bl_{\mathbb{C}^m} \mathbb{C}^n := \{ (z,x) \in \mathbb{C}^n \times \mathbb{P}^{n-m-1} \, : \, z_j x_i = z_i x_j, \, i,j = m+1, \dots, n \}\]
    with a blow up map 
    \[ \sigma :Bl_{\mathbb{C}^m} \mathbb{C}^n \longrightarrow \mathbb{C}^n\]
    defined as $\sigma(z,x) = z.$
\end{defn}

In this case the exceptional divisor is defined as 
\[ E := \sigma^{-1}(\mathbb{C}^m)\]
and given a subspace $V \subset \mathbb{C}^n$ properly containing $\mathbb{C}^m$ then we define 
\[ \overline{V} := \overline{\sigma^{-1}(V \setminus \mathbb{C}^m)} \]
the proper transform of $V$, and
\[ \sigma^{-1}(V) \]
the total transform of $V$.

It is possible also to generalize these definitions to manifolds in $\mathbb{C}^n$.

We can now define the blow up version of the De Concini-Procesi wonderful model.

\begin{defn} \label{DP:blowup}
    Let $\mathcal{A}$ be an arrangements of real or complex linear subspaces in $V$. Let $X_1, \dots, X_t$ be a linear extension of the opposite order $\mathcal{L}(\mathcal{A})^{op}_{> \hat{0}}$ on $\mathcal{L}(\mathcal{A})_{> \hat{0}}$. The De Concini - Procesi wonderful model $Y_{\mathcal{A}}$ for $\mathcal{A}$ is a result of successively blowing up subspaces $X_1, \dots, X_t$ and respectively their proper transforms.
\end{defn} 

\begin{es}
    Consider the braid arrangements model $\mathcal{A}_2$ in 
    $V = \faktor{\mathbb{R}^3}{\Delta} \simeq \mathbb{R}^2$ where $\Delta$ is the diagonal in $\mathbb{R}^3$.
    Choosing in $\mathbb{R}^3$ the coordinates $(X_1, X_2, X_3)$, the arrangements $\mathcal{A}_2$ is 
    \begin{center} $ \left\{\begin{matrix}
    X_1 - X_2 = 0 \\ X_2 - X_3 = 0 \\ X_1 - X_3 = 0.  
    \end{matrix}\right.$ \end{center}
    Considering in $V$ the basis 
    \begin{center} $ \left\{\begin{matrix}
    x = X_1 - X_2 \\y = X_2 - X_3 
    \end{matrix}\right.$ \end{center}
    we obtain \[x + y = X_1 - X_2 + X_2 -  X_3 = X_1 - X_3,\] so in $V$ the arrangement is 
    \begin{center} $ \left\{\begin{matrix*}[l]
    x = 0 \\ 
    y = 0 \\ 
    x+y = 0.
    \end{matrix*}\right.$ \end{center}
    Following the construction of Definition \ref{DP:blowup}, the De Concini-Procesi wonderful model $Y_{\mathcal{A}_2}$ is the blowup of $V$ at $\{0\}$. 
    The blow up map is
    \[ \sigma: Bl_{\{0\}}V = Y_{\mathcal{A}_2} \longrightarrow V,\]
    and the result is an open Möbius band. The exceptional divisor, $D_{123} = \sigma^{-1}(\{0\}) \simeq \mathbb{RP}^1$, intersects transversally with the proper transforms $D_{ij}$ of the hyperplanes $H_{ij}$ for $1 \le j < j \le 3$.
\end{es}

\begin{defn}
    Let $\mathcal{L}$ a lattice, define the order complex $\Delta(\mathcal{L})$ of $\mathcal{L}$ as the complex whose elements are the chains of $\mathcal{L}$.    
\end{defn}

Now we enunciate some properties of the model. 

\begin{thm}
    \begin{enumerate}
        \item The arrangement model $Y_{\mathcal{A}}$ is a smooth variety with a natural projection map to the original ambient space
        \[ \pi:Y_{\mathcal {A}} \longrightarrow V,\]
        which is one-to-one on the arrangement complement $\mathcal{M}(\mathcal{A})$.

        \item The complement of $\pi^{-1}(\mathcal{M}(\mathcal{A}))$ in $Y_{\mathcal{A}}$ is a divisor with normal crossings; the irreducible components are the proper transforms $D_X$ of intersections $X$ in $\mathcal{L}(\mathcal{A})$,
        \[ Y_{\mathcal{A}} \setminus \pi^{-1}(\mathcal{M}(\mathcal{A})) = \bigcup_{X \in \mathcal{L}(\mathcal{A}) \setminus \{\hat{0}\}} D_X.\]

        \item Irreducible components $D_X$ for $X \in \mathcal{A} \subseteq 
        \mathcal{L}(\mathcal{A})_{\ge \hat{0}}$ intersects if and only if $\mathcal{S}$ is a linearly ordered subset in  $\mathcal{L}(\mathcal{L}(\mathcal{A}) \setminus \{\hat{0}\}$. If we think about $Y_{\mathcal{A}}$ as stratified by the irreducible components of the normal crossing divisor and their intersections, then the poset of strata coincides with the face poset of the order complex $\Delta(\mathcal{L}(\mathcal{A}) \setminus \{\hat{0}\})$.
    \end{enumerate}
\end{thm}

We define now the face poset of a set complex.

\begin{defn}
    Let $\mathcal{S}$ be a set, and $\Delta \subset \mathcal{P}(\mathcal{S})$ a set complex. An element $X \in \Delta$ si called a face, and the set $\mathcal{F}(\Delta)$ is the poset of faces of $\Delta$ ordered by inclusion.
\end{defn}


In definition \ref{DP:proj} and \ref{DP:blowup} we defined two model construction on the whole lattice $\mathcal{L}(\mathcal{A})$, but we can restrict the definition to a building set.
Consider for instance a building set $\mathcal{G} \subseteq \mathcal{L}(\mathcal{A})$ where  $\mathcal{A} := \{ H_i \, : \, i \in I\}$ is a linear subspace arrangements in a real or complex vector space $V$. Consider the map \[ \rho : \mathcal{M}(\mathcal{A}) \longrightarrow V \times \prod_{H \in \mathcal{G}} \mathbb{P}\left( \faktor{V}{H} \right)\]
as defined before and define $Y_{\mathcal{A},\mathcal{G}}$ as the closure of the image of $\rho$.

The blow up version of the model si obtained substituting a linear extension of $\mathcal{L}_{> \hat{0}}$ with a non increasing order on $\mathcal{G}$, and successivlei blowing up subspaces in $\mathcal{G}$.

The model $Y_{\mathcal{A}, \mathcal{G}}$ has the following properties

\begin{thm}
    \begin{enumerate}
        \item The arrangement model $Y_{\mathcal{A}, \mathcal{G}}$ is a smooth variety with a natural projection map to the original ambient space
        \[ \pi:Y_{\mathcal{A}, \mathcal{G}} \longrightarrow V,\]
        which is one-to-one on the arrangement complement $\mathcal{M}(\mathcal{A})$.

        \item The complement of $\pi^{-1}(\mathcal{M}(\mathcal{A}))$ in $Y_{\mathcal{A}, \mathcal{G}}$ is a divisor with normal crossings; the irreducible componnents are the proper transforms $D_X$ of intersections $X$ in $\mathcal{G}$,
        \[ Y_{\mathcal{A}, \mathcal{G}} \setminus \pi^{-1}(\mathcal{M}(\mathcal{A})) = \bigcup_{X \in \mathcal{G}} D_X.\]

        \item Irreducible components $D_X$ for $X \in \mathcal{S} \subseteq 
        \mathcal{G}$ intersects if and only if $\mathcal{S}$ is a linearly ordered subset in  $\mathcal{G}$. If we think about $Y_{\mathcal{A}, \mathcal{G}}$ as stratified by the irreducible components of the normal crossing divisor and their intersections, then the poset of strata coincides with the face poset of nested sets complex $\mathcal{F}(\mathcal{N}(\mathcal{L}, \mathcal{G}))$.
    \end{enumerate}
\end{thm}


We finish this chapter by giving a geometrical interpretations of the Chow ring $\mathcal{A}(\mathcal{L}, \mathcal{G})$.

Consider a linear essential hyperplanes arrangement $\mathcal{A}$ in a complex vector space $V$.
Considering $\mathcal{G} \subset \mathcal{L}(\mathcal{A})$ a building set, we already studied the Chow ring $\mathcal{A}(\mathcal{L}, \mathcal{G})$.

The goal now is to define a projective version of the De Concini-Procesi wonderful model in order to give a geometrical meaning to $\mathcal{A}(\mathcal{L}, \mathcal{G})$.
We start by defining the projectivization of an hyperplanes arrangement.

\begin{defn}
    Given an arrangement $\mathcal{A}$ as above, then define the projectivization of $\mathcal{A}$ as
    \[ \mathbb{P}\mathcal{A} := \{ \mathbb{P}(H) \, : \, H \in \mathcal{A}  \}\]
    and the complement
    \[ \mathcal{M}(\mathbb{P}\mathcal{A}) := \mathbb{P}(V) \setminus \bigcup \mathbb{P}\mathcal{A}. \]
    The lattice $\mathcal{L}(\mathbb{P}\mathcal{A})$ is defined as the lattice of intersections of $\mathbb{P}\mathcal{A}$.
\end{defn}

Then we define the projective De Concini-Procesi wonderful model.

\begin{defn}
    Let $\mathcal{A}$ be an arrangements of real or complex linear subspaces in $V$, and consider his projectivization $\mathbb{P}\mathcal{A}$ in $\mathbb{P}V$.
    
    Let $\mathbb{P}X_1, \dots, \mathbb{P}X_t$ be a linear extension of the opposite order $\mathcal{L}(\mathbb{P}\mathcal{A})^{op}_{> \hat{0}}$ on $\mathcal{L}(\mathbb{P}\mathcal{A})_{> \hat{0}}$. The projective De Concini-Procesi wonderful model $Y_{\mathcal{A}}^\mathbb{P}$ for $\mathbb{P}\mathcal{A}$ is a result of successively blowing up subspaces $\mathbb{P}X_1, \dots, \mathbb{P}X_t$, respectively their proper transforms.
\end{defn}

As we did before we can consider a building set $\mathcal{G} \subseteq \mathcal{L}(\mathbb{P}\mathcal{A})$, and define the projective version of the De Concini-Procesi model for building set. Denote this model with $Y_{\mathcal{A}, \mathcal{G}}^{\mathbb{P}}$

Finally the following theorem gives us the geometric interpretation of the Chow ring. The proof of the theorem can be found in \cite{DP2}.
\begin{thm}
    Let $\mathcal{L} = \mathcal{L}(\mathcal{A})$ the lattice of a linear essential hyperplanes arrangement $\mathcal{A}$, and $\mathcal{G}$ a building set which contains the total intersections of $\mathcal{A}$. Then, $\mathcal{A}(\mathcal{L}, \mathcal{G})$ is isomorphic to the integral cohomology algebra of the projective arrangement model $Y_{\mathcal{A},\mathcal{G}}^{\mathbb{P}}$
    \[ \mathcal{A}(\mathcal{L}, \mathcal{G}) \simeq H^*(Y_{\mathcal{A},\mathcal{G}}^{\mathbb{P}}, \mathbb{Z}).\]
\end{thm}

\begin{comment}
\section{Geometric interpretation of the Chow Ring}
In this section we provide two different definitions of the De Concini-Procesi wonderful arrangement models. The first one describes the model as the closure of an open embeddings of the arrangement complement in a product of projective spaces. The second definition describes the model as the result of a sequence of blowups of $\mathbb{P}^n$ along certain subspaces.
A more complete treatment of these arguments can be found in \cite{feichtner2004conciniprocesiwonderfularrangementmodels}.

Consider a vector space $V$ of finite dimension on a field $\mathbb{K}$, we define the projective space on $V$ as the quotient:
\[ \mathbb{P}(V) := \frac{V \setminus \{ 0\}}{\sim} \]
where the relation $\sim$ is defined as:
\[ v \sim w \iff \exists \lambda \in \mathbb{K} \text{ such that } v = \lambda w.\]
If the vector space $V$ is on $\mathbb{R}$ the relation is defined with $\lambda \in \mathbb{R}$.
We also have a natural projection map $\pi:V  \setminus \{ 0 \}  \longrightarrow\mathbb{P}(V)$.

The space $\mathbb{P}(V)$ parametrizes 1-dimensional subspace of $V$.

Given $X$ a subset of $V$, we shall denote by $<X>$ the subspace of $V$ generated by the elements of $X$.

Consider now a subspace arrangement $\mathcal{A} := \{ H_i \, : \, i \in I \}$ in $V$ and define the complement of $\mathcal{A}$ as:
\[ \mathcal{M}(\mathcal{A}) := \bigcup_{i \in I} V \setminus H_i.\]
For each $H_i \in \mathcal{A}$ there exists two natural projection maps:
\[ p : V \longrightarrow \faktor{V}{H_i} \text{ and } \pi:\faktor{V}{H_i} \setminus \{ 0 \} \longrightarrow \mathbb{P} \left( \faktor{V}{H_i} \right).\]
Thus composition $\pi \circ p$ is a surjective map:
\[ \pi_{H_i} : V \setminus H_i \longrightarrow \faktor{V}{H_i} \setminus \{ 0 \} \longrightarrow \mathbb{P} \left( \faktor{V}{H_i} \right)\]
defined outside $H_i$, and thus we can define a regular morphism 
\[ \mathcal{M}(\mathcal{A}) \longrightarrow \prod_{i \in I} \mathbb{P} \left( \faktor{V}{H_i} \right) .\]
The graph of this morphism is a closed subset of $\mathcal{M}(\mathcal{A}) \times \prod_{i \in I} \mathbb{P} \left( \faktor{V}{H_i} \right)$ with embeds as open subset in $V \times \prod_{i \in I} \mathbb{P}\left( \faktor{V}{H_i}\right)$.
Finally we have an embedding
\[ \rho \, : \, \mathcal{M}(\mathcal{A}) \longrightarrow V \times \prod_{i \in I} \mathbb{P} \left( \faktor{V}{H_i} \right).\]
Explicitly the map $\rho$ act like this: 
\[\rho(x) = \left(x, \, \left(\frac{<x, H_i>}{H_i}\right)_{i \in I} \right),\]
and we use it to define the De Concini-Procesi model:

\begin{defn}
We define the De Concini-Procesi wonderful model for $\mathcal{A}$ as the closure of $\rho(\mathcal{M}(\mathcal{A}))$ and we denote it by $\mathcal{Y_A}$.    
\end{defn}

Another possible definition of the De Concini-Procesi wonderful model is given by successive blow-ups of subspaces. We first define what is a linear extension if a lattice.

\begin{defn}
    Let $\mathcal{L}$ a lattice and $\le$ $\le^*$ two order relations on $\mathcal{L}$. $\le^*$ is a linear extension of $(\mathcal{L}, \le)$ when
    \begin{enumerate}
        \item $\le^*$ is a totale order;
        \item given $x,y \in \mathcal{L}$ 
        \[ x \le y \implies x \le^* y.\]
    \end{enumerate}
\end{defn}

\begin{defn} \label{DP:blowup}
    Let $\mathcal{A}$ be an arrangements of real or complex linear subspaces in $V$. Let $X_1, \dots, X_t$ be a linear extension of the opposite order $\mathcal{L}(\mathcal{A})^{op}_{> \hat{0}}$ on $\mathcal{L}(\mathcal{A})_{> \hat{0}}$. The De Concini - Procesi wonderful model $\mathcal{Y}_{\mathcal{A}}$ for $\mathcal{A}$ is a result of successively blowing up subspaces $X_1, \dots, X_t$, respectively their proper transforms.
\end{defn} 

\begin{es}
    Consider the braid arrangements model $\mathcal{A}_2$ in 
    $V = \faktor{\mathbb{R}^3}{\Delta} \simeq \mathbb{R}^2$ where $\Delta$ is the diagonal in $\mathbb{R}^3$.
    Choosing in $\mathbb{R}^3$ the coordinates $(X_1, X_2, X_3)$, the arrangements $\mathcal{A}_2$ is 
    \begin{center} $ \left\{\begin{matrix}
    X_1 - X_2 = 0 \\ X_2 - X_3 = 0 \\ X_1 - X_3 = 0  
    \end{matrix}\right.$ \end{center}
    and in $V$ the basis 
    \begin{center} $ \left\{\begin{matrix}
    x = X_1 - X_2 \\y = X_2 - X_3 
    \end{matrix}\right.$ \end{center}
    we obtain \[x + y = X_1 - X_2 + X_2 -  X_3 = X_1 - X_3,\] so in $V$ the arrangement is 
    \begin{center} $ \left\{\begin{matrix*}[l]
    x = 0 \\ 
    y = 0 \\ 
    x+y = 0.
    \end{matrix*}\right.$ \end{center}
    Following the construction of Definition \ref{DP:blowup}, the De Concini-Procesi wonderful model $\mathcal{Y}_{\mathcal{A}_2}$ is the blowup of $V$ at $\{0\}$. 
    The blow up map is
    \[ \sigma: Bl_{\{0\}}V = \mathcal{Y}_{\mathcal{A}_2} \longrightarrow V.\]
    and the result is an open Möbius band. The exceptional divisor is $\sigma^{-1}(\{0\}) \simeq \mathbb{RP}^1$.
\end{es}

\begin{defn}
    Let $\mathcal{L}$ a lattice, define the order complex $\Delta(\mathcal{L})$ of $\mathcal{L}$ as the complex whose elements are the chains of $\mathcal{L}$.    
\end{defn}

Now we enunciate some properties of the model. 

\begin{thm}
    \begin{enumerate}
        \item The arrangement model $\mathcal{Y}_{\mathcal{A}}$ is a smooth variety with a natural projection map to the original ambient space
        \[ \pi:\mathcal{Y}_{\mathcal {A}} \longrightarrow V,\]
        which is one-to-one on the arrangement complement $\mathcal{M}(\mathcal{A})$.

        \item The complement of $\pi^{-1}(\mathcal{M}(\mathcal{A}))$ in $\mathcal{Y}_{\mathcal{A}}$ is a divisor with normal crossings; the irreducible componnents are the proper transforms $D_X$ of intersections $X$ in $\mathcal{L}(\mathcal{A})$,
        \[ \mathcal{Y}_{\mathcal{A}} \setminus \pi^{-1}(\mathcal{M}(\mathcal{A})) = \bigcup_{X \in \mathcal{L}(\mathcal{A})_{\ge \hat{0}}} D_X.\]

        \item Irreducible components $D_X$ for $X \in \mathcal{A} \subseteq 
        \mathcal{L}(\mathcal{A})_{\ge \hat{0}}$ intersects if and only if $\mathcal{S}$ is a linearly ordered subset in  $\mathcal{L}(\mathcal{A})_{\ge \hat{0}}$. If we think about $\mathcal{Y}_{\mathcal{A}}$ as stratified by the irreducible components of the normal crossing divisor and their intersections, then the poset of strata coincides with the face poset of the order complex $\Delta(\mathcal{L}_{\ge \hat{0}})$.
    \end{enumerate}
\end{thm}

Now consider a linear essential hyperplanes arrangement $\mathcal{A}$ in a complex vector space $V$ then $\mathcal{L}(A)$ is an atomic lattice.
Considering $\mathcal{G} \subset \mathcal{L}(\mathcal{A})$ a building set, it is well defined the Chow ring $\mathcal{A}(\mathcal{L}, \mathcal{G})$.

To state the geometrical interpretations of the Chow ring of an arrangements of hyperplane we need to define the projective version of the De Concini-Procesi wonderful model.

\begin{defn}
    Given an arrangement $\mathcal{A}$ as above, then define the projectivization of $\mathcal{A}$ as
    \[ \mathbb{P}\mathcal{A} := \{ \mathbb{P}(H) \, : \, H \in \mathcal{A}  \}\]
    and the complement
    \[ \mathcal{M}(\mathbb{P}\mathcal{A}) := \mathbb{P}(V) \setminus \bigcup \mathbb{P}\mathcal{A}. \]
    The lattice $\mathcal{L}(\mathbb{P}\mathcal{A})$ is defined as the lattice of intersections of $\mathbb{P}\mathcal{A}$.
\end{defn}

\begin{defn}
    Let $\mathcal{A}$ be an arrangements of real or complex linear subspaces in $V$, and consider his projectivization $\mathbb{P}\mathcal{A}$ in $\mathbb{P}V$.
    
    Let $\mathbb{P}X_1, \dots, \mathbb{P}X_t$ be a linear extension of the opposite order $\mathcal{L}(\mathbb{P}\mathcal{A})^{op}_{> \hat{0}}$ on $\mathcal{L}(\mathbb{P}\mathcal{A})_{> \hat{0}}$. The projective De Concini-Procesi wonderful model $\mathcal{Y}_{\mathcal{A}}^\mathbb{P}$ for $\mathbb{P}\mathcal{A}$ is a result of successively blowing up subspaces $\mathbb{P}X_1, \dots, \mathbb{P}X_t$, respectively their proper transforms.
\end{defn}

The same construction can be defined for a building set $\mathcal{G}$ of $\mathcal{L}(\mathcal{A})$ obtaining $\mathcal{Y}_{\mathcal{A},\mathcal{G}}^{\mathbb{P}}$·
The proof of the theorem can be found in \cite{DP2}.
\begin{thm}
    Let $\mathcal{L} = \mathcal{L}(\mathcal{A})$ the lattice of a linear essential hyperplanes arrangement $\mathcal{A}$, and $\mathcal{G}$ a building set which contains the total intersections of $\mathcal{A}$. Then, $\mathcal{A}(\mathcal{L}, \mathcal{G})$ is isomophic to the integral cohomology algebra of the projective arrangement model $\mathcal{Y}_{\mathcal{A},\mathcal{G}}^{\mathbb{P}}$:
    \[ \mathcal{A}(\mathcal{L}, \mathcal{G}) \simeq H^*(\mathcal{Y}_{\mathcal{A},\mathcal{G}}^{\mathbb{P}}, \mathbb{Z}).\]
\end{thm}

\end{comment}




\chapter{Symmetric Chains and equivariant relations}

In this chapter we discuss the main theorem of this work.
In the first section we introduce symmetric chain decompositions for a poset. 
Then we prove a theorem about the existence of injections and isomorphisms between certain grades of the Feichtner-Yuzviski basis. We finish by discussing the hypothesis of the theorem and by giving a conuterexample where the theorem fails.

\section{Symmetric Chain Decomposition}
In this section we introduce symmetric chain decomposition which is a way to decompose a poset in a disjoint union of symmetric total ordered set.

\begin{defn}
    Let $(\mathcal{P}, \le)$ a ranked poset with rank function $rk : \mathcal{P} \longrightarrow \mathbb{N}$, we say that the elements $x_1, \dots, x_h \in \mathcal{P}$ form a symmetric chain if :
    \begin{enumerate}
        \item $x_{i+1} \gtrdot x_{i}$ for all $i < h$;
        \item $rk(x_1) + rk(x_h) = rk(\mathcal{P}).$
    \end{enumerate}
    where $rk(\mathcal{P})$ is the largest rank in $\mathcal{P}$.
\end{defn}

\begin{rmk}
    Suppose that $rk(\mathcal{P}) = k$, then consider $x_1, \dots, x_h \in \mathcal{P}$ a symmetric chain in $\mathcal{P}$. Then, by condition (2) of the above definition we have that $rk(x_1) = k - rk(x_h)$, so the length of the chain above the middle rank of the poset $\mathcal{P}$ is equal to the length of the chain under the middle rank of the poset $\mathcal{P}$.
\end{rmk}

We give here some examples.

\begin{es} \label{ex:chains}
    \begin{enumerate}
        \item Let $X$ be a finite set, consider the power set $\mathcal{P}(X)$ with rank function $rk : \mathcal{P}(X) \longrightarrow \mathbb{N}$ defined by $rk(A) = |A|$. Then a collection $A_1, \dots, A_h \in \mathcal{P}(X)$ form a symmetric chain if:
        \begin{enumerate}
            \item $A_i \subset A_{i+1}$ and $|A_{i+1}| = 1 + |A_i|$ for $i<h$;
            \item $|A_1| + |A_h| = |X|$.
        \end{enumerate}
        Consider for example the set $X = \{a,b,c,d\}$, in Figure \ref{fig:symmchain1} is depicted a symmetric chain decomposition of $\mathcal{P}(X)$.

        \begin{figure}[h!]
        \begin{center} \begin{tikzpicture}
            \node (top) at (0,0) {$\{ a, b, c, d \}$};
            \node (n1) at (-4.4, -2)  {$\{a, b, c \}$};
            \node (n2) at (-1.8, -2)  {$\{a, b, d\}$};
            \node (n3) at (1.8, -2)  {$\{a, c, d \}$};
            \node (n4) at (4.4, -2)  {$\{ b, c, d \}$};

            \draw[thick, red] (top) -- (n1);
            \draw[lightgray] (top) -- (n2);
            \draw[lightgray] (top) -- (n3);
            \draw[lightgray] (top) -- (n4);

            \node (n5) at (-5, -5)  {$\{ a,b \}$};
            \node (n6) at (-3, -5)  {$\{ a,c \}$};
            \node (n7) at (-1, -5)  {$\{ a,d \}$};
            \node (n8) at (1, -5)  {$\{ b,c \}$};
            \node[magenta] (n9) at (3, -5)  {$\{ b, d \}$};
            \node[cyan] (n10) at (5, -5)  {$\{ c,d \}$};

            \draw[thick, red] (n1) -- (n5);
            \draw[lightgray] (n1) -- (n6);
            \draw[lightgray] (n1) -- (n8);
            \draw[lightgray] (n2) -- (n5);
            \draw[thick, blue] (n2) -- (n7);
            \draw[lightgray] (n2) -- (n9);
            \draw[thick, green] (n3) -- (n6);
            \draw[lightgray] (n3) -- (n7);
            \draw[lightgray] (n3) -- (n10);
            \draw[thick, violet] (n4) -- (n8);
            \draw[lightgray] (n4) -- (n9);
            \draw[lightgray] (n4) -- (n10);
            
            
            \node (n11) at (-4.4, -8)  {$\{ a\}$};
            \node (n12) at (-1.8, -8)  {$\{ b \}$};
            \node (n13) at (1.8, -8)  {$\{ c \}$};
            \node (n14) at (4.4, -8)  {$\{ d \}$};

            \draw[thick, red] (n5) -- (n11);
            \draw[lightgray] (n5) -- (n12);
            \draw[lightgray] (n6) -- (n11);
            \draw[thick, green] (n6) -- (n13);
            \draw[lightgray] (n7) -- (n11);
            \draw[thick, blue] (n7) -- (n14);
            \draw[thick, violet] (n8) -- (n12);
            \draw[lightgray] (n8) -- (n13);
            \draw[lightgray] (n9) -- (n12);
            \draw[lightgray] (n9) -- (n14);
            \draw[lightgray] (n10) -- (n13);
            \draw[lightgray] (n10) -- (n14);
            
            \node (bottom) at (0, -10) {$\emptyset$};

            \draw[thick, red] (bottom) -- (n11);
            \draw[lightgray] (bottom) -- (n12);
            \draw[lightgray] (bottom) -- (n13);
            \draw[lightgray] (bottom) -- (n14);
            
        \end{tikzpicture} \end{center}
        \caption{A symmetric chain decomposition of $\mathcal{P}(\{a,b,c,d\})$}
        \label{fig:symmchain1}
        \end{figure}
        Where the symmetric chains are:
        \[ \emptyset \subset \{a\} \subset \{a,b\} \subset \{a,b,c\} \subset \{a,b,c,d\} \]
        \[ \{b\} \subset \{b,c\} \subset \{b,c,d\}\]
        \[ \{c\} \subset \{a,c\} \subset \{a,c,d\}\]
        \[ \{d\} \subset \{a,d\} \subset \{a,b,d\}\]
        \[ \{b,d\} \]
        \[ \{ c,d \}. \]
    
        \item Consider $n \in \mathbb{N}$ and the poset ordered by divisibility $(Div(n), \mid\,)$ with rank function $rk(d) = $"number of prime factor of $d$ counted with multiplicity", then $d_1, \dots, d_h \in Div(n)$ form a symmetric chain if:
        \begin{enumerate}
            \item $d_i$ divides $d_{i+1}$, and $\frac{d_{i+1}}{d_i}$ is prime for each $i < h$;
            \item $rk(d_1) + rk(d_h) = r(m)$.
        \end{enumerate}

        For example consider the poset of divisors of $n = 180 = 2^2 \cdot 3^2 \cdot 5$, then a symmetric chains decomposition (depicted in Figure \ref{symmchain2}) is:
            \[ 1 \prec 2 \prec 2^2 \prec 2^2 \cdot 3 \prec 2^2 \cdot 3^2 \prec 2^2 \cdot 3^2 \cdot 5\]
            \[ 3 \prec 2 \cdot 3 \prec 2 \cdot 3^2 \prec 2 \cdot 3^2 \cdot 5\]
            \[ 5 \prec 3 \cdot 5 \prec 2 \cdot 3 \cdot 5 \prec 2^2 \cdot 3 \cdot 5\]
            \[ 3^2 \prec 3^2 \cdot 5\]
            \[ 2 \cdot 5 \prec 2^2 \cdot 5\]
        where with the relation $\prec$ we denote the divisibility relation.
            

        \begin{figure}[h!]
        \begin{center} \begin{tikzpicture}
            \node (top) at (0,0) {$2^2 \cdot 3^2 \cdot 5$};
            \node (n1) at (-3.75, -2)  {$2^2 \cdot 3 \cdot 5$};
            \node (n2) at (0, -2)  {$2 \cdot 3^2 \cdot 5$};
            \node (n3) at (3.75, -2)  {$2^2 \cdot 3^2$};

            \draw[lightgray] (top) -- (n1);
            \draw[lightgray] (top) -- (n2);
            \draw[thick, red] (top) -- (n3);

            \node (n5) at (-5, -5)  {$2^2 \cdot 5$};
            \node (n6) at (-2.5, -5)  {$3^2 \cdot 5$};
            \node (n7) at (0, -5)  {$2\cdot 3\cdot 5$};
            \node (n9) at (2.5, -5)  {$2 \cdot 3^2$};
            \node(n10) at (5, -5)  {$2^2 \cdot 3$};

            \draw[lightgray] (n5) -- (n1);
            \draw[lightgray] (n6) -- (n2);
            \draw[thick, blue] (n7) -- (n1);
            \draw[lightgray] (n7) -- (n2);
            \draw[thick, green] (n9) -- (n2);
            \draw[lightgray] (n9) -- (n3);
            \draw[lightgray] (n10) -- (n1);
            \draw[thick, red] (n10) -- (n3);
            
            \node (n11) at (-5, -8)  {$2 \cdot 5$};
            \node (n12) at (-2.5, -8)  {$3^2$};
            \node (n13) at (0, -8)  {$3\cdot 5$};
            \node (n14) at (2.5, -8)  {$2 \cdot 3$};
            \node (n15) at (5, -8)  {$2^2$};


            \draw[thick, magenta] (n11) -- (n5);
            \draw[lightgray] (n11) -- (n7);
            \draw[thick, violet] (n12) -- (n6);
            \draw[lightgray] (n12) -- (n9);
            \draw[lightgray] (n13) -- (n6);
            \draw[thick, blue] (n13) -- (n7);
            \draw[lightgray] (n14) -- (n7);
            \draw[thick, green] (n14) -- (n9);
            \draw[lightgray] (n14) -- (n10);
            \draw[lightgray] (n15) -- (n5);
            \draw[thick, red] (n15) -- (n10);            
            
            \node (n16) at (-3.75, -11)  {$5$};
            \node (n17) at (0, -11)  {$3$};
            \node (n18) at (3.75, -11)  {$2$};


            \draw[lightgray] (n16) -- (n11);
            \draw[thick, blue] (n16) -- (n13);
            \draw[lightgray] (n17) -- (n12);
            \draw[lightgray] (n17) -- (n13);
            \draw[thick, green] (n17) -- (n14);
            \draw[lightgray] (n18) -- (n14);
            \draw[thick, red] (n18) -- (n15);
            
            \node (bottom) at (0, -13) {$1$};
            
            \draw[lightgray] (bottom) -- (n16);
            \draw[lightgray] (bottom) -- (n17);
            \draw[thick, red] (bottom) -- (n18);

            
            
        \end{tikzpicture} \end{center}        
        \caption{A symmetric chain decomposition of Div$(180)$}
        \label{symmchain2}
        \end{figure}

    \end{enumerate}
\end{es}

We want now to understand how these symmetric chains are constructed.
We begin with the case of divisibility poset, and then we generalize to a general poset.

\begin{thm}
    The set of divisors of a number ordered by divisibility can be expressed as a disjoint union of symmetric chains.
\end{thm}

\begin{proof}
    Denote the divisibility relation with $\prec$, i.e.
    \[ r \prec s \iff r \mid s.\]
    Given $m \in \mathbb{N}$, then we proceed by induction on the number of distinct prime number in the decomposition of $m$, $n$.
    
    Base case: n=1. In this case $m = p^{\alpha}$, with $\alpha \in \mathbb{N}$, so we have a symmetric chain:
    \[ 1 \prec p \prec p^2 \prec \dots \prec p^{\alpha -1} \prec p^{\alpha}\]
    containing all the divisors of $m$.

    Inductive step: suppose that the statement is true for $n$, so for $m = p_1^{k_1} \cdots p_n^{k_n}$

    We want to prove that the statement is true for $n+1$. Consider $p$ one prime in the decomposition of $m$, then we can write $m = m_1 \cdot p^{\alpha}$ with $p \not \mid m_1$, then the number of prime number appearing in the decomposition of $m_1$ is $n$.
    By inductive hypothesis we know that the divisibility poset of $m_1$ can be expressed as a disjoint union of symmetric chain, we want to construct symmetric chains for $m$ using the symmetric chains of $m_1$.
    
    Let $d_1, \dots, d_h$ be one of the symmetric chains for $m_1$. Consider now all the divisors of $m$ of the form:
    \[ d_i \cdot p^{\beta}, \, 0 \le \beta \le \alpha, \, 1 \le i \le h.\]

    Write all these divisors in a rectangular array and then peel off chains as in the figure:

    \begin{center} $\begin{matrix}
        d_1 & d_2 & \dots & d_{h-2} & d_{h-1} & d_{h} \\
        d_1 p & d_2 p & \dots & d_{h-2} p & d_{h-1}p  &  d_{h}p \\
        d_1 p^2 & d_2 p^2 & \dots & d_{h-2} p^2 & d_{h-1}p^2  &  d_{h}p^2 \\
        \vdots & \vdots & \cdots & \vdots & \vdots & \vdots \\
        d_1 p^{\alpha} & d_2 p^{\alpha} & \dots & d_{h-2} p^{\alpha} & d_{h-1}p^{\alpha}  &  d_{h}p^{\alpha} \\
    \end{matrix}$ \end{center}
    Then the outer layer gives the first chain
    \[d_1 \prec d_2 \prec \dots \prec d_{h-2} \prec d_{h-1} \prec d_{h} \prec d_{h}p \prec \dots \prec d_{h}p^{\alpha} \]

    which satisfies condition (1) of the definition of symmetric chain and is symmetric since
    \[ rk(d_1) + rk(d_h p^{\alpha}) = rk(d_1) + rk(d_h) + \alpha = rk(m_1) + rk(p^{\alpha}) = rk(m),\]
    where $rk(d_1) + rk(d_h) = rk(m_1)$ by inductive hypothesis.
    Thus every layer gives a symmetric chain, and every divisor of $m$ can be obtained in this way from a symmetric chain of $m_1$.
\end{proof}

We give here an example to illustrate the procedure given by the theorem.
\begin{es}
        In the Example \ref{ex:chains}.2 we can use the theorem to construct in a recursive way the symmetric chain decomposition.
        Consider $m_1 = 2^2$, then we have only a symmetric chains:
        \[ 1 \prec 2 \prec 2^2\]
        now we consider the divisors of type $2^r \cdot 3^k$ with $r,k=0,1,2$ and we obtain the matrix:
        \begin{center} $\begin{matrix}
            1 & 2 & 2^2 \\
            3 & 2 \cdot 3 & 2^2 \cdot 3 \\
            3^2 & 2 \cdot 3^2 & 2^2 \cdot 3^2
        \end{matrix}$ \end{center}
        and we obtain three symmetric chains for $2^2 \cdot 3^2$:
        \[ 1 \prec 2 \prec 2^2 \prec 2^2 \cdot 3 \prec 2^2 \cdot 3^2\]
        \[ 3 \prec 2 \cdot 3 \prec 2 \cdot 3^2\]
        \[ 3^2 \]
        Consider the first symmetric chain $1 \prec 2 \prec 2^2 \prec 2^2 \cdot 3 \prec 2^2 \cdot 3^2$, then construct the matrix:
        \begin{center} $\begin{matrix}
            1 & 2 & 2^2 & 2^2 \cdot 3 & 2^2 \cdot 3^2 \\
            5 & 2 \cdot 5 & 2^2 \cdot 5 & 2^2 \cdot 3 \cdot 5 & 2^2 \cdot 3^2 \cdot 5
        \end{matrix}$ \end{center}
        and we obtain two more chains:
        \[ 1 \prec 2 \prec 2^2 \prec 2^2 \cdot 3 \prec 2^2 \cdot 3^2 \prec 2^2 \cdot 3^2 \cdot 5\]
        \[ 5 \prec 2 \cdot 5 \prec 2^2 \cdot 5 \prec 2^2 \cdot 3 \cdot 5.\]
        Doing the same with the chain $3 \prec 2 \cdot 3 \prec 2 \cdot 3^2$:
        \begin{center} $\begin{matrix}
            3 & 2 \cdot 3 & 2 \cdot 3^2 \\
            3 \cdot 5 & 2 \cdot 3 \cdot 5 & 2 \cdot 3^2 \cdot 5
        \end{matrix}$ \end{center}

        and obtain two more symmetric chains:
        \[3 \prec 2 \cdot 3 \prec 2 \cdot 3^2 \prec 2 \cdot 3^2 \cdot 5\]
        \[3 \cdot 5 \prec 2 \cdot 3 \cdot 5 \]

        Finally consider the one element chain $3^2$:
        \begin{center} $\begin{matrix}
            3^2 \\
            3^2 \cdot 5
        \end{matrix}$ \end{center}
        and obtain the last symmetric chain:
        \[ 3^2 \prec 3^2 \cdot 5.\]
        
        So in the end we found the chains:
        \[ 1 \prec 2 \prec 2^2 \prec 2^2 \cdot 3 \prec 2^2 \cdot 3^2 \prec 2^2 \cdot 3^2 \cdot 5\]
        \[3 \prec 2 \cdot 3 \prec 2 \cdot 3^2 \prec 2 \cdot 3^2 \cdot 5\]
        \[5 \prec 2 \cdot 5 \prec 2^2 \cdot 5 \prec 2^2 \cdot 3 \cdot 5\]        
        \[3 \cdot 5 \prec 2 \cdot 3 \cdot 5 \]
        \[ 3^2 \prec 3^2 \cdot 5.\]

        In this case we found a different symmetric chain decomposition respect to the one considered in Example \ref{ex:chains}.2, depicted in Figure \ref{symmchain3}.

        \begin{figure}
        \begin{center} \begin{tikzpicture}
            \node (top) at (0,0) {$2^2 \cdot 3^2 \cdot 5$};
            
            \node (n1) at (-3.75, -2)  {$2^2 \cdot 3 \cdot 5$};
            \node (n2) at (0, -2)  {$2 \cdot 3^2 \cdot 5$};
            \node (n3) at (3.75, -2)  {$2^2 \cdot 3^2$};

            \draw[lightgray] (top) -- (n1);
            \draw[lightgray] (top) -- (n2);
            \draw[thick, red] (top) -- (n3);

            \node (n5) at (-5, -5)  {$2^2 \cdot 5$};
            \node (n6) at (-2.5, -5)  {$3^2 \cdot 5$};
            \node (n7) at (0, -5)  {$2\cdot 3\cdot 5$};
            \node (n9) at (2.5, -5)  {$2 \cdot 3^2$};
            \node(n10) at (5, -5)  {$2^2 \cdot 3$};

            \draw[thick, blue] (n5) -- (n1);
            \draw[lightgray] (n6) -- (n2);
            \draw[lightgray] (n7) -- (n1);
            \draw[lightgray] (n7) -- (n2);
            \draw[thick, green] (n9) -- (n2);
            \draw[lightgray] (n9) -- (n3);
            \draw[lightgray] (n10) -- (n1);
            \draw[thick, red] (n10) -- (n3);
            
            \node (n11) at (-5, -7)  {$2 \cdot 5$};
            \node (n12) at (-2.5, -7)  {$3^2$};
            \node (n13) at (0, -7)  {$3\cdot 5$};
            \node (n14) at (2.5, -7)  {$2 \cdot 3$};
            \node (n15) at (5, -7)  {$2^2$};

            \draw[thick, blue] (n11) -- (n5);
            \draw[lightgray] (n11) -- (n7);
            \draw[thick, violet] (n12) -- (n6);
            \draw[lightgray] (n12) -- (n9);
            \draw[lightgray] (n13) -- (n6);
            \draw[thick, magenta] (n13) -- (n7);
            \draw[lightgray] (n14) -- (n7);
            \draw[thick, green] (n14) -- (n9);
            \draw[lightgray] (n14) -- (n10);
            \draw[lightgray] (n15) -- (n5);
            \draw[thick, red] (n15) -- (n10);            
            
            \node (n16) at (-3.75, -10)  {$5$};
            \node (n17) at (0, -10)  {$3$};
            \node (n18) at (3.75, -10)  {$2$};


            \draw[thick, blue] (n16) -- (n11);
            \draw[lightgray] (n16) -- (n13);
            \draw[lightgray] (n17) -- (n12);
            \draw[lightgray] (n17) -- (n13);
            \draw[thick, green] (n17) -- (n14);
            \draw[lightgray] (n18) -- (n14);
            \draw[thick, red] (n18) -- (n15);
            
            \node (bottom) at (0, -12) {$1$};
            
            \draw[lightgray] (bottom) -- (n16);
            \draw[lightgray] (bottom) -- (n17);
            \draw[thick, red] (bottom) -- (n18);
        \end{tikzpicture} \end{center} 
        \caption{A symmetric chain decomposition for Div$(180)$.}
        \label{symmchain3}
        \end{figure}
\end{es}

\begin{rmk}
    \begin{enumerate}
        \item As seen in the previous two examples symmetric chain decomposition are not unique.
        
        \item The recursive procedure can be extended to monomial. In fact: consider the monomial $p = x_1 ^{n_1} \cdots x_k ^{n_k}$, then we can consider $x_1, \dots, x_k$ as the prime factors appearing in the decomposition of $p$. So considering the poset of divisors of $p$ ordered by divisibility, we can use the procedure described in the theorem to find a symmetric chain decomposition.
    
        \item The procedure can also be extended to set. It is possible in fact to give a bijective relation between sets and monomial:
        \[ \{1, 2, \dots, n\} \longleftrightarrow x_1 \cdot x_2 \cdots x_n,\]
        and observe that the two relations are equivalent:
        \[ \{i_1, \dots, i_k\} \subseteq \{j_1, \dots, j_r\} \iff x_{i_1} \cdots x_{i_k} \mid x_{j_1} \cdots x_{j_r}.\]
        So we can treat the case of poset and monomial in the same way.
    \end{enumerate}
\end{rmk}


\section {The Main Theorem}

Let $\mathcal{M}$ be a simple matroid with a lattice of flats $\mathcal{L} = \mathcal{L}_{\mathcal{M}}$, $\mathcal{G} \subset \mathcal{L} \setminus \{ \hat{0} \}$ that contains $\hat{1} = E$. Let $G \subset Aut(\mathcal{L})$ a subgroup setwise stabilizing $\mathcal{G}$, i.e.
\[ \forall g \in G, \, x \in \mathcal{G} \implies g(x) \in \mathcal{G},\]

and such that $G$ satisfies the stabilizer condition:
\begin{equation} \label{stabcond}
    N= \{ F_i \}_{i = 1, \dots, l} \in \mathcal{N}(\mathcal{L}, \mathcal{G}), \, g \in G, \, g(N) = N \implies g(F_i) = F_i \, \forall i \in \{1, \dots, l\}.
\end{equation}

We also recall here the definition of the Feichtner-Yuzvinsky monomial basis for the Chow ring $\mathcal{A}(\mathcal{L}, \mathcal{G})$:
\[ FY := \left\{ x_{F_1}^{m_1} \cdots x_{F_l}^{m_l} \, : \, N=\{F_1, \dots, F_l \} \in \mathcal{N}(\mathcal{L},\mathcal{G}), \, 1 \le m_i \le m_N(F_i), \, i \in \{ 1, \dots, l\} \right\}\]

where $m_N(F) = rk(F) - rk(\vee N_{\le F}).$

We can now proceed with the proof of the main theorem.

\begin{thm} \label{mainthm}
    Let $\mathcal{M}$ be a simple matroid with a lattice of flats $\mathcal{L} = \mathcal{L}_{\mathcal{M}}$, $\mathcal{G} \subset \mathcal{L} \setminus \{ \hat{0} \}$ that contains $\hat{1} = E$. Let $G \subset Aut(\mathcal{L})$ a subgroup setwise stabilizing $\mathcal{G}$, and satisfying the condition \ref{stabcond}. Then there exists:
    \begin{enumerate}
        \item $G$-equivariant bijections $\pi : FY^k \longrightarrow FY^{r-k}$, for $k \le \frac{r}{2}$;

        \item $G$-equivariant injections $\lambda : FY^k \longrightarrow FY^{k+1}$, for $k < \frac{r}{2}$.
    \end{enumerate}
\end{thm}

\begin{proof}
    Define the function
    \[ supp_+ : FY \longrightarrow \mathcal{N}(\mathcal{L}, \mathcal{G})\]
    called extended support map, defined as follows: if $m_i \ge 1$ for $1 \le i \le l$, then
    \[ supp_+(x_{F_1}^{m_1} \cdots x_{F_l}^{m_1l}) := \{F_1, \dots, F_l \} \cup \{E\};\]
    observe that $supp_+$ is well-defined because $\mathcal{N}(\mathcal{L}, \mathcal{G})$ is a simplicial complex, i.e.
    \[ \{F_1, \dots, F_l\} \in \mathcal{N}(\mathcal{L}, \mathcal{G}) \implies \{F_1, \dots, F_l\} \cup \{E \} \in \mathcal{N}(\mathcal{L}, \mathcal{G}). \]
    
    We want to use the function $supp_+$ to cover the set $FY$ using its fiber. Consider $N^+ =\{F_1, \dots, F_l, E\} \subset \mathcal{G}$ a $\mathcal{G}$-nested set in the image of $supp_+$ then the fiber has the following description:
    \[ supp_+^{-1} (N^+) = \{x_{F_1}^{m_1} \cdots x_{F_l}^{m_l} \cdot x_E^{m_{l+1}}  \}\]
    where the exponents satisfy the inequalities:
    \[ 1 \le m_i \le m_{N^+}(F_i) - 1, \text{ for } 1 \le i \le l;\]
    \[ 0 \le m_{l+1} \le m_{N^+}(E)- 1\]

    As a consequence of this description the degree of the monomials in $supp_+^{-1}(N^+)$ lie in the range $\left[ l, r-l \right] \subset \mathbb{N}$. In fact, to obtain the minimal degree we consider the monomial with the minimum exponent in every variable, this is achieved by the monomial $x_{F_1} ^ {1} \cdots x_{F_l} ^ {1}  x_E^0$, which has degree 
    \[ \deg(x_{F_1} ^ {1} \cdots x_{F_l} ^ {1}  x_E^0) = l.\]
    To obtain the maximum degree we need to consider the maximum exponent in every variable, and this is achieved by the monomial $x_{F_1} ^ {m_{N^+}(F_1) -1} \cdots x_{F_l} ^ {m_{N^+}(F_l) -1} x_E^{m_{N^+}(E)-1}$
    which has degree 
    
    \begin{align*}    
        \deg(x_{F_1} ^ {m_{N^+}(F_1) -1} \cdots x_{F_l} ^ {m_{N^+}(F_l) -1} x_E^{m_{N^+}(E)-1}) &= \sum_{F \in N^+}(m_{N^+}(F) -1)
        \\&= rk(\vee N^+) - |N^+| \\ &= (r+1) - (l+1) = r-l.
    \end{align*}

    Consider a $G$-orbit of $\mathcal{G}$-nested set, i.e.
    \[ \mathcal{O}(N) = \{g\cdot N \, : \, N \in \mathcal{N}(\mathcal{L}, \mathcal{G})\},\]
    and consider a representative of this orbit $N = \{F_1, \dots, F_l \}$.
    Choose in this representative $N$ a linear order $(F_1, \dots, F_l)$, and on every other element of the orbit $g(N) \in \mathcal{O}(N)$ the order $(g(F_1), \dots, g(F_l))$. Observe now that the choice of the order in $g(N)$ is independent of $g \in G$. In fact by condition \ref{stabcond}, if $g, h \in G$ are such that $g(N) = h(N)$, then $h^{-1}g(N) = N$, so, since $h^{-1}g \in G$, $h^{-1}g(F_i) = F_i$ and $g(F_i)= h(F_i)$  for $i = 1, \dots, l$.
    Thus we proved that:
    \[ \forall g, h \in G, \  g(N) = h(N) \implies g(F_i) = h(F_i), \ \forall i \in \{1, \dots, l\}.\]

    For what we just proved, for every $\mathcal{G}$-nested set $N^+ = \{ F_1, \dots, F_l, E\}$ in the image of $supp_+$, we can choose a linear order on the elements $(F_1, \dots, F_l)$, such that these orders satisfy the following condition: given $N, N'$ $\mathcal{G}$-nested sets with chosen linear orders $(F_1, \dots, F_l)$, $(F_1', \dots, F_l')$, then:
    \[ \forall g \in G, \  g(N) = N' \implies g(F_i) = F_i', \  \forall i \in \{ 1,2,\dots,l\}.\]

    In particular if $g(N) = N'$, then in $N'$ we have the linear order $(g(F_1), \dots, g(F_l))$.

    Observe now that the set $supp_+^{-1}(N^+)$ when ordered via divisibility, gives a poset isomorphic to a cartesian product of chain
    \begin{equation} \label{prodposet}
    supp_+^{-1}(N^+) \simeq C_{m_{N^+}(F_1) -1} \times C_{m_{N^+}(F_2) -1} \times \dots \times C_{m_{N^+}(F_l) -1} \times C_{m_{N^+}(E)}.
    \end{equation}
    where $C_m$ denotes a chain (totally ordered set) of size $m$.
        
    In fact, given $N^+ = \{F_1, \dots, F_l, F_{l+1} = E\}$, consider two monomial $x_{F_1}^{m_1} x_{F_2}^{m_2} \cdots x_{F_l}^{m_l}x_{F_{l+1}}^{F_{l+1}}$ and $x_{F_1}^{n_1}x_{F_2}^{n_2} \cdots x_{F_l}^{n_l}x_{F_{l+1}}^{n_{l+1}}$ in $supp_+^{-1}(N^+)$, and since by hypothesis $supp_+^{-1}(N^+)$ is ordered by divisibility we have the following condition:
    \begin{equation} \label{conddiv}
     x_{F_1}^{m_1} x_{F_2}^{m_2} \cdots x_{F_l}^{m_l}x_{F_{l+1}}^{m_{l+1}} \mid x_{F_1}^{n_1}x_{F_2}^{n_2} \cdots x_{F_l}^{n_l}x_{F_{l+1}}^{n_{l+1}}  \iff m_1 < n_1, \, m_2 < n_2, \, \dots, m_{l+1} < n_{l+1},
     \end{equation}
    where $<$ is the usual order in the natural number.

    Define
    \[ C_m := \{ 1 < 2 < 3 < \dots < m-1 \}\]
    so by the definition of the product of poset, on the poset $C_m \times C_n$ we have the order:
    \begin{equation} \label{defprodrel} 
    (p, q) < (p', q') \iff p < p', \, q < q'.
    \end{equation}
    Thus, since the exponent of $x_{F_i}$ ranges in $\left[ 1, 2, \dots, m_{N^+}(F_i)-1 \right] \subset \mathbb{N}$, by the definition of $C_m$, condition \ref{conddiv}, and \ref{defprodrel} we have the isomorphism

      \[  supp_+^{-1}(N^+) \simeq C_{m_{N^+}(F_1) -1} \times C_{m_{N^+}(F_2) -1} \times \dots \times C_{m_{N^+}(F_l) -1} \times C_{m_{N^+}(E)}.\]

    By the what we discussed in the previous section we know that every ranked $\mathcal{P}$ of rank $r$ that is product of chain has a symmetric chain decomposition, i.e. \[ \mathcal{P} = \bigsqcup_{i=1}^t \mathcal{P}_i \] where $\mathcal{P}_i$ is a chain passing through the ranks \[ \rho_i, \, \rho_i + 1, \, \dots, r - \rho_i -1, \, r- \rho_i \] for some $\rho_i \in \{0, 1, 2, \dots, \floor{\frac{r}{2}} \}$. 

    So for every poset of the type \[ C_{m_{N^+}(F_1) -1} \times C_{m_{N^+}(F_2) -1} \times \dots \times C_{m_{N^+}(F_l) -1} \times C_{m_{N^+}(E)}\]
    we can choose a symmetric chain decomposition and a linear order $(F_1, \dots, F_l)$, and this induce a symmetric chain decomposition on each fiber poset $supp_+^{-1}(F_1, \dots, F_l, E)$.

    Observe now that the structure of these symmetric chains depend only on the linear orders $(F_1, \dots, F_l)$ and on the numerical sequences \[m_{N ^+}(F_1), m_{N ^+}(F_2), \dots,m_{N ^+}(F_l), m_{N ^+}(E),\] and it doesn't require finer information about the nested set $N^+$.

    Now we can define the two maps $\pi$ and $\lambda$ by describing their behavior on the chains of the symmetric chain decomposition of the fibers. Let $a$ be any $FY$-monomial with $k = \deg(a)$, then $a$ lies on a unique fiber $supp_+^{-1}(N^+)$ for some $\mathcal{G}$-nested set $N^+ =\{F_1, \dots ,F_l, E\}$, and on a unique symmetric chain $C_i$, of the symmetric chain decomposition of $supp_+^{-1}(N^+)$,
    \[ C_i = \{ a_\rho \lessdot a_{\rho +1 } \lessdot \dots \lessdot a_{r-\rho-r} \lessdot a_{r-\rho}\},\]
    where $\deg(a_j) = j$ with $j = \rho, \rho +1 , \dots, r- \rho -1, r - \rho$.

    Then, since $a \in C_i$, then exits $k \in \{ \rho, \rho +1 , \dots, r- \rho -1, r - 
    \rho \}$, such that $a = a_k$, so we define:
    \[ \lambda (a) := a_{k+1} \text{ if $k<\frac{r}{2}$}\]
    \[ \pi (a) := a_{r-k} \text{ if $k \le \frac{r}{2}$}.\]
    

    We now want to prove that $\lambda$ is an injection.
    Taking $a \in FY^{k+1}$, with $k<\frac{r}{2}$, there exists a $\mathcal{G}$ nested set $N^+$ such that $a \in supp_+^{-1}(N^+)$, since the fibers cover all $FY$. Given a symmetric chain decomposition of $supp_+^{-1}(N^+) = \bigsqcup_{i=1}^l C_i$, there exists $m \in \{1, 2, \dots, l\}$ such that $a \in C_m$ and $\rho_m < k$, i.e.
    \[ C_m = \{ a_{\rho_m} \lessdot a_{{\rho_m} +1 } \lessdot \dots \lessdot a_{r-\rho_m-r} \lessdot a_{r-\rho_m}\} \]
    so there exists a monomial $a' \in C_m$ such that $\deg(a') = k$, thus $\lambda(a') = a$.

    To prove that the map $\pi : FY^{k} \longrightarrow FY^{r-k}$, observe that the fibers are disjoint. Taking an element $a \in FY^{r-k}$, with $k \le \frac{r}{2}$, there exists a unique fiber $supp_+^{-1}(N^+)$ such that $a \in supp_+^{-1}(N^+)$, and given a symmetric chain decomposition of $supp_+^{-1}(N^+) = \bigsqcup_{i=1}^l C_i$, there exists $m \in \{1, 2, \dots, l\}$ such that $a \in C_m$ and $\rho_m < k$, i.e.
    \[ C_m = \{ a_{\rho_m} \lessdot a_{{\rho_m} +1 } \lessdot \dots \lessdot a_{r-\rho_m-r} \lessdot a_{r-\rho_m}\} \]
    so there exists a monomial $a' \in C_m$ such that $\deg(a') = k$, thus $\pi(a') = a$.
    By the uniqueness of the fiber, the element $a' \in FY^k$ is unique, so $\pi$ is an isomorphism.

    Finally, we prove that $\pi$ and $\lambda$ are $G$-equivariant.
    Consider $N^+ = \{F_1, \dots, F_l, F_{l+1} := E \}$, and since $\pi$ and $\lambda$ act similarly we can consider a function $f$ either equalt to $\pi$ or $\lambda$. Then $f$ maps a monomial $a = \prod_{i = 1}^{l+1} x_{F_i}^{m_i}$ as follows:
    \[ a = \prod_{i = 1}^{l+1} x_{F_i}^{m_i} \longmapsto f(a) = \prod_{i = 1}^{l+1} x_{F_i}^{m_i'}\]
    where the exponents $m_1',\,  \dots, \, m_l', \, m_{l+1}'$ are uniquely determined by the linear ordering $(F_1, \dots, F_l)$ and by the data:
    \[\left( (m_1, \dots, m_l, \, m_{l+1}), (m_{N ^+}(F_1),\, m_{N ^+}(F_2),\, \dots,m_{N ^+}(F_l),\, m_{N ^+}(E)) \right).\]
    We already proved in Corollary \ref{cor:ZGmod} that 
    \[ \forall g \in G, \ m_{g(N)} (g(F)) = m_N(F),\] moreover, by what we discussed above, our choice of the linear ordering $(F_1, \dots, F_l)$ dictates that $g(N^+) = \{ g(F_1), \dots, g(F_l), E\}$ has the linear ordering $(g(F_1), \dots, g(F_l))$.
    Thus $f$ sends
    \[ g(a) = \prod_{i=1}^{l+1} x_{g(F_i)}^{m_i} \longmapsto f(g(a)) = \prod_{i=1}^{l+1} x_{g(F_i)}^{m_i'} = g\left(\prod_{i=1}^{l+1} x_{F_i}^{m_i'} \right) = g(f(a)),\]
    so $f$ is $G$-equivariant.
    
\end{proof}

We propose here an example of symmetric chain decomposition of a fiber that we will also use later.

\newpage
\begin{es} \label{ex:fiber}

Let $\mathcal{L} = \mathcal{L}_{\mathcal{M}}$ be the lattice of flats of a simple matroid $\mathcal{M}$ having $rk(E) = 10 = r+1$ with $r=9$ and $\mathcal{G} = \mathcal{G_{\max}}$. Consider a $\mathcal{G}$-nested set $N^+ = \{F_1, F_2, E\}$ where $F_1 < F_2$ are flats with $rk(F_1) = 3$, $rk(F_2) = 7$. Linearly ordering $\{F_1, F_2 \}$ as $(F_1, F_2)$, then the divisibility poset on the fiber $supp_+^{-1}(\{F_1, F_2, E \})$ is depicted in Figure \ref{fig:suppfiber1}.
And a possible choice of symmetric chain decomposition looks as depicted in Figure \ref{fig:symmchainsupp}.

\begin{figure}[h!]
\begin{center} \begin{tikzpicture}
    \node (top) at (0,0) {$x_{F_1}^2x_{F_2}^3x_E^2$};
            
    \node (n1) at (-2.4, -2)  {$x_{F_1}^2x_{F_2}^3x_E$};
            \node (n2) at (0, -2)  {$x_{F_1}x_{F_2}^3x_E^2$};
            \node (n3) at (2.4, -2)  {$x_{F_1}^2x_{F_2}^2x_E^2$};

            \draw (top) -- (n1);
            \draw (top) -- (n2);
            \draw (top) -- (n3);
            
            \node (n5) at (-4.8, -4)  {$x_{F_1}^2x_{F_2}^3$};
            \node (n6) at (-2.4, -4)  {$x_{F_1}x_{F_2}^3x_E$};
            \node (n7) at (0, -4)  {$x_{F_1}^2x_{F_2}^2x_E$};
            \node (n8) at (2.4, -4)  {$x_{F_1}x_{F_2}^2x_E^2$};
            \node (n9) at (4.8, -4) {$x_{F_1}^2x_{F_2}x_E^2$};

            \draw (n1) -- (n5);
            \draw (n1) -- (n6);
            \draw (n1) -- (n7);
            \draw (n2) -- (n6);
            \draw (n2) -- (n8);
            \draw (n3) -- (n7);
            \draw (n3) -- (n8);
            \draw (n3) -- (n9);
                        
            \node (n10) at (-4.8, -6)  {$x_{F_1}x_{F_2}^3$};
            \node (n11) at (-2.4, -6)  {$x_{F_1}^2x_{F_2}^2$};
            \node (n12) at (0, -6)  {$x_{F_1}x_{F_2}^2x_E$};
            \node (n13) at (2.4, -6)  {$x_{F_1}^2x_{F_2}x_E$};
            \node (n14) at (4.8, -6) {$x_{F_1}x_{F_2}x_E^2$};

            \draw (n5) -- (n10);
            \draw (n5) -- (n11);
            \draw (n6) -- (n10);
            \draw (n6) -- (n12);
            \draw (n7) -- (n11);
            \draw (n7) -- (n12);
            \draw (n7) -- (n13);
            \draw (n8) -- (n12);
            \draw (n8) -- (n14);
            \draw (n9) -- (n13);
            \draw (n9) -- (n14);
            
            \node (n15) at (-2.4, -8)  {$x_{F_1}^2x_{F_2}x_{F_3}$};
            \node (n16) at (0, -8)  {$x_{F_1}x_{F_2}^2x_{F_3}$};
            \node (n17) at (2.4, -8)  {$x_{F_1}x_{F_2}x_{F_3}^2$};

            \draw (n10) -- (n15);
            \draw (n11) -- (n15);
            \draw (n12) -- (n15);
            \draw (n11) -- (n16);
            \draw (n13) -- (n16);
            \draw (n12) -- (n17);
            \draw (n13) -- (n17);
            \draw (n14) -- (n17);
            
            \node (bottom) at (0, -10) {$x_{F_1}x_{F_2}$};

            \draw (n15) -- (bottom);
            \draw (n16) -- (bottom);
            \draw (n17) -- (bottom);
\end{tikzpicture} \end{center}
\caption{Hasse diagram of the fiber $supp_+^{-1}(F_1, F_2, E)$.}
\label{fig:suppfiber1}
\end{figure}


\begin{figure}[h!]
\begin{center} \begin{tikzpicture}
    \node (top) at (0,0) {$x_{F_1}^2x_{F_2}^3x_E^2$};
            
    \node (n1) at (-2.4, -2)  {$x_{F_1}^2x_{F_2}^3x_E$};
            \node (n2) at (0, -2)  {$x_{F_1}x_{F_2}^3x_E^2$};
            \node (n3) at (2.4, -2)  {$x_{F_1}^2x_{F_2}^2x_E^2$};

            \draw (top) -- (n2);
            
            \node (n5) at (-4.8, -4)  {$x_{F_1}^2x_{F_2}^3$};
            \node (n6) at (-2.4, -4)  {$x_{F_1}x_{F_2}^3x_E$};
            \node (n7) at (0, -4)  {$x_{F_1}^2x_{F_2}^2x_E$};
            \node (n8) at (2.4, -4)  {$x_{F_1}x_{F_2}^2x_E^2$};
            \node (n9) at (4.8, -4) {$x_{F_1}^2x_{F_2}x_E^2$};

            \draw (n1) -- (n6);
            \draw (n2) -- (n8);
            \draw (n3) -- (n9);
                        
            \node (n10) at (-4.8, -6)  {$x_{F_1}x_{F_2}^3$};
            \node (n11) at (-2.4, -6)  {$x_{F_1}^2x_{F_2}^2$};
            \node (n12) at (0, -6)  {$x_{F_1}x_{F_2}^2x_E$};
            \node (n13) at (2.4, -6)  {$x_{F_1}^2x_{F_2}x_E$};
            \node (n14) at (4.8, -6) {$x_{F_1}x_{F_2}x_E^2$};

            \draw (n5) -- (n10);
            \draw (n6) -- (n12);
            \draw (n7) -- (n11);
            \draw (n8) -- (n14);
            \draw (n9) -- (n13);
            
            \node (n15) at (-2.4, -8)  {$x_{F_1}^2x_{F_2}x_{F_3}$};
            \node (n16) at (0, -8)  {$x_{F_1}x_{F_2}^2x_{F_3}$};
            \node (n17) at (2.4, -8)  {$x_{F_1}x_{F_2}x_{F_3}^2$};

            \draw (n12) -- (n15);
            \draw (n13) -- (n16);
            \draw (n14) -- (n17);
            
            \node (bottom) at (0, -10) {$x_{F_1}x_{F_2}$};

            \draw (n17) -- (bottom);
\end{tikzpicture} \end{center}
\caption{A possible symmetric chain decomposition for $supp_+^{-1}(F_1, F_2, E)$.}
\label{fig:symmchainsupp}
\end{figure}
\end{es}    

\newpage
We propose here an alternative proof in the case $\mathcal{G} = \mathcal{G}_{\max}$ which is of a purely combinatorial nature.
We start by defining a particular encoding of $FY$-monomials in a fiber of $supp_+$. 

\begin{defn}
    Consider all $FY$-monomials $a = x_{F_1}^{m_1} \cdots x_{F_l}^{m_l}x_E^{m_{l+1}}$ having a fixed extended support set \[ supp_+ (a) = \{F_1 \le \dots \le F_l \le F_{l+1} = E \},\] so $m_1, \dots, m_l \ge 1$ and $m_{l+1} \ge 0$. We define a two steps encoding of $a$ as:
    \begin{enumerate}
        \item[$1^{st} step:$] encode $a$ via a sequence $\mathcal{D}(a)$ of length $r$ of three symbols $\times$, $\bullet $ and the empty space, defined as follows:
        \begin{enumerate}
            \item[(1)] $\mathcal{D}(a)$ has $\bullet$ in the positions $rk(F_1), \dots, rk(F_l)$;
            \item[(2)] $\mathcal{D}(a)$ has $\times$ in the first consecutive $m_i$ positions to the left of $rk(F_i)$ for each $i=1, \dots, l, l+1$;
            \item[(3)] $\mathcal{D}(a)$ has a blank space in the remaining positions.
        \end{enumerate}

        \item[$2^{nd} step:$] in the second setp we econde $\mathcal{D}(a)$ as a length $r$ parenthesis sequence, i.e. in $\{(,) \}^r$, having:
        \begin{enumerate}
            \item[(1)] a right parenthesis ")" in the position of each $\bullet$, blank space;
            \item[(2)] a left parenthesis "(" in each position of each $\times$.
        \end{enumerate}
    \end{enumerate}
\end{defn}

To clarify the encoding algorithm consider the following example.
\begin{es} \label{ex:econding}
    Consider the matroid $\mathcal{M}$ and its flats $F_1 < F_2$ as discussed in Example \ref{ex:fiber}. So we have $rk(F_1) = 3$, $rk(F_2) = 7$ and $rk(E) = 10$, so $r = 9$, and the monomials lie in the fiber $supp_+^{-1}(\{F_1, F_2, E \})$. The monomial $x_{F_1}  x_{F_2}^2$ by the first step encodes as
    
    \begin{center} $\begin{matrix}
        1 & 2  & 3 & 4 & 5 & 6 & 7 & 8 & 9 \\
        \ & \times & \bullet & \ & \times & \times & \  & \  &
    \end{matrix}$ \end{center}

    so $\mathcal{D}(x_{F_1} x_{F_2}^2) = \{ \_ , \times , \bullet , \_ , \times , \times , \_  , \_  \}$

    Then following the second step we obtain
    
    \begin{center} $\begin{matrix}
        1 & 2  & 3 & 4 & 5 & 6 & 7 & 8 & 9 \\
        \ & \times & \bullet & \ & \times & \times & \  & \  & \\
        ) & ( & ) & ) & ( & ( & )  & )  & )
    \end{matrix}$ \end{center}

    so the $\{ (,) \}^r$ sequence is $\{) , ( , ) , ) , ( , ( , )  , )  , ) \}$.
\end{es}

\begin{rmk}
    Observe that is possible to follow the backward procedure, so starting with the $\{ (,) \}^r$ sequence we can recover the initial monomial.
    First we recover $\mathcal{D}(a)$ from the $\{ (,) \}^r$-sequence. The procedure is the following:
    \begin{enumerate}
        \item[(1)] $\mathcal{D}(a)$ has $\times$ occurring in the positions of the left parenthesis "(";
        \item[(2)] $\mathcal{D}(a)$ has $\bullet$ occurring exactly in the positions of the right parenthesis ")" in a consecutive pair "()", while the blank spaces occur in the position of all other right parenthesis.
    \end{enumerate}

    To recover the monomial $a$ from $\mathcal{D}(a)$, consider $supp_+(a) = \{ F_1, F_2, \dots , F_l, E\}$, then $m_i$ can be read as the number of $\times$ in $\mathcal{D}(a)$ between the positions $rk(F_{i-1})$ and $rk(F_i)$, with usual convention $F_0 = \hat{0}$, and $F_{l+1}=E$.
\end{rmk}

We are now able to give a combinatorical proof of the Theorem \ref{mainthm}. 
During the proof we will use examples to understand how it works.

\begin{proof}[Proof of Theorem \ref{mainthm} for $\mathcal{G} = \mathcal{G}_{\max}$]
Given $a \in FY^k$ with $k \le \frac{r}{2}$, consider the $\{ (,) \}^r$-sequence associated to $a$, and call it $\mathcal{P}_1(a)$.

Define the sequence of all the the paired parenthesis of the type "()", $\mathcal{S}_1(a)$, as a sequence that preserve only the pairs "()", an in the other places has a blank space.
Consider for example the sequence in Example \ref{ex:econding}:
\[ \mathcal{P}_1(a) = \{) , ( , ) , ) , ( , ( , )  , )  , ) \} \]
then the sequence of paired parenthesis is $\mathcal{S}_1(a) = \{\_ , ( , ) , \_ , \_, ( , ) , \_ , \_ \}$.

Then define a new $\{ (,) \}^r$-sequence, called $\mathcal{P}_2(a)$ obtained by removing the pairs in $\mathcal{S}_1(a)$ from $\mathcal{P}_1(a)$. 
In our example we obtain the sequence \[ \mathcal{P}_2(a) = \{) , ) , (  , )  , ) \}, \] and then define the sequence $\mathcal{S}_2(a)$ formed as $\mathcal{S}_1(a)$ but starting by the sequence $\mathcal{P}_2(a)$. So in our example we obtain:  \[ \mathcal{S}_2(a) = \{ \_ , \_ , (  , )  , \_ \}. \]
Remove from $\mathcal{P}_2(a)$ the pairs in $\mathcal{S}_2(a)$ to obtain $\mathcal{P}_3(a)$, and repeat the process until we obtain $\mathcal{P}_k(a)$ such that $\mathcal{S}_k(a) = \emptyset$, i.e. there are no more pairs "()" in $\mathcal{P}_k(a)$.
In our example we stop at $k = 3$ since in \[ \mathcal{P}_3(a)=\{ ),),) \} \] there are no pair of type "()".

Suppose in general we removed a number $\rho$ of pairs "()", so the final parenthesis sequence consists of $r - 2\rho$ parenthesis of this form:
    \[ \mathcal{P}(a) =  \underbrace{)) \dots ))}_\text{$k - \rho$} \underbrace{(( \dots ((}_\text{$r - k - \rho$}.\]

We want to use this new parenthesis sequence to place $a$ in a chain of monomials 
    \[ C_i = \{ a_{\rho} \lessdot a_{\rho +1} \lessdot \dots \lessdot a_{r-\rho-1} \lessdot a_{r - \rho}\},\]
    where $\deg(a_j) = j$ for $j = \rho, \rho +1, \dots, r- \rho$ with $a = a_k$. To do that we define the symmetric chain $C_i$ as the chain of monomials whose set of paired parenthesis agree exactly with those of $a$, both in their positions and left-right pairing structure, i.e.
    \[ C_i = \{ a' \in FY \, : \, \mathcal{S}_k (a) = \mathcal{S}_k(a'), \, \forall k \ge 0 \}. \]

    Once we define a chain $C_i$ such that $ a \in C_i$, we can define the maps $\pi, \lambda$ as follows:
    \[ \pi (a) := a_{r-k} \text{ for } k \le \frac{r}{2},\]
    \[ \lambda (a) := a_{k+1} \text{ for } k < \frac{r}{2}.\]
       
To clarify how $\pi$ and $\lambda$ act on the monomials via the $\{(,)\}^r$-sequence, consider the following symmetric chain

    \begin{center} $\begin{matrix}
        \                      & 1 & 2 & 3 & 4 & 5 & 6 & 7 & 8 & 9 \\
        x_{F_1}^2 x_{F_2}^3x_E & \times & \times & \bullet & \times &\times & \times &\bullet & \  & \times \\
        \  & \textcolor{orange}{(} & \underline{(} & \underline{)} & \textcolor{orange}{(} & \underline{(} & \underline{(} & \underline{)} & \underline{)} & \textcolor{orange}{(} \\
        \uparrow &&&&&&&&&\\
        &&&&&&&&&\\
        x_{F_1} x_{F_2}^3 x_E & \  & \times & \bullet & \times &\times & \times &\bullet & \  & \times \\
        \  & ) & \underline{(} & \underline{)} & \textcolor{orange}{(} & \underline{(} & \underline{(} & \underline{)} & \underline{)} & \textcolor{orange}{(} \\
        \uparrow &&&&&&&&&\\
        &&&&&&&&&\\
        x_{F_1} x_{F_2}^2x_E & \  & \times & \bullet & \  &\times & \times &\bullet & \  & \times \\
        \  & ) & \underline{(} & \underline{)} & ) & \underline{(} & \underline{(} & \underline{)} & \underline{)} & \textcolor{orange}{(} \\
        \uparrow &&&&&&&&&\\
        &&&&&&&&&\\
        x_{F_1} x_{F_2}^2 & \  & \times & \bullet & \  &\times & \times &\bullet & \  & \  \\
        \  & ) & \underline{(} & \underline{)} & ) & \underline{(} & \underline{(} & \underline{)} & \underline{)} & )
    \end{matrix}$   \end{center}

Observe that paired parenthesis, that are underlined, are fixed throughout the chain, and moving up the chain unpaired right parenthesis change one-by-one to left parenthesis (the changed parenthesis are colored in \textcolor{orange}{orange}).

Using this example it is clear how $\pi$ and $\lambda$ acts:
    \begin{enumerate}
        \item $\lambda$ changes the leftmost unpaired right parenthesis ")" into an unpaired left parenthesis "(";
        \item $\pi$ swaps the number $k - \rho$ and $r - \rho -k$ of unpaired right and left parenthesis.
    \end{enumerate}
\end{proof}

Finally we discuss the importance of the stabilizer condition wich we recall here: for a building set $N= \{ F_i \}_{i = 1, \dots, l} \in \mathcal{N}(\mathcal{L},\mathcal{G})$
\begin{equation} \label{stabcond2}
     g \in G, \, g(N) = N \implies g(F_i) = F_i, \, \forall i \in \{1, \dots, l\}.
\end{equation}

In the proof of Theorem \ref{mainthm} we used the condition \ref{stabcond2} to guarantee the existence of an order which is invariant on an orbit by the action of $G$ of a nested set.
We want now to prove that Theorem \ref{mainthm} can fail without this assumption.The idea is to find a nested set $N$, setwise stabilized by the group $G$, but for which the \ref{stabcond2} does not hold. 

First we want to define a product of symmetric groups called wreath product.

\begin{defn}
    Consider $n, m \in \mathbb{N}$ and consider the two symmetric groups $\mathfrak{S}_n, \mathfrak{S}_m$ respectively of the sets $\{1, 2, \dots, n\}$ and $\{1,2, \dots, m\}$.
    The wreath product of $\mathfrak{S}_n$ and $\mathfrak {S}_m$, denoted by $\mathfrak{S}_n[\mathfrak{S}_m]$ is the group acting on the set $\{1, \dots, m\}\times \{1, \dots, n\}$ as the symmetric group $\mathfrak{S}_m$ on every set $\{1, \dots, m\}\times\{i\}$ for every $i \in \{1, \dots, n\}$, and as the symmetric group $\mathfrak{S}_n$ on the set $\{1, \dots, n\}$, i.e. for $\sigma \in \mathfrak{S}_m$ and $\tau \in \mathfrak{S}_n$, and for all $i \in \{ 1, \dots, n \}$
    \[ \sigma[\tau] : \{1, \dots, m\} \times \{ i \} \longmapsto \{\sigma(1), \dots, \sigma(m)\} \times \{\tau(i)\}\]
\end{defn}

To clarify the definition we propose here an example.

\begin{es}
    Consider as in the definition $n=3$ and $m=4$, then the group $\mathfrak{S}_3[\mathfrak{S}_4]$ acts on the set $\{1,2,3,4\}\times\{1,2,3\}$. Consider the sets
    \[ \{1,2,3,4 \}\times\{1\} \hspace{12pt} \{1,2,3,4 \}\times\{2\} \hspace{12pt} \{1,2,3,4\}\times \{3\}\]
    then $\mathfrak{S}_3[\mathfrak{S}_4]$ is the group permuting each set as the action of $\mathfrak{S}_4$ and swap also the three sets between them like $\mathfrak{S}_3$.
    For examples some permutations are:
        \[ \{1,2,3,4 \}\times\{2\} \hspace{12pt} \{1,2,3,4 \}\times\{3\} \hspace{12pt} \{1,2,3,4\}\times \{1\}\]
        \[ \{1,4,2,3 \}\times\{3\} \hspace{12pt} \{1,2,3,4 \}\times\{1\} \hspace{12pt} \{1,2,4,3\}\times \{2\}\]
\end{es}

Consider now the matroid $\mathcal{M}$ with $\mathcal{L}_{\mathcal{M}} = \Pi_{12}$ and permutation group $G =$ Aut$(\mathcal{M})$, i.e. the group of all the automorphism on the matroid $\mathcal{M}$. 
Consider the building set $\mathcal{G}_{\min}$ and the nested set $N = \{F_1, F_2, F_3\}$ defined as
\[ F_1 = 1,2,3,4 \, | 5 | \, 6 | \, 7 | \, 8 | \, 9 | \, 10 | \, 11 | \, 12\] 
\[ F_2 = 5,6,7,8 \, | 1 | \, 2 | \, 3 | \, 4 | \, 9 | \, 10 | \, 11 | \, 12\]
\[ F_3 = 9,10,11,12 \, | 1 | \, 2 | \, 3 | \, 4 | \, 5 | \, 6 | \, 7 | \, 8.\]

One can simply prove that $N$ is a nested set computing the join 
\[F_1 \vee F_2 \vee F3 = 1,2,3,4 \, | \, 5,6,7,8 \, | \, 9,10,11,12\] 
which is not in $\mathcal{G}_{\min}$ because it has multiple non-singleton blocks.

In this setting $N^+ = \{F_1, F_2, F_3, E\}$ and so $m_{N^+}(F_i) = 3$ for $i = 1,2,3$, since every $F_i$ has rank 3 in $\Pi_{12}$. Moreover $F_1 \vee F_2 \vee F_3$ has rank $9$, $rk(E) = 11$, thus $m_{N^+}(E) = 2$.

The fiber $supp_+^{-1}(N^+)$ is depicted in Figure \ref{fig:fibersupp2}.

\begin{figure}
\begin{center} \begin{tikzpicture}
    \node (top) at (0,0) {$x_{F_1}^2x_{F_2}^2x_{F_3}^2x_E$};
            
    \node (n1) at (-4, -2)  {$x_{F_1}^2x_{F_2}^2x_{F_3}^2$};
            \node (n2) at (-1.4, -2)  {$x_{F_1}^2x_{F_2}^2x_{F_3}x_E$};
            \node (n3) at (1.4, -2)  {$x_{F_1}^2x_{F_2}x_{F_3}^2x_E$};
            \node (n4) at (4, -2)  {$x_{F_1}x_{F_2}^2x_{F_3}^2x_E$};

            \draw (top) -- (n1);
            \draw (top) -- (n2);
            \draw (top) -- (n3);
            \draw (top) -- (n4);

            \node (n5) at (-6.7, -5)  {$x_{F_1}^2x_{F_2}^2x_{F_3}$};
            \node (n6) at (-4, -5)  {$x_{F_1}^2x_{F_2}x_{F_3}^2$};
            \node (n7) at (-1.4, -5)  {$x_{F_1}x_{F_2}^2x_{F_3}^2$};
            \node (n8) at (1.4, -5)  {$x_{F_1}^2x_{F_2}x_{F_3}x_E$};
            \node (n9) at (4, -5) {$x_{F_1}x_{F_2}^2x_{F_3}x_E$};
            \node(n10) at (6.7, -5) {$x_{F_1}x_{F_2}x_{F_3}^2x_E$};

            \draw (n1) -- (n5);
            \draw (n1) -- (n6);
            \draw (n1) -- (n8);
            \draw (n2) -- (n5);
            \draw (n2) -- (n7);
            \draw (n2) -- (n9);
            \draw (n3) -- (n6);
            \draw (n3) -- (n7);
            \draw (n3) -- (n10);
            \draw (n4) -- (n8);
            \draw (n4) -- (n9);
            \draw (n4) -- (n10);
            
            
            \node (n11) at (-4, -8)  {$x_{F_1}^2x_{F_2}x_{F_3}$};
            \node (n12) at (-1.4, -8)  {$x_{F_1}x_{F_2}^2x_{F_3}$};
            \node (n13) at (1.4, -8)  {$x_{F_1}x_{F_2}x_{F_3}^2$};
            \node (n14) at (4, -8)  {$x_{F_1}x_{F_2}x_{F_3}x_E$};

            \draw (n5) -- (n11);
            \draw (n5) -- (n12);
            \draw (n6) -- (n11);
            \draw (n6) -- (n13);
            \draw (n7) -- (n11);
            \draw (n7) -- (n14);
            \draw (n8) -- (n12);
            \draw (n8) -- (n13);
            \draw (n9) -- (n12);
            \draw (n9) -- (n14);
            \draw (n10) -- (n13);
            \draw (n10) -- (n14);
            
            \node (bottom) at (0, -10) {$x_{F_1}x_{F_2}x_{F_3}$};

            \draw (bottom) -- (n11);
            \draw (bottom) -- (n12);
            \draw (bottom) -- (n13);
            \draw (bottom) -- (n14);
\end{tikzpicture} \end{center}
\caption{The Hasse diagram of $supp_+^{-1}(N^+)$.}
\label{fig:fibersupp2}

\end{figure}

\newpage
Following the decomposition of the fibers in product of chains shown in the proof, we have that $supp_+^{-1}(N^+) \simeq C_2 \times C_2 \times C_2 \times C_2$, 

Consider now $k = 4$. Since $11 = rk(E) = r +1$, we obtain $r = 10$, so since $k < \frac{r}{2}$, by Theorem \ref{mainthm} there should be an injective $G$-equivariant map $\lambda:FY^4 \longrightarrow FY^5$, i.e.
\[ \lambda(g \cdot m) = g \cdot \lambda (m), \,\text{ for all } m \in FY^4, \, g \in G.\]

We want now to show that $\lambda$ preserve stabilizing subgroup of the elements in $FY$.
Consider $m \in FY$ and define the set
\[ \text{Stab}(m) := \{ g \in G \, : \, g \cdot m = m\}.\]
Stab$(m)$ is a subgroup of $G$ since $e \cdot m = m$ and given $g,h \in $ Stab$(m)$
\[ (g h) \cdot m = g \cdot ( h \cdot m) = g \cdot m = m, \]
so $g  h  \in$ Stab$(m)$.
Given $g \in$ Stab$(m)$, we have by $G$-equivariance of $\lambda$
\[g \cdot \lambda(m) = \lambda(g \cdot m) = \lambda(m),\]
thus Stab$(m)$ = Stab$(\lambda(m))$. 
Hence $\lambda$ preserves stabilizing subgroup of elements in $FY$.

Observe now that the monomial $x_{F_1} x_{F_2} x_{F_3} x_E$ has a stabilizer subgroup isomorphic to $\mathfrak{S}_3[\mathfrak{S}_4]$. In fact the monomial $x_{F_1} x_{F_2} x_{F_3} x_E$ is stabilized when $g \in G$ permute the three sets 
\[ \{1,2,3,4 \} \hspace{12pt} \{ 5,6,7,8 \} \hspace{12pt} \{9,10,11,12 \}\]
and every element in each single set.
For example swapping $\{ 1,2,3,4 \}$ and $\{ 5,6,7,8 \}$ is equivalent to swap $F_1$ and $F_2$, instead permuting the elements of the set $\{1,2,3,4 \}$ is equivalent to send $F_1$ to itself.
It is also possible to prove that all monomials in $FY(\Pi_{12})$ which have stabilizer subgroup isomorphic to $\mathfrak{S}_3[\mathfrak{S}_4]$ lie in the fiber $supp^{-1}_+(N^+)$ described above. 


By definition of $\lambda$, the image $\lambda(x_{F_1} x_{F_2} x_{F_3} x_E)$ must lie among the degree 5 monomials in $supp^{-1}(N^+)$, however none of these degree 5 monomials have the same stabilizer subgroup isomorphic to $\mathfrak{S}_3[\mathfrak{S}_4]$ as $x_{F_1} x_{F_2} x_{F_3} x_E$.
For example, the permutation $g = (1,5,9)(2,6,10)(3,7,11)$ sends 
\[ F_1 \longmapsto F_2 \longmapsto F_3 \longmapsto F_1\]
fixing $x_{F_1} x_{F_2} x_{F_3} x_E$ but none of the degree 5 monomials in $supp^{-1}_+(N^+)$.
Thus $\lambda$ is not $G$-equivariant, and the Theorem \ref{mainthm} does not apply.

\nocite{feichtner2004conciniprocesiwonderfularrangementmodels}
\addcontentsline{toc}{chapter}{Bibliography}


\bibliographystyle{unsrt}
\bibliography{bibliography.bib}

\end{document}